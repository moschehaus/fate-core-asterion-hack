\chapter{Triky}
\label{chap:triky}

\section{Co a k čemu je trik}
\label{sec:coakcemu-trik}
Dovednosti postav jsou pořád poměrně obecné - tak například taková \dov{Mobilita} pokrývá běh, skok, lezení, plavání či uhýbání. Představme si dvě postavy, které mají \dov{Mobilitu} na stejné úrovni dovednosti: jedna z nich je mrštné dítě ulice, co se velmi hbitě dokáže pohybovat v městské zástavě a unikat dostižení; druhá je stepní hevren, který je vycvičen k dlouhým běhům a přelézání skal. Striktně vzato by ale dítě ulice mělo zvládat hevrenovu specilazaci na step a skály stejně dobře jako on - a hevren by uměl provozovat parkour stejně dobře jako dítě ulice (třebaže ve městě nikdy nebyl...).\\
K odlišení, jak která postava používá své dovednosti, slouží triky. Jsou to speciální způsoby či změna způsobů, které přidávají a upravují dosavadní fungování dovedností. Tím velmi konkretizují, co je postava zač: dítě ulice by v našem příkladu mohlo mít například trik \trk{Parkour} \ref{sec:mobilita-parkour} a najednou je zřejmé, že mezí jeho a hevrenovou \dov{Mobilitou} je ve skutečnosti velký rozdíl.\\
Tedy shrnuto: trik doplňuje, upravuje či rozvádí použití dovedností, a tedy vytváří charakteristický prvek pro každou postavu. Vedle aspektů a dovedností jsou triky to podstatné, co ji definuje.


\section{Efekty triků}
\label{sec:trik-efekty}
Jak jsme viděli výše, vymezení pojmu trik je poměrně obecné. K popsání pravidel triků je proto dobré (nutné) je nějak kategorizovat. Kategorie popsané níže jsou platné pro \textbf{základní} (nejčastější) triky - rozhodně nejsou vyčerpávající nebo ideální. Naopak, je dobré hráče podporovat v tom, aby se nenechali příliš svazovat těmito pravidly (Zlaté pravidlo!) a vymýšleli si triky vlastní. Kategorie níže pak představují kompromis mezi pravidlovou uchopitelností a volnou fantazií.\\
Obecně vzato trik může mít prakticky libovolná pravidla. Ty triky, které nespadají do kategorií níže, zde označujeme jako speciální triky a věnujeme si jim v kapitole věnované specialitám \ref{sec:specialni-triky}.\\
Některé triky jsou mocnější, pokročilejší než jiné, některé zase vyžadují předchozí zvládnutí jiného triku. Tyto vztahy pak popisuje sekce níže věnovaná rodinám triků \ref{sec:trik-rodiny}. 

\subsection{Přidání bonusu k akci}
\label{sec:trik-bonus}
Základní efekt triku je přidání bonusu k konkrétní akci za specifických podmínek. Ty by měly být poměrně úzké - hráč by neměl mít možnost použít trik na každou situaci (neměl by nahradit používaní jiných dovedností.)\\
Nejjednodušší způsob, jak specifikovat podmínky, je následující: postava dostane ``nějaký bonus'', když používá dovednost \dov{Dovednost} při akci \akc{Akce} a je u toho navíc ``něco splněno''. Pro představu uveďme několik příkladů a pak konkretizujme, co znamená ``nějaký bonus'' a ``něco splněno''.

\begin{itemize}
\item \trk{Kšeftař}:
  Získáte +2 na \akc{Vytvoření výhody} pomocí \dov{Vztahů}, když se snažíte vyjednat lepší cenu (zboží, služby, odměny).
\item \trk{Mistr převleků}:
  Získáte +2 na \akc{Vytvoření výhody} pomocí \dov{Klamu}, když se snažíte vytvořit si kostým či převlek.
\item \trk{Přivlastnění pasti}:
  Kdykoliv uspějete se stylem na \akc{Překonání} pomocí \dov{Zlodějiny} ve snaze zneškodnit past či nástrahu, můžete místo toho past předělat, aby sloužila ve váš prospěch, čímž získáte výhodu \asp{Chráněno pastí} s pasivní +2 opozicí na překonání.
\item \trk{Bytelná konstrukce}:
  Jakýkoliv výrobek získáný \dov{Řemesly} pomocí \akc{Vytvoření výhody} sloužící k něčemu obranému (zbroj, štít, zátarasa, palisáda apod.) získává automaticky bonus +2 k pasivní opozice, jež je nutné překonat, aby byl zničen.
\item \trk{Navalení se štítem}:
  Kdykoliv máte štít a používáte \dov{Mobilitu} s \akc{Překonáním} k přesunu do jiné zóny a uspějete, můžete do protivníka vrazit v rychlosti štítem a způsobit mu tak dvouposunový stres.

\end{itemize}
``Nějaký bonus'' může být cokoliv, co lze efektivně převést na bonus +2 k hodu:

\begin{itemize}
\item jednoduše +2 k hodu (viz \trk{Kšeftař}, \trk{Mistr převleků})
\item vytvoření +2 pasivní opozice (viz \trk{Bytelná konstrukce})
\item vytvoření výhody (aspektu), jejíž odstranění vyžaduje úspěch proti +2 pasivní opozici (viz \trk{Přivlastnění pasti})
\item obdoba 2 posunového zásahu (jež je nutné pokrýt stresem či následkem) (viz \trk{Navalení se štítem})
\end{itemize}

``Něco splněno'' dále specifikuje, v jakých příběhových podmínkách lze trik využít:

\begin{itemize}
\item když má postava vhodné vybavení (viz štít u \trk{Navalení se štítem})
\item když postava využívá trik proti určitému protivníkovi (například proti myšlenkovým bytostem)
\item když se postava účastní nějakého typu \per{Peripetie} (například pouze ve \per{Střetu})
\item když se před použitím triku něco (ne)přihodilo (postava neobdržela žádné zranění, nepoužila trik v témže sezení apod.)
\end{itemize}

\subsection{Přidání nové akce k dovednosti}
\label{sec:trik-pridani}

Zajímavým efektem triku může být přidání nějaké akce k dovednosti, jež předtím toto použití neumožňovala (vzpomeňte na to množství vět ``Tuto dovednost nelze použít k akci ...'' u popisu dovedností v \ref{sec:seznam-dov}). I zde je ovšem potřeba blíže specifikovat, za jakých podmínek lze trik použít: pořád totiž nechceme, aby trik zcela nahradil nějakou jinou dovednost.  Na začátek opět uveďme pár příkladů:

\begin{itemize}
\item \trk{Skrytý útok}:
  Můžete použít dovednost \dov{Skrývání} na provedení \akc{Útoku}, jestliže o vás váš cíl neví.
\item \trk{Nikdy nejste v bezpečí}:
  Můžete použít dovednost \dov{Zlodějina} k provádění duševních \akc{Útoků} a \akc{Vytváření výhod}, pokud provedete loupež takovým způsobem, že donutí oběť pochybovat o vlastní bezpečnosti.
\item \trk{Zbraně vlastní výroby}:
  Můžete použít dovednost \dov{Řemesla} na akce \akc{Útok, Obrana} namísto \dov{Boj se zbraní} resp. \dov{Boj na dálku}, jestliže k boji používate svépomocí vyrobené zbraně.
\end{itemize}

Ona dodatečná podmínka ``jestliže o vás váš cíl neví/pokud provedete loupež takovým způsobem, že.../jestliže používate svépomocí vyrobené zbraně...'' zajišťuje, že přidáním nové akci k dovednosti efektivně neodstraníme jinou dovednost ze hry; to je velmi podstatné jednak pro vyváženost, druhak pro zábavu - bylo by nuda, když by postava dokázala všechno vyřešit jedinou dovedností, ke které má mnoho triků.

\subsection{Vytvoření výjimky z pravidel}
\label{sec:trik-vyjimky}
Posledním a nejobecnějším způsobem, jak může trik pravidla upravovat, je vytvoření nějaké výjimky. Teoreticky může vytvářet výjimku z libovolného pravidla; platí ale, že čím základnější pravidlo ovlivňuje, tím těžší je dohlédnout, jaký efekt tento trik bude mít na dynamiku celé hry. Uveďme několik typických výjimek, které triky mohou dovolovat:

\begin{itemize}
\item použít stejnou dovednost vícekrát během jediné \per{Výzvy} (běžně by bylo možné použít jí pouze jedinkrát)
\item nerozdělovat posuny při akci proti skupině, ale házet proti každému zvlášť
\item házet si na aktivní opozici, i když by běžně bylo logické použít opozici pasivní
\item umožnit zahrát více akcí v jednom kole
\item vynutit přijmutí následku oproti stresu
\end{itemize}

V žádném případě se nejedná o konečný výčet (jiné efekty jsou mj. popsány v seznamech triků níže); uveďme i přesto několik příkladů

\begin{itemize}
\item \trk{Stroj na přemýšlení}:
  Můžete použít dovednost \dov{Učenost} dvakrát během jedné \per{Výzvy}.
\item \trk{Svazovač}:
  Kdykoliv na někom vytvoříte výhodu pomocí \dov{Řemesel} tím, že jej nějak spoutáte, smíte proti němu vždy házet na aktivní opozici, když se pokouší osvobodit.
\item \trk{Mnohonásobný výstřel}:
  Pomocí dovednosti \dov{Boj na dálku} se můžete pokusit zasáhnout více nepřátel v téže zóně za předpokladu, že na každého z nich máte neskrytý výhled (nemusíte se přesouvat i v rámci zóny); na každý další terč kromě prvního má útočník postih -2 k hodu.
\end{itemize}

\section{Rodiny triků}
\label{sec:trik-rodiny}

Jak uvidíme později, ve Fate Core neexistuje nic jako povolání známé nejen z pravidel M16 (čaroděj, bojovník, hraničář, střelec, hrdlořez, theurg, alchymista, zloděj, psionik, kněz atd.) To může být ostatně v mnohém výhoda; bez rigidního rozdělení si každý hráč může vytvořit povolání zcela dle svých potřeb. \\
I tak je ovšem běžné, že nějaká postava má své triky hodně podobné; to může například vycházet z tréninku či velmi úzkého zaměření (zabiják nestvůr). Způsob, jak se vypořádat s blízkými triky, je právě rodina triků. Všechny triky se společným jmenovatelem se spojí do rodiny (stromu) a ty jsou pak mezi sebou svázané. Podmínkou je, že k získání hlubšího, pokročilejšího triku je vždy potřeba vlastnit nějaký základnější.

\subsection{Skládaní efektů (``do hloubky'')}
\label{sec:trik-skladani}
První možnost je složit efekty a trik tak rozvinout ``do hloubky''. Tím se trik stane ještě silnější v původním efektu:

\begin{itemize}
\item jestliže trik přidával bonus k akci, ten další může přidávat ještě větší bonus (či nějaký dodatečný efekt odpovídající dvou posunům)
\item jestliže trik přidával akci k dovednosti, přidejte k dovednosti další akci
\item jestliže trik umožňoval získat výjimku z pravidel, udělejte výjimku ještě výjimečnější
\end{itemize}

\textbf{DOPLNIT PŘÍKLADY}

\subsection{Větvení efektů (``do šíře'')}
\label{sec:trik-vetveni}
Jinou možností je efekty větvit - trik rozvinout ``do šíře''. To pak znamená, že k jednomu efektu (přidání bonusu, přidání akce, vytvoření výjimky) se přidá nový jiný efekt. Obecně vzato se efekty mohou doplňovat: jestliže například trik umožňuje použít \dov{Zlodějinu} pro \akc{Útok} a větvený trik přídává +2 k \dov{Zlodějině} (za nějakých podmínek) použitou na \akc{Útok}, pak lze použít \dov{Zlodějinu} na \akc{Útok} (díky 1. triku) s bonusem +2 (díky druhému triku).

\textbf{DOPLNIT PŘÍKLADY}

\section{Základní seznam triků}
\label{sec:trik-zakladni}

\subsection{Boj beze zbraně}
\label{sec:trik-bbz}
\begin{itemize}
\item\trk{Jau, to bolí}:
  \label{sec:bbz-jau}
  Můžete použít dovednost \dov{Boj beze zbraně}, namísto \dov{Provokace} abyste někoho ``přesvědčili'' pomocí vyhrožování fyzickým násilím (doplněné o naznačení jako chycení za flígr apod.); to lze použít, pokud se vás protivník může fyzicky bát: tedy na drobounkého úředníka \trk{Jau, to bolí} použít lze, byť ho chrání zákon, ale na velezkušeného žoldáka (nebo zkrátka žoldáka doplněného o početnou družinu) toto fungovat nebude.
\item\trk{Útok na slabiny}:
  \label{sec:bbz-slabiny}
  Získáváte bonus +2 na akce \akc{Vytváření výhody} pomocí \dov{Boje beze zbraně} kdykoliv si chcete pomoci efektivními (ale trochu špinavými) útoky na slabá místa nepřátel: šťourání do očí, kopání do rozkroku, přímé kopy na kolena, útoky na ohryzek apod. K tomu je ovšem potřeba slabá místa znát - u humanoidů lze použít např. ta zmiňovná, u jiných bytostí může být situace komplikovanější.
\item\trk{Férovka}:
  Jestliže se vám podaří vytvořit na nepříteli používající zbraň výhodu odrážející skutečnost, že jste u něj mnohem blíže než je efektivní vzdálenost pro používání jeho zbraně (aspekty jako \asp{Nalepený, Nabalený, Nevzdálím se o krok}), můžete jej za vyvolání tohoto aspektu donutit, aby na svůj \akc{Útok}/\akc{Obranu} použil \dov{Boj beze zbraně} namísto \dov{Boje se zbraní} (v běžném případě by vám výhoda dávala klasický efekt vyvolání aspektu, ale neumožnila diktovat, jakou dovednost protivník použije).
\end{itemize}

  
\subsection{Boj na dálku}
\label{sec:trik-bnd}
\begin{itemize}
\item\trk{Jeden terč}:
\label{sec:bnd-odstrelovac}
Získáváte bonus +2 na \dov{Vytvoření výhody} reprezentující zamíření proti protivníkovi, kterého jste už alespoň jednou úspěšne zasáhli. Takto lze výhodně zaměřovat postupně nanejvýše dva protivníky za konflikt.

\item\trk{Mnohonásobný výstřel}:
\label{sec:bnd-mnohonasobny}
Pomocí dovednosti \dov{Boj na dálku} se můžete pokusit zasáhnout více nepřátel v téže zóně za předpokladu, že na každého z nich máte neskrytý výhled (nemusíte se přesouvat i v rámci zóny); na každý další terč kromě prvního má útočník postih -2 k hodu. Toto lze použít pouze se zbraněmi, které nevyžadují 1 kolo nabíjení.

\item \trk{Odstřelovač}:
\label{sec:bnd-odstrelovac}
Smíte útočit dovedností \dov{Boj na dálku} i na vzdálenější protivníky. Bez postihu lze střílet na postavy až 2 zóny vzdálené\footnote{Tedy na postavy v téže zóně, vedlejší zóně a ještě zóně o jedna vzdálené od té vedlejší.}, střelba do vzdálenější zón je pak zatížena postihem -2/zóna navíc. 

\end{itemize}
  
\subsection{Boj se zbraní}
\label{sec:trik-bsz}
\begin{itemize}
\item\trk{Odzbrojení}:
  Získáváte bonus +2 pro pokusy o \akc{Vytvoření výhody} pomocí odzbrojení svého protivníká; k tomu je potřeba, aby protivník měl v ruce nějakou zbraň (tedy takto samozřejmě nelze odzbrojit nestvůru, co útočí svými tesáky). Po úspěšném odzbrojení se může protivník svou zbraň získat zpět: jestliže útočník nevěnuje své kolo snaze tomu zabránit, zbraň získává v rámci téhož kola zpět; útočník se ovšem může rozhodnout mu v tom aktivně bránit: v takovém případě je to jeho akce a používají se relevantní dovednosti jako \dov{Zlodějina, Pozornost, Mobilita, Kondice}.

\item\trk{Úder hruškou}:
\label{sec:bsz-hruska}
Jestliže jste v situaci, kdy byste nemohli běžně používat svoji zbraň\footnote{Poznamenejme, že v takové situaci mají jednoruční zbraně postih k hodnocení zbraně -1 a obouruční dokonce -2.} (v mačkanici lidí, v ztísněném prostoru), můžete dvakrát za konflikt zasáhnout svého protivníka hruškou, čímž zvýšíte závažnost zásahu o +2.

\item\trk{Štítař}:
  \label{sec:bsz-stitar}
Jestliže používate štít při \akc{Plné obraně} získáváte bonus +4 oproti běžným +2 při obraně proti útokům vedeným zepředu/z boku protivníky, o kterých máte povědomí; tedy takto se nelze bránit útokům z úkrytu či zezad. 
\end{itemize}



\subsection{Empatie}
\label{sec:trik-empatie}

\begin{itemize}
\item\trk{Odhalování lží}:
\label{sec:empatie-odhalovani} Získáváte bonus +2 na všechny hody, které slouží k odhalení lží - ať už cílené na vás či na někoho jiného.

\item\trk{Cit pro potíže}:
\label{sec:empatie-cit} Můžete použít \dov{Empatii} k určení iniciativy ve fyzických konfliktech (namísto \dov{Pozornosti}), jestliže máte během scény (či před ní) dostatek příležitostí s účastníky konfliktu alespoň chvíli mluvit.

\item\trk{Terapeut}:
\label{sec:empatie-terapuet} Jednou za sezení můžete duševně zraněné postavě snížit závažnost jednoho neextrémního následku o jednu úroveň (odstranit Drobný, snížit Mírný na Drobný či snížit Vážný na Mírný). K tomu je potřeba s postavou odehrát rozhovor a uspět na hod na \dov{Empatii}, přičemž pasivní opozice je dle tabulky níže:
\end{itemize}

\begin{table}[h]  
\centering
\begin{tabular}[h]{c|c}
Následek & Pasivní opozice \\ \hline
Drobný (+2) -> žádný & +2 \\
Mírný (+4) -> Drobný (+2) & +3 \\
Vážný (+6) -> Mírný (+4) & +4\\
\end{tabular}
\end{table}

\subsection{Jezdectví}
\label{sec:trik-jezd}
\begin{itemize}
  
\item\trk{Klíště}:
\label{sec:jezdectvi-kliste}
Získáváte bonus +2 kdykoliv během honičky/závodu pronásledujete jiného jezdce.

\item\trk{Hyjé!}:
\label{sec:jezdectvi-hyje}
Ze svého oře dokážete vyždímat více, než se zdá možné. Kdykoliv v při akci, která je zejména o rychlosti, remízujete, je to jako kdybyste uspěli.

\item\trk{Zvíře je kámoš}:
\label{sec:jezdectvi-kamos}
Dvakrát za sezení můžete zvířeti ``dát'' jednoduchý rozkaz a ono jej provede - a to i v případě, že u něj nejste. Typicky to může být ``počkej tady'', ``přijď za mnou'', ``dojdi tam'' apod.

\end{itemize}

\subsection{Klam}
\label{sec:trik-klam}

\begin{itemize}
\item\trk{Lež za lží}:
\label{sec:klam-lez}
Získáváte bonus +2 při \akc{Vytváření výhody} pomocí \dov{Klamu} proti někomu, kdo už během tohoto sezení nějaké z vašich lží uvěřil.

\item\trk{Myšlenkové hry}:
\label{sec:klam-mysl}
Můžete použít dovednost \dov{Klam} namísto \dov{Provokace} k provedení duševního útoku.

\item\trk{Muž mnoha tváří}:
\label{sec:klam-muz}
Kdykoliv potkáte někoho nového, můžete prohlásit, že jste jej už potkali - ale pod jinou (smyšlenou) identitou. Při interakci s touto postavou pak lze využít \dov{Klam} namísto \dov{Vztahů}; k tomu by si postava měla vytvořit situační aspekt reprezentující onu identitu.
\end{itemize}

\subsection{Kondice}
\label{sec:trik-kondice}

\begin{itemize}
\item\trk{Zápasník}:
\label{sec:kondice-zapas}
Získáváte bonus +2 na akce \akc{Vytvoření výhody}, pokud se snažíte nepřítele chytit, lapit či zaklesnout pomocí \dov{Kondice}.

\item\trk{Drsňák}:
\label{sec:kondice-drsnak}
Můžete použít dovednost \dov{Kondice} k akci \akc{Obrana} proti \dov{Boji beze zbraně} či \dov{Boji se zbraní}, pokud je zbraň tupá. V případě remízy však dostáváte zranění o velikosti jednoho posunu.

\item\trk{Nezdolný}:
\label{sec:kondice-nezdolny}
Jednou za sezení může postava přeměnit Mírný (+4) následek na Drobný (+2) (pokud jej nemá zaškrtnutý) či se úplně Drobného následku zbavit.
\end{itemize}

\subsection{Kontakty}
\label{sec:trik-kontakty}

\begin{itemize}
\item\trk{Drbna}:
\label{sec:kontakty-drbna}
Získáváte bonus +2, na akce \akc{Překonání} či \akc{Vytvoření výhody} pomocí rozšiřování pomluv dovedností \dov{Kontakty}.

\item\trk{Ucho v pozoru}:
\label{sec:kontakty-ucho}
Můžete použít dovednost \dov{Kontakty} k určení iniciativy ve fyzickém konfliktu (namísto \dov{Pozornosti}), pokud se konflikt odehrává v prostředí, ve kterém máte vybudouvanou síť.

\trk{Síla reputace}:
\label{sec:kontakty-reputace}
Můžete použít dovednost \dov{Kontakty} na akci \akc{Vytvoření výhody} namísto \dov{Provokace}, kdykoliv se snažíte vyvolat v ostatních strach svojí zlověstnou reputací. S tímto trikem by měl být spojený vhodný aspekt.
\end{itemize}

\subsection{Medicína}
\label{sec:trik-medicina}
\begin{itemize}
  
\item\trk{Ambulance}:
  \label{sec:medicina-ambulance}
Získáváte bonus +2 k hodům na \dov{Medicínu}, pokud pracujete v podmínkách aspoň trochu uzpůsobeným na lékařskou práci - lampa, stůl, čisté nástroje. Je dobré poznamenat, že v běžných případech by toto k získání bonusu (resp. hodu na \akc{Vytváření výhody}) nestačilo - to by si vyžadovalo opravdovou ambulanci.

\item\trk{Výživový specialista}:
  \label{sec:medicina-vyziva}
Můžete použít \dov{Medicínu} namísto \dov{Vyšetřování} nebo \dov{Pozornosti} abyste rozpoznali, zda-li je poživatina otrávená. Když jste otráveni jedem v jídle či pití, můžete namísto \dov{Kondice} použít \dov{Medicínu}, abyste jedu odolali.

\item\trk{Bojový doktor}:
  \label{sec:medicina-boj}
  Za bod osudu se můžete pokusit hodit na \dov{Medicínu} a ošetřit postavu i během probíhajícího konfliktu. Takto lze vyléčit pouze stres a to v celkové hodnotě rovné výsledku hodu, přičemž základní opozice je pasivní a rovna +0, nicméně může být jiná (včetně aktivní) na základě okolností.
\end{itemize}

\subsection{Mobilita}
\label{sec:trik-mobilita}
\begin{itemize}
  
\item\trk{Parkour}:
\label{sec:mobilita-parkour}
Získáváte bonus +2 na akce \akc{Překonání} pomocí \dov{Mobility} pokud postava hopsá po střechách či v jiné zástavbě.

\item\trk{Sprinter}:
\label{sec:mobilita-sprinter}
Můžete se posunout až o dvě zóny zdarma (namísto jedné) v rámci fyzického konfliktu, pokud tomu nebrání žádný aspekt prostředí.


\item\trk{Hbité zmizení}:
\label{sec:mobilita-zmizeni}
Můžete použít dovednost \dov{Mobilita} namísto \dov{Skrývání}, kdykoliv vám bleskurychlé pohyby pomohou dostat se na místo, kde nejste vidět. To dobře funguje např. mezi lidmi či objekty, ale nikoli na otevřené ploše nebo při statickém kamuflování.
\end{itemize}

\subsection{Pozornost}
\label{sec:trik-pozornost}
\begin{itemize}

\item\trk{Čtenář těla}:
\label{sec:pozornost-ctenar}
Můžete použít dovednost \dov{Pozornost} namísto \dov{Empatie} či \dov{Vztahů}, abyste odhalili nějaký aspekt postavy, pakliže máte čas postavu nějakou dobu pozorovat.

\item\trk{Střelba od boku}:
\label{sec:pozornost-strelba}
Můžete použít \dov{Pozornost} namísto \dov{Boje na dálku} abyste provedli rychlý instinktivní výstřel od boku (nejste např. v situaci, kdy ležíte zamířen na terč).

\item\trk{Šestý smysl}:
\label{sec:pozornost-smysl}
Kdykoliv někdo proti postavě s tímto trikem háže proti \dov{Pozornosti} s pasivní opozicí, je tato pasivní opozice zvýšena o +2; to může nastat například v pokusech o krádež, útok ze zálohy apod.
\end{itemize}




\subsection{Provokace}
\label{sec:trik-provokace}
\begin{itemize}
  
\item\trk{Zbroj ze strachu}:
\label{sec:provokace-zbroj}
Můžete použít \dov{Provokaci} k akci \akc{Obrana} abyste se vyhli fyzickým útokům dokud poprvé neobdržíte zranění. Protivníky dokážete tak znejistit v \akc{Útoku} na vás, že si to raději rozmyslí; jakmile vás ale poprvé zraní, stejný efekt již nenastane.

\item\trk{Cos to řekl?!}:
\label{sec:provokace-cos}
Když na protivníkovi vytvoříte výhodu pomocí \dov{Provokace}, můžete ho pomocí onoho volného vyvolání přinutit, aby svoji další akci směřoval na vás - například \akc{Útok}.

\item\trk{Šťouřání}:
\label{sec:provokace-stourani}
Můžete použít dovednost \dov{Provokace} namísto \dov{Vztahů} či \dov{Empatie}, abyste odhalili aspekt na kartě postavy tím, že ji provokujete a sledujete její reakci. Cíl se klasicky brání \dov{Vůlí}.
\end{itemize}

\subsection{Řemesla}
\label{sec:trik-remesla}
\begin{itemize}
  
\item\trk{Vždy připraven}:
\label{sec:remesla-pripraven}
V jakoukoliv situaci máte vhodné nástroje na provedení (rozumné) řemeslné práci - i kdybyste byli od svých věcí odděleni. To například znamená, že pokácet strom a nastražit past můžete i bez sekery a pily, ale těžko tak vyrobíte obléhací stroj.

\item\trk{Lepší než nový}:
\label{sec:remesla-lepsi}
Kdykoliv na akci \akc{Překonání} při opravování nějakého stroje uspějete se stylem, namísto běžného posílení můžete na stroj umístit nový aspekt reprezentující nějaký upgrade.

\item \trk{Zbraně vlastní výroby}:
\label{sec:remesla-zbrane}
 Můžete použít dovednost \dov{Řemesla} na akce \akc{Útok, Obrana} namísto \dov{Boj se zbraní} resp. \dov{Boj na dálku}, jestliže k boji používate svépomocí vyrobené zbraně.

\end{itemize}



\subsection{Skrývání}
\label{sec:trik-skryvani}
\begin{itemize}
  
\item\trk{Jen tvář v davu}:
\label{sec:skryvani-tvar}
Získáváte bonus +2 kdykoliv se snažíte zmizet v nějakém zalidněném prostředí - náměstí, trh, místnost plná lidí. Co je a není zalidněné prostředí záleží na okolnostech.

\item\trk{Nejasný cíl}:
\label{sec:skryvani-nejasny}
Za předpokladu, že jste ve tmě, stínu či jinak těžko spatřitelní, můžete použít \dov{Skrývání} na \akc{Obranu} proti útokům pomocí \dov{Boji na dálku} z jiných zón.

\item\trk{Kam zmizel?!}:
\label{sec:skryvani-zmizel}
  Za zaplacení bodu osudu zmízíte z okamžitého povědomí všech postav - ztratíte se z dohledu, jste odstraněni z herní mapy a nikdo neví, jaké bude vaše další počínání. To lze ve většině kontextů, kdy je prostředí alespoň trochu strukturované; například takto nelze zmizet uprostřed prázdné zamknuté malé místnosti nebo široširého oceánu.

\item\trk{Skrytý útok}:
\label{sec:skryvani-utok}
 Smíte používat dovednost \dov{Skrývání} na provedení \akc{Útoku}, pokud o vás vaše oběť neví (ve smyslu, že neví o vaší přítomnosti - pokud se mu jen dostanete do zad, nelze v tomto případě \trk{Skrytý útok} použít.)
\end{itemize}

\subsection{Učenost}
\label{sec:trik-ucenost}
\begin{itemize}
  
\item\trk{Specialista}:
\label{sec:ucenost-specialista}
Můžete si zvolit jeden obor poznání, ve kterém se specializujete. Na všechny hody na \dov{Učenost} z tohoto oboru pak získáváte bonus +2.

\item\trk{To nedává smysl}:
\label{sec:ucenost-smysl}
Můžete použít dovednost \dov{Učenost} na obranu před akcemi dovednosti \dov{Provokace}, pokud jste schopni situaci racionalizovat či v ní nalézt logické nekonzistence. Například u vyverního řevu vám toto nepomůže...

\item\trk{O tom jsem četl!}:
\label{sec:ucenost-cetl}
Za utracení bodu osudu můžete prohlásit, že přesně o této situaci jste již četl a použít tak \dov{Učenost} namísto libovolné jiné dovednosti pro účel hodu nebo výměny. Přitom musíte být schopen ospravedlnit, že o oné situaci se skutečně píše: to může fungovat například při boji s myšlenkovou bytostí a využití nějaké její specifické slabiny, ale těžko při odolávání efektu jedu.
\end{itemize}

\subsection{Vůle}
\label{sec:trik-vule}
\begin{itemize}
  
\item\trk{Nebojsa}
\label{sec:vule-nebojsa}:
Získáváte bonus +2 při \akc{Obraně} proti dovednosti \dov{Provokace} použité na zastrašování či jiné vyvolávání strachu.

\trk{Duch nad tělem}:
\label{sec:vule-duch}
Můžete použít dovednost \dov{Vůle} namísto \dov{Kondice} v situacích, kdy je třeba vyvinout velké fyzické úsilí.

\item\trk{Mně nelze ublížit}:
\label{sec:vule-ublizit}
Ve scéně se můžete rozhodnout ignorovat libovolný následek - ten zůstává zaškrtnutý, ale nelze jej vyvolat ani vynutit žádnou postavou. Po skončení scény se však projeví o to více - jeho závažnost o jedna vzroste, tedy z Drobného na Mírný, z Mírného na Vážný a z Vážného na Extrémní.
\end{itemize}

\subsection{Vyšetřování}
\label{sec:trik-vysetrovani}
\begin{itemize}

\item\trk{Mikrovýrazy}:
\label{sec:vysetrovani-mikro}
Můžete použít dovednost \dov{Vyšetřování} namísto \dov{Empatie} či \dov{Vůle}, abyste se bránili dovednosti \dov{Klam} pečlivým čtením řeči těla protivníka.

\item\trk{Fízlování}:
\label{sec:vysetrovani-fizlovani}:
Po úspěšném \akc{Vytvoření výhody} pomocí \dov{Vyšetřování} při odposlouchávání rozhovoru můžete vytvořit o jeden aspekt navíc; nemáte na něj sice volné vyvolání, ale taky tím získáváte vypravěčskou pravomoc o detailech konverzace.


\item\trk{Detektiv}:
\label{sec:vysetrovani-detektiv}
Získáváte bonus +2, kdykoliv používáte \dov{Vyšetřování} abyste prošetřili místo činu (potyčka, vražda, krádež, vloupání) za účelem získání nějakého detailu či povědomí, co se na místě dělo.

\item\trk{Stopování}:
\label{vysetrovani-stopovani}
Získáváte +2, jestliže používáte \dov{Vyšetřování} k naleznutí či sledování stop (nejen doslova například v bahně) a vodítek.

\end{itemize}
\subsection{Vztahy}
\label{sec:trik-vztahy}
\begin{itemize}

\item\trk{Miláček publika}:
  \label{sec:vztahy-demgagog}
  Získáváte bonus +2 kdykoliv používáte \dov{Vztahy} při projevu před davem.

\item\trk{Individuální péče}:
  \label{sec:vztahy-individual}
  Při používání \dov{Vztahů} na více postav nemusíte svůj hod rozdělovat mezi všechny postavy, ale na každou si můžete hodit zvlášť za předpokladu, že alespň část sdělení je relevantní pro každého z posluchačů.

\item\trk{Pití pro všechny!}:
  \label{sec:vztahy-piti}
  Dvakrát za sezení můžete povýšit posílení získané při použití \dov{Vztahů} na plnohodnotný situační aspekt.

\item\trk{Populární}:
  \label{sec:vztahy-popularni}
  Jestliže jste v oblasti, kde jste oblíbený, můžete používat \dov{Vztahy} namísto \dov{Kontaktů} (na naleznutí správného člověka apod.) K tomu je potřeba mít vhodný (situační) aspekt, jež tuto oblíbenost odráží, nebo zaplatit bod osudu.
  
\end{itemize}

\subsection{Zlodějina}
\label{sec:trik-zlodejina}

\begin{itemize}
\item\trk{Zadní vrátka}:
  \label{sec:zlodejina-vratka}
  Získáváte bonus +2 na \dov{Zlodějinu}, kdykoliv ji používáté s akcí \akc{Překonání} abyste opustili lokaci (zónu), tedy například vypáčili dveře \textbf{ven}.

\item\trk{Bezpečnostní odborník}:
  \label{sec:zlodejina-odbornik}
  Můžete poskytovat aktivní opozici kdykoliv se někdo snaží překonat vaše ``bezpečnostní zařízení'' - např. past, nástrahu, a to i když u toho přímo nejste. Bez tohoto triku by se házelo proti pasivní opozici.

\item\trk{Co se šeptá}:
  \label{sec:zlodejina-septani}
  Při jednání se zloději a podobnými můžete používat \dov{Zlodějinu} místo \dov{Kontaktů}.
\end{itemize}
  



%%% Local Variables:
%%% mode: LaTeX
%%% TeX-master: "../main"
%%% End:
