\chapter{Triky}
\label{chap:triky}

\section{Co a k čemu je trik}
\label{sec:coakcemu-trik}

\section{Efekty triků}
\label{sec:trik-efekty}

\subsection{Přidání nové akce k dovednosti}
\label{sec:trik-pridani}

\subsection{Přidání bonusu k akci}
\label{sec:trik-bonus}

\subsection{Vytvoření výjimky z pravidel}
\label{sec:trik-vyjimky}

\section{Rodiny triků}
\label{sec:trik-rodiny}

\subsection{Větvení efektů}
\label{sec:trik-vetveni}

\subsection{Skládaní efektů}
\label{sec:trik-skladani}

\section{Základní seznam triků}
\label{sec:trik-zakladni}

\subsection{Boj beze zbraně}
\label{sec:trik-bbz}
\begin{itemize}
  \item
\end{itemize}
  
\subsection{Boj na dálku}
\label{sec:trik-bnd}
\begin{itemize}
  \item
\end{itemize}
  
\subsection{Boj se zbraní}
\label{sec:trik-bsz}
\begin{itemize}
  \item
\end{itemize}

\subsection{Empatie}
\label{sec:trik-empatie}

\begin{itemize}
\item\trk{Odhalování lží}:
\label{sec:empatie-odhalovani} Získáváte bonus +2 na všechny hody, které slouží k odhalení lží - ať už cílené na vás či na někoho jiného.

\item\trk{Cit pro potíže}:
\label{sec:empatie-cit} Můžete použít \dov{Empatii} k určení iniciativy ve fyzických konfliktech (namísto \dov{Pozornosti}), jestliže máte během scény (či před ní) dostatek příležitostí s účastníky konfliktu alespoň chvíli mluvit.

\item\trk{Terapeut}:
\label{sec:empatie-terapuet} Jednou za sezení můžete duševně zraněné postavě snížit závažnost jednoho neextrémního následku o jednu úroveň (odstranit Drobný, snížit Mírný na Drobný či snížit Vážný na Mírný). K tomu je potřeba s postavou odehrát rozhovor a uspět na hod na \dov{Empatii}, přičemž pasivní opozice je dle tabulky níže:
\end{itemize}

\begin{table}  
\centering
\begin{tabular}[h]{c|c}
Následek & Pasivní opozice \\ \hline
Drobný (+2) -> žádný & +2 \\
Mírný (+4) -> Drobný (+2) & +3 \\
Vážný (+6) -> Mírný (+4) & +4\\
\end{tabular}
\end{table}

\subsection{Jezdectví}
\label{sec:trik-jezd}
\begin{itemize}
  
\item\trk{Klíště}:
\label{sec:jezdectvi-kliste}
Získáváte bonus +2 kdykoliv během honičky/závody pronásledujete jiného jezdce.

\item\trk{Hyjé!}:
\label{sec:jezdectvi-hyje}
Ze svého oře dokážete vyždímat více, než se zdá možné. Kdykoliv v při akci, která je zejména o rychlosti, remízujete, je to jako kdybyste uspěli.

\item\trk{Zvíře je kámoš}:
\label{sec:jezdectvi-kamos}
Dvakrát za sezení můžete zvířeti ``dát'' jednoduchý rozkaz a ono jej provede - a to i v případě, že u něj nejste. Typicky to může být ``počkej tady'', ``přijď za mnou'', ``dojdi tam'' apod.

\end{itemize}

\subsection{Klam}
\label{sec:trik-klam}

\begin{itemize}
\item\trk{Lež za lží}:
\label{sec:klam-lez}
Získáváte bonus +2 při \akc{Vytváření výhody} pomocí \dov{Klamu} proti někomu, kdo už během tohoto sezení nějaké z vašich lží uvěřil.

\item\trk{Myšlenkové hry}:
\label{sec:klam-mysl}
Můžete použít dovednost \dov{Klam} namísto \dov{Provokace} k provedení duševního útoku.

\item\trk{Muž mnoha tváří}:
\label{sec:klam-muz}
Kdykoliv potkáte někoho nového, můžete prohlásit, že jste jej už potkali - ale pod jinou (smyšlenou) identitou. Při interakci s touto postavou pak lze využít \dov{Klam} namísto \dov{Vztahů}; k tomu by si postava měla vytvořit situační aspekt reprezentující onu identitu.
\end{itemize}

\subsection{Kondice}
\label{sec:trik-kondice}

\begin{itemize}
\item\trk{Zápasník}:
\label{sec:kondice-zapas}
Získáváte bonus +2 na akce \akc{Vytvoření výhody}, pokud se snažíte nepřítele chytit, lapit či zaklesnout pomocí \dov{Kondice}.

\item\trk{Drsňák}:
\label{sec:kondice-drsnak}
Můžete použít dovednost \dov{Kondice} k akci \akc{Obrana} proti \dov{Boji beze zbraně} či \dov{Boji se zbraní}, pokud je zbraň tupá. V případě remízy však dostáváte zranění o velikosti jednoho posunu.

\item\trk{Nezdolný}:
\label{sec:kondice-nezdolny}
Jednou za sezení může postava přeměnit Mírný (+4) následek na Drobný (+2) (pokud jej nemá zaškrtnutý) či se úplně Drobného následku zbavit.
\end{itemize}

\subsection{Kontakty}
\label{sec:trik-kontakty}

\begin{itemize}
\item\trk{Drbna}:
\label{sec:kontakty-drbna}
Získáváte bonus +2, na akce \akc{Překonání} či \akc{Vytvoření výhody} pomocí rozšiřování pomluv dovedností \dov{Kontakty}.

\item\trk{Ucho v pozoru}:
\label{sec:kontakty-ucho}
Můžete použít dovednost \dov{Kontakty} k určení iniciativy ve fyzickém konfliktu (namísto \dov{Pozornosti}), pokud se konflikt odehrává v prostředí, ve kterém máte vybudouvanou síť.

\trk{Síla reputace}:
\label{sec:kontakty-reputace}
Můžete použít dovednost \dov{Kontakty} na akci \akc{Vytvoření výhody} namísto \dov{Provokace}, kdykoliv se snažíte vyvolat v ostatních strach svojí zlověstnou reputací. S tímto trikem by měl být spojený vhodný aspekt.
\end{itemize}

\subsection{Medicína}
\label{sec:trik-medicina}
\begin{itemize}
  
\item\trk{Ambulance}:
  \label{sec:medicina-ambulance}
Získáváte bonus +2 k hodům na \dov{Medicínu}, pokud pracujete v podmínkách aspoň trochu uzpůsobeným na lékařskou práci - lampa, stůl, čisté nástroje. Je dobré poznamenat, že v běžných případech by toto k získání bonusu (resp. hodu na \akc{Vytváření výhody}) nestačilo - to by si vyžadovalo opravdovou ambulanci.

\item\trk{Výživový specialista}:
  \label{sec:medicina-vyziva}
Můžete použít \dov{Medicínu} namísto \dov{Vyšetřování} nebo \dov{Pozornosti} abyste rozpoznali, zda-li je poživatina otrávená. Když jste otráveni jedem v jídle či pití, můžete namísto \dov{Kondice} použít \dov{Medicínu}, abyste jedu odolali.

\item\trk{Bojový doktor}:
  \label{sec:medicina-boj}
  Za bod osudu se můžete pokusit hodit na \dov{Medicínu} a ošetřit postavu i během probíhajícího konfliktu. Takto lze vyléčit pouze stres a to v celkové hodnotě rovné výsledku hodu, přičemž základní opozice je pasivní a rovna +0, nicméně může být jiná (včetně aktivní) na základě okolností.
\end{itemize}

\subsection{Mobilita}
\label{sec:trik-mobilita}
\begin{itemize}
  
\item\trk{Parkour}:
\label{sec:mobilita-parkour}
Získáváte bonus +2 na akce \akc{Překonání} pomocí \dov{Mobility} pokud postava hopsá po střechách či v jiné zástavbě.

\item\trk{Sprinter}:
\label{sec:mobilita-sprinter}
Můžete se posunout až o dvě zóny zdarma (namísto jedné) v rámci fyzického konfliktu, pokud tomu nebrání žádný aspekt prostředí.


\item\trk{Hbité zmizení}:
\label{sec:mobilita-zmizeni}
Můžete použít dovednost \dov{Mobilita} namísto \dov{Skrývání}, kdykoliv vám bleskurychlé pohyby pomohou dostat se na místo, kde nejste vidět. To dobře funguje např. mezi lidmi či objekty, ale nikoli na otevřené ploše nebo při statickém kamuflování.
\end{itemize}

\subsection{Pozornost}
\label{sec:trik-pozornost}
\begin{itemize}

\item\trk{Čtenář těla}:
\label{sec:pozornost-ctenar}
Můžete použít dovednost \dov{Pozornost} namísto \dov{Empatie} či \dov{Vztahů}, abyste odhalili nějaký aspekt postavy, pakliže máte čas postavu nějakou dobu pozorovat.

\item\trk{Střelba od boku}:
\label{sec:pozornost-strelba}
Můžete použít \dov{Pozornost} namísto \dov{Boje na dálku} abyste provedli rychlý instinktivní výstřel od boku (pokus nejste např. v situaci, kdy ležíte zamířen na terč).
\end{itemize}

\subsection{Provokace}
\label{sec:trik-provokace}
\begin{itemize}
  
\item\trk{Zbroj ze strachu}:
\label{sec:provokace-zbroj}
Můžete použít \dov{Provokaci} k akci \akc{Obrana} abyste se vyhli fyzickým útokům dokud poprvé neobdržíte zranění. Protivníky dokážete tak znejistit v \akc{Útoku} na vás, že si to raději rozmyslí; jakmile vás ale poprvé zraní, stejný efekt již nenastane.

\item\trk{Cos to řekl?!}:
\label{sec:provokace-cos}
Když na protivníkovi vytvoříte výhodu pomocí \dov{Provokace}, můžete ho pomocí onoho volného vyvolání přinutit, aby svoji další akci směřoval na vás - například \akc{Útok}.

\item\trk{Šťouřání}:
\label{sec:provokace-stourani}
Můžete použít dovednost \dov{Provokace} namísto \dov{Vztahů} či \dov{Empatie}, abyste odhalili aspekt na kartě postavy tím, že ji provokujete a sledujete její reakci. Cíl se klasicky brání \dov{Vůlí}.
\end{itemize}

\subsection{Řemesla}
\label{sec:trik-remesla}
\begin{itemize}
  
\item\trk{Vždy připraven}:
\label{sec:remesla-pripraven}
V jakoukoliv situaci máte vhodné nástroje na provedení (rozumné) řemeslné práci - i kdybyste byli od svých věcí odděleni. To například znamená, že pokácet strom a nastražit past můžete i bez sekery a pily, ale těžko tak vyrobíte obléhací stroj.

\item\trk{Lepší než nový}:
\label{sec:remesla-lepsi}
Kdykoliv na akci \akc{Překonání} při opravování nějakého stroje uspějete se stylem, namísto běžného posílení můžete na stroj umístit nový aspekt reprezentující nějaký upgrade.
\end{itemize}

\subsection{Skrývání}
\label{sec:trik-skryvani}
\begin{itemize}
  
\item\trk{Jen tvář v davu}:
\label{sec:skryvani-tvar}
Získáváte bonus +2 kdykoliv se snažíte zmizet v nějakém zalidněném prostředí - náměstí, trh, místnost plná lidí. Co je a není zalidněné prostředí záleží na okolnostech.

\item\trk{Nejasný cíl}:
\label{sec:skryvani-nejasny}
Za předpokladu, že jste ve tmě, stínu či jinak těžko spatřitelní, můžete použít \dov{Skrývání} na \akc{Obranu} proti útokům pomocí \dov{Boji na dálku} z jiných zón.
\end{itemize}

\subsection{Učenost}
\label{sec:trik-ucenost}
\begin{itemize}
  
\item\trk{Specialista}:
\label{sec:ucenost-specialista}
Můžete si zvolit jeden obor poznání, ve kterém se specializujete. Na všechny hody na \dov{Učenost} z tohoto oboru pak získáváte bonus +2.

\item\trk{To nedává smysl}:
\label{sec:ucenost-smysl}
Můžete použít dovednost \dov{Učenost} na obranu před akcemi dovednosti \dov{Provokace}, pokud jste schopni situaci racionalizovat či v ní nalézt logické nekonzistence. Například u vyverního řevu vám toto nepomůže...

\item\trk{O tom jsem četl!}:
\label{sec:ucenost-cetl}
Za utracení bodu osudu můžete prohlásit, že přesně o této situaci jste již četl a použít tak \dov{Učenost} namísto libovolné jiné dovednosti pro účel hodu nebo výměny. Přitom musíte být schopen ospravedlnit, že o oné situaci se skutečně píše: to může fungovat například při boji s myšlenkovou bytostí a využití nějaké její specifické slabiny, ale těžko při odolávání efektu jedu.
\end{itemize}

\subsection{Vůle}
\label{sec:trik-vule}
\begin{itemize}
  
\item\trk{Nebojsa}
\label{sec:vule-nebojsa}:
Získáváte bonus +2 při \akc{Obraně} proti dovednosti \dov{Provokace} použité na zastrašování či jiné vyvolávání strachu.

\trk{Duch nad tělem}:
\label{sec:vule-duch}
Můžete použít dovednost \dov{Vůle} namísto \dov{Kondice} v situacích, kdy je třeba vyvinout velké fyzické úsilí.

\item\trk{Mně nelze ublížit}:
\label{sec:vule-ublizit}
Ve scéně se můžete rozhodnout ignorovat libovolný následek - ten zůstává zaškrtnutý, ale nelze jej vyvolat ani vynutit žádnou postavou. Po skončení scény se však projeví o to více - jeho závažnost o jedna vzroste, tedy z Drobného na Mírný, z Mírného na Vážný a z Vážného na Extrémní.
\end{itemize}

\subsection{Vyšetřování}
\label{sec:trik-vysetrovani}
\begin{itemize}

\item\trk{Mikrovýrazy}:
\label{sec:vysetrovani-mikro}
Můžete použít dovednost \dov{Vyšetřování} namísto \dov{Empatie} či \dov{Vůle}, abyste se bránili dovednosti \dov{Klam} pečlivým čtením řeči těla protivníka.

\item\trk{Fízlování}
\label{sec:vysetrovani-fizlovani}:
Po úspěšném \akc{Vytvoření výhody} pomocí \dov{Vyšetřování} při odposlouchávání rozhovoru můžete vytvořit o jeden aspekt navíc; nemáte na něj sice volné vyvolání, ale taky tím získáváte vypravěčskou pravomoc o detailech konverzace.
\end{itemize}


\subsection{Vztahy}
\label{sec:trik-vztahy}
\begin{itemize}

\item\trk{Miláček publika}:
  \label{sec:vztahy-demgagog}
  Získáváte bonus +2 kdykoliv používáte \dov{Vztahy} při projevu před davem.

\item\trk{Individuální péče}:
  \label{sec:vztahy-individual}
  Při používání \dov{Vztahů} na více postav nemusíte svůj hod rozdělovat mezi všechny postavy, ale na každou si můžete hodit zvlášť za předpokladu, že alespň část sdělení je relevantní pro každého z posluchačů.

\item\trk{Pití pro všechny!}:
  \label{sec:vztahy-piti}
  Dvakrát za sezení můžete povýšit posílení získané při použití \dov{Vztahů} na plnohodnotný situační aspekt.

\item\trk{Populární}:
  \label{sec:vztahy-popularni}
  Jestliže jste v oblasti, kde jste oblíbený, můžete používat \dov{Vztahy} namísto \dov{Kontaktů} (na naleznutí správného člověka apod.) K tomu je potřeba mít vhodný (situační) aspekt, jež tuto oblíbenost odráží, nebo zaplatit bod osudu.
  
\end{itemize}

\subsection{Zlodějina}
\label{sec:trik-zlodejina}

\begin{itemize}
\item\trk{Zadní vrátka}:
  \label{sec:zlodejina-vratka}
  Získáváte bonus +2 na \dov{Zlodějinu}, kdykoliv ji používáté s akcí \akc{Překonání} abyste opustili lokaci (zónu), tedy například vypáčili dveře \textbf{ven}.

\item\trk{Bezpečnostní odborník}:
  \label{sec:zlodejina-odbornik}
  Můžete poskytovat aktivní opozici kdykoliv se někdo snaží překonat vaše ``bezpečnostní zařízení'' - např. past, nástrahu, a to i když u toho přímo nejste. Bez tohoto triku by se házelo proti pasivní opozici.

\item\trk{Co se šeptá}:
  \label{sec:zlodejina-septani}
  Při jednání se zloději a podobnými můžete používat \dov{Zlodějinu} místo \dov{Kontaktů}.
\end{itemize}
  



%%% Local Variables:
%%% mode: LaTeX
%%% TeX-master: "../main"
%%% End:
