\chapter{Triky}
\label{chap:triky}

\section{Co a k čemu je trik}
\label{sec:coakcemu-trik}

\section{Efekty triků}
\label{sec:trik-efekty}

\subsection{Přidání nové akce k dovednosti}
\label{sec:trik-pridani}

\subsection{Přidání bonusu k akci}
\label{sec:trik-bonus}

\subsection{Vytvoření výjimky z pravidel}
\label{sec:trik-vyjimky}

\section{Rodiny triků}
\label{sec:trik-rodiny}

\subsection{Větvení efektů}
\label{sec:trik-vetveni}

\subsection{Skládaní efektů}
\label{sec:trik-skladani}

\section{Základní seznam triků}
\label{sec:trik-zakladni}

\subsection{Boj beze zbraně}
\label{sec:trik-bbz}

\subsection{Boj na dálku}
\label{sec:trik-bnd}

\subsection{Boj se zbraní}
\label{sec:trik-bsz}

\subsection{Empatie}
\label{sec:trik-empatie}

\subsubsection{Odhalování lží}
\label{sec:empatie-odhalovani}
Získáváte bonus +2 na všechny hody, které slouží k odhalení lží - ať už cílené na vás či na někoho jiného.

\subsubsection{Cit pro potíže}
\label{sec:empatie-cit}
Můžete použít \textit{Empatii} k určení iniciativy ve fyzických konfliktech (namísto \textit{Pozornosti}), jestliže máte během scény (či před ní) dostatek příležitostí s účastníky konfliktu alespoň chvíli mluvit.

\subsubsection{Terapeut}
\label{sec:empatie-terapuet}
Jednou za sezení můžete duševně zraněné postavě snížit závažnost jednoho neextrémního následku o jednu úroveň (odstranit Drobný, snížit Mírný na Drobný či snížit Vážný na Mírný). K tomu je potřeba s postavou odehrát rozhovor a uspět na hod na \textit{Empatii}, přičemž pasivní opozice je dle tabulky níže:

\begin{table}  
\centering
\begin{tabular}[h]{c|c}
Následek & Pasivní opozice \\ \hline
Drobný (+2) -> žádný & +2 \\
Mírný (+4) -> Drobný (+2) & +3 \\
Vážný (+6) -> Mírný (+4) & +4\\
\end{tabular}
\end{table}

\subsection{Jezdectví}
\label{sec:trik-jezd}

\subsubsection{Klíště}
\label{sec:jezdectvi-kliste}

Získáváte bonus +2 kdykoliv během honičky/závody pronásledujete jiného jezdce.

\subsubsection{Hyjé!}
\label{sec:jezdectvi-hyje}

Ze svého oře dokážete vyždímat více, než se zdá možné. Kdykoliv v při akci, která je zejména o rychlosti, remízujete, je to jako kdybyste uspěli.

\subsection{Klam}
\label{sec:trik-klam}

\subsubsection{Lež za lží}
\label{sec:klam-lez}
Získáváte bonus +2 při \texttt{Vytváření výhody} pomocí \textit{Klamu} proti někomu, kdo už během tohoto sezení nějaké z vašich lží uvěřil.

\subsubsection{Myšlenkové hry}
\label{sec:klam-mysl}
Můžete použít dovednost \textit{Klam} namísto \textit{Provokace} k provedení duševního útoku.

\subsubsection{Muž mnoha tváří}
\label{sec:klam-muz}
Kdykoliv potkáte někoho nového, můžete prohlásit, že jste jej už potkali - ale pod jinou (smyšlenou) identitou. Při interakci s touto postavou pak lze využít \textit{Klam} namísto \textit{Vztahů}; k tomu by si postava měla vytvořit situační aspekt reprezentující onu identitu.


\subsection{Kondice}
\label{sec:trik-kondice}

\subsubsection{Zápasník}
\label{sec:kondice-zapas}
Získáváte bonus +2 na akce \texttt{Vytvoření výhody}, pokud se snažíte nepřítele chytit, lapit či zaklesnout pomocí \textit{Kondice}.

\subsubsection{Drsňák}
\label{sec:kondice-drsnak}
Můžete použít dovednost \textit{Kondice} k akci \texttt{Obrana} proti \textit{Boji beze zbraně} či \textit{Boji se zbraní}, pokud je zbraň tupá. V případě remízy však dostáváte zranění o velikosti jednoho posunu.

\subsubsection{Nezdolný}
\label{sec:kondice-nezdolny}
Jednou za sezení může postava přeměnit Mírný (+4) následek na Drobný (+2) (pokud jej nemá zaškrtnutý) či se úplně Drobného následku zbavit.


\subsection{Kontakty}
\label{sec:trik-kontakty}

\subsubsection{Drbna}
\label{sec:kontakty-drbna}
Získáváte bonus +2, na akce \texttt{Překonání} či \texttt{Vytvoření výhody} pomocí rozšiřování pomluv dovedností \textit{Kontakty}.

\subsubsection{Ucho v pozoru}
\label{sec:kontakty-ucho}
Můžete použít dovednost \textit{Kontakty} k určení iniciativy ve fyzickém konfliktu (namísto \textit{Pozornosti}), pokud se konflikt odehrává v prostředí, ve kterém máte vybudouvanou síť.

\subsubsection{Síla reputace}
\label{sec:kontakty-reputace}
Můžete použít dovednost \textit{Kontakty} na akci \texttt{Vytvoření výhody} namísto \textit{Provokace}, kdykoliv se snažíte vyvolat v ostatních strach svojí zlověstnou reputací. S tímto trikem by měl být spojený vhodný aspekt.

\subsection{Medicína}
\label{sec:trik-medicina}

\subsection{Mobilita}
\label{sec:trik-mobilita}

\subsubsection{Parkour}
\label{sec:mobilita-parkour}
Získáváte bonus +2 na akce \texttt{Překonání} pomocí \textit{Mobility} pokud postava hopsá po střechách či v jiné zástavbě.

\subsubsection{Sprinter}
\label{sec:mobilita-sprinter}
Můžete se posunout až o dvě zóny zdarma (namísto jedné) v rámci fyzického konfliktu, pokud tomu nebrání žádný aspekt prostředí.

\subsection{Pozornost}
\label{sec:trik-pozornost}

\subsubsection{Čtenář těla}
\label{sec:pozornost-ctenar}

Můžete použít dovednost \textit{Pozornost} namísto \textit{Empatie} či \textit{Vztahů}, abyste odhalili nějaký aspekt postavy, pakliže máte čas postavu nějakou dobu pozorovat.

\subsubsection{Střelba od boku}
\label{sec:pozornost-strelba}
Můžete použít \textit{Pozornost} namísto \textit{Boje na dálku} abyste provedli rychlý instinktivní výstřel od boku (pokus nejste např. v situaci, kdy ležíte zamířen na terč).

\subsection{Provokace}
\label{sec:trik-provokace}

\subsubsection{Zbroj ze strachu}
\label{sec:provokace-zbroj}

Můžete použít \textit{Provokaci} k akci \texttt{Obrana} abyste se vyhli fyzickým útokům dokud poprvé neobdržíte zranění. Protivníky dokážete tak znejistit v \texttt{Útoku} na vás, že si to raději rozmyslí; jakmile vás ale poprvé zraní, stejný efekt již nenastane.

\subsubsection{Cos to řekl?!}
\label{sec:provokace-cos}
Když na protivníkovi vytvoříte výhodu pomocí \textit{Provokace}, můžete ho pomocí onoho volného vyvolání přinutit, aby svoji další akci směřoval na vás - například \texttt{Útok}.

\subsubsection{Šťouřání}
\label{sec:provokace-stourani}

Můžete použít dovednost \textit{Provokace} namísto \textit{Vztahů} či \textit{Empatie}, abyste odhalili aspekt na kartě postavy tím, že ji provokujete a sledujete její reakci. Cíl se klasicky brání \textit{Vůlí}.

\subsection{Řemesla}
\label{sec:trik-remesla}

\subsubsection{Vždy připraven}
\label{sec:remesla-pripraven}

V jakoukoliv situaci máte vhodné nástroje na provedení (rozumné) řemeslné práci - i kdybyste byli od svých věcí odděleni. To například znamená, že pokácet strom a nastražit past můžete i bez sekery a pily, ale těžko tak vyrobíte obléhací stroj.

\subsubsection{Lepší než nový}
\label{sec:remesla-lepsi}

Kdykoliv na akci \texttt{Překonání} při opravování nějakého stroje uspějete se stylem, namísto běžného posílení můžete na stroj umístit nový aspekt reprezentující nějaký upgrade.

\subsection{Skrývání}
\label{sec:trik-skryvani}

\subsubsection{Jen tvář v davu}
\label{sec:skryvani-tvar}

Získáváte bonus +2 kdykoliv se snažíte zmizet v nějakém zalidněném prostředí - náměstí, trh, místnost plná lidí. Co je a není zalidněné prostředí záleží na okolnostech.

\subsubsection{Nejasný cíl}
\label{sec:skryvani-nejasny}

Za předpokladu, že jste ve tmě, stínu či jinak těžko spatřitelní, můžete použít \textit{Skrývání} na \texttt{Obranu} proti útokům pomocí \textit{Boji na dálku} z jiných zón.

\subsection{Učenost}
\label{sec:trik-ucenost}

\subsubsection{Specialista}
\label{sec:ucenost-specialista}

Můžete si zvolit jeden obor poznání, ve kterém se specializujete. Na všechny hody na \textit{Učenost} z tohoto oboru pak získáváte bonus +2.

\subsubsection{To nedává smysl}
\label{sec:ucenost-smysl}

Můžete použít dovednost \textit{Učenost} na obranu před akcemi dovednosti \textit{Provokace}, pokud jste schopni situaci racionalizovat či v ní nalézt logické nekonzistence. Například u vyverního řevu vám toto nepomůže...

\subsubsection{O tom jsem četl!}
\label{sec:ucenost-cetl}

Za utracení bodu osudu můžete prohlásit, že přesně o této situaci jste již četl a použít tak \textit{Učenost} namísto libovolné jiné dovednosti pro účel hodu nebo výměny. Přitom musíte být schopen ospravedlnit, že o oné situaci se skutečně píše: to může fungovat například při boji s myšlenkovou bytostí a využití nějaké její specifické slabiny, ale těžko při odolávání efektu jedu.

\subsection{Vůle}
\label{sec:trik-vule}

\subsubsection{Nebojsa}
\label{sec:vule-nebojsa}

Získáváte bonus +2 při \texttt{Obraně} proti dovednosti \textit{Provokace} použité na zastrašování či jiné vyvolávání strachu.

\subsubsection{Duch nad tělem}
\label{sec:vule-duch}

Můžete použít dovednost \textit{Vůle} namísto \textit{Kondice} v situacích, kdy je třeba vyvinout velké fyzické úsilí.

\subsubsection{Mně nelze ublížit}
\label{sec:vule-ublizit}

Ve scéně se můžete rozhodnout ignorovat libovolný následek - ten zůstává zaškrtnutý, ale nelze jej vyvolat ani vynutit žádnou postavou. Po skončení scény se však projeví o to více - jeho závažnost o jedna vzroste, tedy z Drobného na Mírný, z Mírného na Vážný a z Vážného na Extrémní.


\subsection{Vyšetřování}
\label{sec:trik-vysetrovani}

\subsubsection{Mikrovýrazy}
\label{sec:vysetrovani-mikro}

Můžete použít dovednost \textit{Vyšetřování} namísto \textit{Empatie} či \textit{Vůle}, abyste se bránili dovednosti \textit{Klam} pečlivým čtením řeči těla protivníka.

\subsubsection{Fízlování}
\label{sec:vysetrovani-fizlovani}

Po úspěšném \texttt{Vytvoření výhody} pomocí \textit{Vyšetřování} při odposlouchávání rozhovoru můžete vytvořit o jeden aspekt navíc; nemáte na něj sice volné vyvolání, ale taky tím získáváte vypravěčskou pravomoc o detailech konverzace.



\subsection{Vztahy}
\label{sec:trik-vztahy}

\subsection{Zlodějina}
\label{sec:trik-zlodejina}




%%% Local Variables:
%%% mode: LaTeX
%%% TeX-master: "../main"
%%% End:
