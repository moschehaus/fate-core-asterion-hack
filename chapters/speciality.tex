% !TEX root = main.tex

\documentclass[../main.tex]{subfiles}
\begin{document}

\chapter{Speciality}

Obecný výklad specialit bude hodně ...obecný. To je způsobeno tím, že pod speciality se schová ledasco - vlastně je to všechno, čemu umožňujeme nějaké speciální zacházení s pravidly:

\begin{itemize}
\item magické a nadpřirozené schopnosti: magie, alchymie, theurgie, kněžství
\item speciální předměty: zbraně, zbroje, očarované vybavení
\item organizace, místa, lokace
\end{itemize}

Nejčastěji jsou speciality spojeny s nějakou postavou (např. postava oplývá nějakými schopnostmi, vlastní nějaký předmět); striktně vzato tomu ale tak být nemusí a jako specialitu ošetřovat i například celou lokaci (systém pastí v starověké pyramidě) či postavu jako celek (jezdecké zvíře). V dalších kapitolách rozvedeme tyto speciality podrobněji.

\section{Vytváření specialit}
\label{sec:spec-vytvareni}
Každému se v hlavě honí fantaskní představy o epičnosti všeho herního. To je pro tvorbu specialit dobrý základ, tyto představy je ale třeba nejprve prohnat různými filtry.\\
Tak zejména, vytváření specialit by mělo být součástí vytváření postav, respektive herního prostředí. Z toho plyne, že existence, důležitost a rozsah specialit by měl být dán domluvou všech hráčů (nejen Vypravěče a jednoho hráče). Není zábavné, když má jedna postava víc specialit než běžných dovedností, kdežto ostatní nemají žádné, nebo když je postava díky specialitám mnohonásobně zdatnější. \footnote{Pokud jsou s tím ovšem hráči v pořádku, není v tomto žádný problém.}\\
Aby byly speciality vyvážené a zajímavé, je důležité myslet zejména na

\begin{itemize}
\item Zapadají speciality vhodným způsobem do herního světa? (Nechci na Asterionu vlastnit vrhač částic?)
\item Co \textit{přesně} speciality umožňují? Toto by mělo být popsáno dostatečně přesnými pravidly. \footnote{``Jednou za sezení můžu toto a toto / když já dělám toto a někdo něco jiného, pak můžu zadarmo udělat toto''.}
\item Jakým způsobem souvisí speciality s postavami? To jest, upravuje speciality dovednosti, triky, obnovu, zranění?
\item Jaké je svolení (jak je možné, že specialitu vlastním?) a jaká je cena speciality (čím jsem zaplatil, že ji vlastním)?
\end{itemize}


\subsection{Jak speciality souvisí s herním prostředím}
\label{sec:spec-prvky}

Snad každý fantasy setting obsahuje nějaký prvek magie a Asterion rozhodně není výjimkou. Pro něj je navíc typické velké množství bohů a to jak aktivně se do dění ve světě zapojují (i když o tom hráči ani netuší...). Nejpřirozeněji lze toto nadpřirozeno popsat specialiatmi, ale ty musí být s ``pravidly'' magie na Asterionu slučitelné: pokud například na Asterionu neexistuje nekromancie \footnote{Nekromancie na Asterionu existuje, takže tento příklad nedává smysl.}, těžko pak hráč může chtít pomocí speciality umět oživovat mrtvé.\\
To se samozřejmě netýká jen magie a bohů. Jinou zajímavou specialitou mohou být bojová umění. Ty na Asterionu mají bohatou tradici u skřítků, kteří je ale kvůli špatným zkušenostem neučí jiné národy - proto by postava s bojovým uměním buď měla být skřítek, nebo umět dobře vysvětlit, jak se tyto schopnosti naučil.\\
Nejen, že by měly speciality respektovat herní prostředí, mohou z něj dokonce vznikat. Tak lze elegantně popsat různé instituce (Almendorská tajná služba, Noční let,...), či místa (Albirejské stoky, Dračí údolí, Stínový svět, Poušť Mar' Nub...), které nestačí vystihnout jednoduchým aspektem. V sekci~\ref{sec:spec-fraktal} o Fateovém fraktálu ukážeme, jakým báječným způsobem lze celou instituci či místo namapovat na jedinou \textit{postavu}.

\subsection{Co speciality dělají}
\label{sec:spec-delaji}
Zde toho nenapíšeme moc - snad jen omýlanou frázi, že speciality nějak upravují či rozvíjejí pravidla. To lze samozřejmě činit různými způsoby a vyjádřit je všechny je nemožné. \\
Je ovšem podstatné zdůraznit, že toto upravování pravidel by mělo být ov jasných předem stanovených mezích. Umožňuje vám specialita seslat ochranný štít a použít jej na \akc{Obranu}? Je tomu ale potřeba věnovat celou svoji akci, nebo to zvládnete v reakci na něčí jiný \akc{Útok}? Musíte předtím tušit, že na vás někdo bude útočit, nebo to může být i z překvapení? A tak podobně.

\subsection{Jak speciality souvisí s postavami}
\label{sec:spec-postavy}
Abychom mohli speciality nějak univerzálně uchopit, je dobré jejich mechaniku propojit s jinými prvky postav: dovednostmi, triky, obnovou, předměty, aspekty.



\subsection{Svolení a cena}
\label{sec:spec-svolenicena}


\subsection{Fateový fraktál}
\label{sec:spec-fraktal}

\section{Magie}
\label{sec:magie}





\end{document}

%%% Local Variables:
%%% mode: LaTeX
%%% TeX-master: "../main"
%%% End:
