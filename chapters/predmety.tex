
\documentclass[../main.tex]{subfiles}
\begin{document}


\chapter{Předměty jako speciality}
\label{chap:magicke-spec-predmety}

Mechanika Fateového fraktálu \Cref{sec:spec-fraktal} umožňuje elegantním a jednotným způsobem nakládat s (příběhově významnými) předměty jako s specialitami, resp. jako s postavami, a přiřadit jim aspekty, stresy/následky, dovednosti či triky. Sic elegantní, je takovéto nakládání s předměty poněkud komplikované, a proto jej v běžném hraní aplikujeme pouze na předměty významnější - typicky \emph{magické} \Cref{sec:magicke-vybaveni}. V dalším rozebereme mechaniku týkající se několika běžných skupin předmětů: zbraně a zbroje, alchymistické výrobky (lektvary, amulety, pasti a jedy) a magické předměty (dočasné předměty, artefakty). Nebude překvapením, že v \emph{Realistickém hraní} jsou předměty nezanedbatelně podstatnější, než v hraním příběhovém; v rámci tohoto přístupu se proto dočkáme rozsáhlejšího popisu.

\section{Zbraně a zbroje}
\label{sec:zbrane-zbroje}

Při fyzickém konfliktu \Cref{sec:obecna-pravidla} velmi často hraje důležitou roli vybavení. Šermíř s \asp{Jiskřící čepelí} \sidenote{Dlouhý jednoruční lintirový meč, jež Rangwathovi umožňuje provést \emph{3 útoky během jediné výměny}.} \emph{Rangwatha, Černého hraničáře} bude mít proti sokovi s klackem v ruce nespornou výhodu. Jak postihnout kvalitu a vlastnosti zbraní a zbrojí rozebíráme v další kapitole. Poznamenejme s předstihem, že v případě \emph{Příběhového hraní} budou zbraně a zbroje představovat jen prostou opravu k celkovému výsledku hodu v rámci \per{Konfliktu}. Při  \emph{Realistickém hraní} bude situace mnohem komplikovanější.

\begin{Real}[Zbraně a zbroje]
	V případě \emph{Realistického hraní} bereme v potaz i následující faktory:
	\begin{itemize}
		\item výdrž vybavení,
		\item náročnost opravy vybavení,
		\item peníze nutné k nákupu/opravě,
		\item další vlastnosti (typicky výhody) zbraní.
	\end{itemize}
	Poškození zbraní/zbrojí je popsáno v sekcích \Cref{sec:zbrane}, \Cref{sec:zbroje}.
\end{Real}


\subsection{Zbraně}
\label{sec:zbraně}

V případě \emph{Příběhového hraní} se postupuje dle standardních pravidel \per{Konfliktu}, jak popsáno v \Cref{sec:obecna-pravidla}, přičemž hodnocení zbraní je dáno následovně.


\begin{table}[h]  
\centering
\begin{tabular}[h]{c|c}
Hodnocení zbraně & Zbraň \\ \hline
+1 & dýka, nůž, obušek \\
+2 & tesák, meč, sekera, palcát, řemdih \\
+3 & obouruční meč/sekera/palcát/kladivo, kopí, biják\\
\end{tabular}
\end{table}

\begin{Real}[Zbraně]
	Každá zbraň má svojí výdrž reprezentovanou \emph{následky}; jestliže to ovšem není příběhově podstatné, následky explicitně nepopisujeme a též na ně útočník nezískává volná vyvolání. Počet dostupných následků a stejně tak jejich velikost je různá pro různé zbraně. Jakmile výdrž zbraně klesne pod jednu polovinu svého maxima, snižuje se hodnocení zbraně o jedna; zbraně s hodnocením +1 jsou výjimkou, jejich hodnocení se nikdy nesnižuje. Zbraň se poškozuje při výsledcích akcí následovně:
	\begin{itemize}
		\item při neúspěšném \akc{Útoku} se útočníkova zbraň obecně nepoškozuje. Jestliže ovšem obránce uspěje se stylem na \akc{Obranu}, může se rozhodnout namísto získání posílení poškodit útočníkovu zbraň za tolik posunů, kolik činila jeho míra úspěchu. \sidenotemark.
		\item při remíze jsou zbraně obou útočníku poškozeny za výsledek jejich hodu,
		\item při úspěchu je zbraň útočníka poškozena za jeho míru úspěchu (\emph{bez} započítání zbraní a zbrojí!),
		\item při úspěchu se stylem může útočník snížit svoji míru úspěchu o jedna a rozhodnout se svoji zbraň vůbec nepoškodit. \sidenotemark 
	\end{itemize}
\end{Real}
\sidenotetext{Při blokování úderů útočníka se útočníkova zbraň jistě poškozuje; jestliže se obránce ubránil s grácií, dokáže se kromě vyhnutí se zranění vyhnout i poškození svoji zbraně (či zbroje).}
\sidenotetext{To lze kombinovat s klasickým snížení o jedna výměnou za \asp{Posílení.}}
\todo{
\begin{itemize}
	\item obusek,
	\item skrtici drat,
	\item sart,
	\item Mahensky bic,
	\item Perilonske ostri,
	\item flusacka,
\end{itemize}
}

\subsubsection{Jednoruční krátké zbraně}
\label{sec:zbrane_jedna}

\begin{predmet}[Dýka]
\begin{itemize}
	\item \textbf{Popis:} Nejběžnější dobrodruhova zbraň.
	\item \textbf{Výdrž:} (\emph{následky}): \policko{+1}{12}
	\item \textbf{Vlastnosti:} Hodnocení zbraně +1. 
\end{itemize}
\end{predmet}

\begin{predmet}[Obušek]
\begin{itemize}
	\item \textbf{Popis:} Kus tvrdého dřeva, hodí se k omračování.
	\item \textbf{Výdrž:} (\emph{následky}): \policko{+1}{12} 
	\item \textbf{Vlastnosti:} Hodnocení zbraně +1. \trk{Omráčení}: V případě úspěchu se stylem na akci \akc{Útok} lze případné \asp{Posílení} využít na omráčení nepřítele. Ten musí přehodit svojí \dov{Kondicí} míru úspěchu útočníka; pokud selže, je omráčen na 3 kola.
\end{itemize}
\end{predmet}

\subsubsection{Jednoruční dlouhé zbraně}
\label{sec:zbrane_dva}

\begin{predmet}[Sekera]
\begin{itemize}
	\item \textbf{Popis:} Dřevěné toporo s ocelovou hlavicí.
	\item \textbf{Výdrž:} (\emph{následky}): \policko{+2}{6}
	\item \textbf{Vlastnosti:} Hodnocení zbraně: +2. Snadno se opravuje, je levná.
\end{itemize}
\end{predmet}

\begin{predmet}[Palcát]
\begin{itemize}
	\item \textbf{Popis:} Dřevěné toporo s ocelovou tupou hlavicí.
	\item \textbf{Výdrž:} (\emph{následky}): \policko{+2}{12}
	\item \textbf{Vlastnosti:} Hodnocení zbraně: +2. Hodí se pro boj proti protivníkům s těžšími zbrojemi: palcát má hodnocení zbraně +3, jestliže má protivník zbroj s hodnocením alespoň +2.
\end{itemize}
\end{predmet}

\begin{predmet}[Meč]
\begin{itemize}
		\item \textbf{Popis:} Klasický ocelový meč. \sidenotemark
		\item \textbf{Výdrž:} (\emph{následky}): \policko{+2}{9}
		\item \textbf{Vlastnosti:} Hodnocení zbraně: +2. Hodí se pro boj proti protivníkům s lehkou či žádnou zbrojí: meč má hodnocení zbraně +3, jestliže má více než 3/4 své výdrže a protivník má zbroj s hodnocením nanejvýš +1. \sidenotemark 
	\end{itemize}
	\end{predmet}
\sidenotetext{Je-li třeba, lze rozlišovat kordy, rapíry, flerety, tesáky, dlouhé a krátké meče, šavle, široké meče apod.}
\sidenotetext{Situace beze zbroje odpovídá hodnocení zbroje +0}
	\subsubsection{Obouruční zbraně}
	\label{sec:zbrane_treti}

	\begin{predmet}[Obouruční sekera]
	\begin{itemize}
		\item \textbf{Popis:} Dřevěné toporo s ocelovou hlavicí.
		\item \textbf{Výdrž:} (\emph{následky}): \policko{+3}{6}
		\item \textbf{Vlastnosti:} Hodnocení zbraně: +3. Snadno se opravuje, je levná.
	\end{itemize}
	\end{predmet}

	\begin{predmet}[Obouruční palcát (kladivo, kyj)]
	\begin{itemize}
		\item \textbf{Popis:} Dřevěné toporo s ocelovou tupou hlavicí.
		\item \textbf{Výdrž:} (\emph{následky}): \policko{+3}{12}
		\item \textbf{Vlastnosti:} Hodnocení zbraně: +3. Hodí se pro boj proti protivníkům s těžšími zbrojemi: palcát má hodnocení zbraně +4, jestliže má protivník zbroj s hodnocením alespoň +2.
	\end{itemize}
	\end{predmet}

	\begin{predmet}[Obouruční meč]
	\begin{itemize}
		\item \textbf{Popis:} Klasický ocelový meč. \sidenotemark 
		\item \textbf{Výdrž:} (\emph{následky}): \policko{+3}{9}
		\item \textbf{Vlastnosti:} Hodnocení zbraně: +3. Hodí se pro boj proti protivníkům s lehkou či žádnou zbrojí: meč má hodnocení zbraně +4, jestliže má více než 3/4 své výdrže a protivník má zbroj s hodnocením nanejvýš +1. \sidenotemark
	\end{itemize}
	\end{predmet}
\sidenotetext{Je-li třeba, lze rozlišovat kordy, rapíry, flerety, tesáky, dlouhé a krátké meče, šavle, široké meče apod.}
\sidenotetext{Situace beze zbroje odpovídá hodnocení zbroje +0.}
\begin{predmet}[Kopí]
\begin{itemize}
	\item \textbf{Popis:} Dřevěné ratiště s ocelovou hlavicí.
	\item \textbf{Výdrž:} (\emph{následky}): \policko{+3}{6}
	\item \textbf{Vlastnosti:} Hodnocení zbraně: +3. \trk{Dlouhá špičatá věc:} Při \akc{Vytváření výhody} s kopím v úzkých prostorech představující \asp{Držím si ho od těla} a podobné získává útočník bonus +2.
\end{itemize}
\end{predmet}

\subsection{Střelné zbraně}
\label{sec:strelne}

U zbraní na blízko je možné, že dojde k významnému poškození zbraně i při \emph{úspěšném útoku} - to je snaha vystihnout situaci, kdy k uštědření velikého zranění je potřeba vynaložit velikou sílu, tedy i velmi namáhat zbran. U střelných zbraní toto nedává dobrý smysl: \textcolor{gray}{kupříkladu} luk je napnut při útoku pokaždé stejně a způsobené zranění závisí zejména na místě, kam se podařilo protivníka trefit - tedy odvíjet poškození luku podle uděleného zranění není realistické. Navíc jsou luky a kuše střelbou opotřebovávány jen minimálně\sidenote{Při standardním používání vydrží ramena luku či lučiště kuše desítky let, čas od času je pouze třeba vyměnit tětivu.}  - spotřebovává se hlavně střelivo.

Výdrž střelných zbraní proto nepočítáme klasicky, ale pouze skrze dostupné střelivo. Tedy, při výstřelu neuvažujeme, že by luk jakkoliv spotřeboval, pouze střelec přijde o šipku. Bude-li chtít, může si tak namísto čtverečků představující výdrž zbraně nakreslit čtverečky představující dostupné střelivo. To je poměrně dobře v souladu s předchozím: luk s vyškrtanou tabulkou, tedy bez střeliva, je stejně dobře nepoužitelný jako meč s vyškrtanou tabulkou, tedy zničený.

\emph{"Opravování střelných zbraní"} pak též nelze brát zcela doslovně: jednalo by se případně o pokus o nalezení vystřelených projektilů.


\begin{remark}
	To, že se střelné zbraně neopotřebovávají při použití samozřejmě \emph{neznamená, že jsou nerozbitné}. Stále může dojít k poškození zbraně, zejména v fyzických akcích obsahující pády, nárazy či údery.
\end{remark}

\subsection{Speciální zbraně}
\label{sec:specialni_zbrane}

\subsubsection{Zbraně ze zvláštních materiálů}
\label{sec:materialy}

Ve světě Asterionu existuje řada jedinečných a neobyčejných materiálů. Zbraně a zbroje z nich pak zpravidla propůjčují předmětům další nezanedbatelné výhody. Nejznámnějšími materiály jsou

\begin{itemize}
	\item atorské stříbro,
	\item lintir,
	\item garen,
	\item rajmin,
	\item mistál,
	\item helurim.
\end{itemize}

\textbf{Atorské stříbro}

Pevnostně odpovídá atorské stříbro běžné oceli.
\textbf{Lintir}

\textbf{Garen}

\textbf{Rajmin}
Viz pravidla M16, str. 267 \autocite[267]{stasiakDraciDoupeFate}

\textbf{Mistál}
Viz pravidla M16, str. 266 \autocite[266]{stasiakDraciDoupeFate}

\textbf{Helurim}
Viz pravidla M16, str. 267 \autocite[267]{stasiakDraciDoupeFate}

\subsubsection{Zbraně hrdlořezů}
\label{sec:hrdlorez}

\subsection{Zbroje}
\label{sec:zbroje}

V případě \emph{Příběhového hraní} pouze zbroj obránci přidává bonus pro jeho hody na \akc{Obranu} dle hodnocení zbrojí uvedného v tabulce níže.


\begin{table}[h]  
\centering
\begin{tabular}[h]{c|c}
Hodnocení zbroje & Zbraň \\ \hline
+1 & vycpávaná, prošívaná, kožená \\
+2 & kroužková, destičková, šupinová + helma \\
+3 & kroužková, destičková, šupinová + helma, nátepníky, ramena, kolena, stehna\\
+4 & celoplátová\\
\end{tabular}
\end{table}

\begin{Real}[Zbroje]
	Každá zbroj má svojí výdrž reprezentovanou \emph{stresy}; lze tedy zaškrtnout vždy nanejvýš jeden při každé výměně. Počet dostupných stresů a stejně tak jejich velikost je různá pro různé zbroje. Jakmile výdrž zbroje klesne pod jednu polovinu svého maxima, snižuje se hodnocení zbroje o jedna; zbroje s hodnocením +1 jsou výjimkou, jejich hodnocení se nikdy nesnižuje. 
	Zbroj funguje de facto jako zásobník bonusových políček stresů. Uspěje-li tedy útočník na \akc{Útok}, může obránce použít políčka stresů zbroje, aby se pokryl utržené posuny, čímž uchrání zranění svého těla výměnou za poškození zbroje. \sidenotemark Připomeňme, že z povahy stresů lze vždy použít nanejvýš jedno políčko stresu zbroje, nicméně to lze kombinovat s fyzickými stresy postavy.
\end{Real}
\sidenotetext{Obránce se též může rozhodnout nechat se zranit ihned do těla.}
\begin{predmet}[Vycpávaná zbroj]
\begin{itemize}
	\item \textbf{Popis:} 
	\item \textbf{Výdrž:} (\emph{stresy}:) \policko{+1}{2}
	\item \textbf{Vlastnosti:} 
\end{itemize}
\end{predmet}


\begin{predmet}[Prošívaná zbroj]
\begin{itemize}
	\item \textbf{Popis:} 
	\item \textbf{Výdrž:} (\emph{stresy}:) \policko{+1}{3}
	\item \textbf{Vlastnosti:} 
\end{itemize}
\end{predmet}

\begin{predmet}[Kožená zbroj]
\begin{itemize}
	\item \textbf{Popis:} 
	\item \textbf{Výdrž:} (\emph{stresy}:) \policko{+1}{4}
	\item \textbf{Vlastnosti:} 
\end{itemize}
\end{predmet}


\begin{predmet}[Kroužková zbroj s helmou]
\begin{itemize}
	\item \textbf{Popis:} 
	\item \textbf{Výdrž:} (\emph{stresy}:) \policko{+2}{6}
	\item \textbf{Vlastnosti:} 
\end{itemize}
\end{predmet}

\begin{predmet}[Šupinová zbroj s helmou]
\begin{itemize}
	\item \textbf{Popis:} 
	\item \textbf{Výdrž:} (\emph{stresy}:) \policko{+2}{7}
	\item \textbf{Vlastnosti:} 
\end{itemize}
\end{predmet}

\begin{predmet}[Destičková zbroj s helmou]
\begin{itemize}
	\item \textbf{Popis:} 
	\item \textbf{Výdrž:} (\emph{stresy}:) \policko{+2}{8}
	\item \textbf{Vlastnosti:} 
\end{itemize}
\end{predmet}

\begin{predmet}[Kroužková zbroj s helmou a doplňky]
\begin{itemize}
	\item \textbf{Popis:} 
	\item \textbf{Výdrž:} (\emph{stresy}:) \policko{+3}{9}
	\item \textbf{Vlastnosti:} 
\end{itemize}
\end{predmet}

\begin{predmet}[Šupinová zbroj s helmou a doplňky]
\begin{itemize}
	\item \textbf{Popis:} 
	\item \textbf{Výdrž:} (\emph{stresy}:) \policko{+3}{10}
	\item \textbf{Vlastnosti:} 
\end{itemize}
\end{predmet}

\begin{predmet}[Destičková zbroj s helmou a doplňky]
\begin{itemize}
	\item \textbf{Popis:} 
	\item \textbf{Výdrž:} (\emph{stresy}:) \policko{+3}{11}
	\item \textbf{Vlastnosti:} 
\end{itemize}
\end{predmet}


\begin{predmet}[Plátová zbroj]
\begin{itemize}
	\item \textbf{Popis:} Plechovka
	\item \textbf{Výdrž:} (\emph{stresy}:) \policko{+4}{12}
	\item \textbf{Vlastnosti:} 
\end{itemize}
\end{predmet}

\section{Alchymistické předměty}
\label{sec:alchymisticke-predmety}

\section{Magické vybavení}
\label{sec:magicke-vybaveni}


\end{document}



%%% Local Variables:
%%% mode: LaTeX
%%% TeX-master: "../main"
%%% End:
