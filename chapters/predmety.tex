\chapter{Předměty jako speciality}
\label{chap:magicke-spec-predmety}

\section{Zbraně a zbroje}
\label{sec:zbrane-zbroje}

\subsection{Zbraně}
\label{sec:zbraně}

\todo{
\begin{itemize}
	\item obusek,
	\item skrtici drat,
	\item sart,
	\item Mahensky bic,
	\item Perilonske ostri,
	\item flusacka,
	\item arbalet,
	\item kuse,
	\item luky,
	\item vrhaci predmety
\end{itemize}
}

Vsechny zbrane maji durabilitu danou stresovymi policky, pricemz hodnota stresu je dana hodnocenim zbrane. Nektere zbrane maji dalsi vyhody.

Zraneni do zbrane je dano jako

\begin{itemize}
	\item v pripade neuspechu ve kterem se navic obrance ubrani se stylem, muze se obrance rozhodnout namisto Posileni poskodit utocnikovu zbran v hodnote dane uspechu, se kterym se ubranil
	\item hod na \akc{Utok} v pripade remizy
	\item MU v pripade vyhry na \akc{Utok}
	\item v pripade Uspechu se stylem lze snizit uspech o jedna a ponechat si zbran neposkozenou \marginnote{Toto lze kombinovat s klasickym vzitim posileni.}
\end{itemize}

\subsubsection{Hodnoceni zbrane 1}
\label{sec:zbrane_jedna}

\begin{Predmet}[Dyka]
\begin{itemize}
	\item vydrzi 12x1
\end{itemize}
\end{Predmet}

\begin{Predmet}[Obusek]
\begin{itemize}
	\item vydrzi 12x1
	\item \trk{Omraceni}: V pripade uspechu se stylem na \akc{Utok} lze pripadne Posileni vyuzit na omraceni nepritele. Ten musi prehodit svoji \dov{Kondici} MU utocnika; pokud selze, je omracen na 3 kola.
\end{itemize}
\end{Predmet}

\subsubsection{Hodnoceni zbrane 2}
\label{sec:zbrane_dva}

\begin{Predmet}[Sekera]
\begin{itemize}
  \item
  \item vydrzi 6x2
\end{itemize}
\end{Predmet}

\begin{Predmet}[Palcát]
\begin{itemize}
	\item vydrzi 12x2
	\item proti zbrojim krome s hodnocenim +1 ma hodnoceni zbrane 3
\end{itemize}
\end{Predmet}

\begin{Predmet}[Meč]
\begin{itemize}
	\item vydrzi 9x2
	\item proti zbrojim s hodnocenim +1 ci beze zbroje ma hondnoceni zbrane 3, \emph{jestlize} ma vice nez 3/4 sve vydrze
\end{itemize}
\end{Predmet}

\subsubsection{Hodnoceni zbrane 3}
\label{sec:zbrane_treti}

\begin{Predmet}[Sekera]
\begin{itemize}
  \item vydrzi 6x3
\end{itemize}
\end{Predmet}

\begin{Predmet}[Palcát]
\begin{itemize}
	\item vydrzi 12x3
	\item proti zbrojim krome s hodnocenim +1 ma hodnoceni zbrane 3
\end{itemize}
\end{Predmet}

\begin{Predmet}[Meč]
\begin{itemize}
	\item vydrzi 9x3
	\item proti zbrojim s hodnocenim +1 ci beze zbroje ma hondnoceni zbrane 3, \emph{jestlize} ma vice nez 3/4 sve vydrze
\end{itemize}
\end{Predmet}


\begin{Predmet}[Kopí]
\begin{itemize}
	\item vydrzi 6x2
	\item \trk{Ostrej klacek}: Pri \akc{Vytvareni vyhody} s kopim v uzkych prostorech predstavujici \asp{Drzim si ho od tela} ziskava utocnik bonus +2.
\end{itemize}
\end{Predmet}
\subsection{Zbroje}
\label{sec:zbroje}

Vsechny zbroje krome +1 davaji o jedna mensi kryti pokud jejich durabilita klesne pod pulku.

\begin{Predmet}[Hodnoceni zbroje +1]
\begin{itemize}
	\item prosivanice - 2/3 hity,
	\item vycpavana - 2/3 hity,
	\item kozena - 4/5 hity
\end{itemize}
\end{Predmet}

\begin{Predmet}[Hodnoceni zbroje +2]
\begin{itemize}
	\item krouzkova - 6 hitu ,
	\item supinova - 7 hitu ,
	\item destickova - 8 hitu ,
\end{itemize}
\end{Predmet}

\begin{Predmet}[Hodnoceni zbroje +3]
\begin{itemize}
	\item krouzkova + doplnky - 9 hitu,
	\item supinova + doplnky 10 hitu, 
	\item destickova + doplnky 11 hitu, 
\end{itemize}
\end{Predmet}

\begin{Predmet}[Hodnoceni zbroje +4]
\begin{itemize}
	\item celoplatova zbroj - 12 hitu
\end{itemize}
\end{Predmet}

\section{Alchymistické předměty}
\label{sec:alchymisticke-predmety}

\section{Magické vybavení}
\label{sec:magicke-vybaveni}





%%% Local Variables:
%%% mode: LaTeX
%%% TeX-master: "../main"
%%% End:
