\documentclass[../main.tex]{subfiles}
\graphicspath{{\subfix{../graphics/}}}
\begin{document}

\section{Aspekty, narativ a příběhová pravomoc}
\label{sec:aspekty-narativ-pribeh}

``Pod vším ve Fate hledej aspekty''. Skutečně, aspekty hrají ve Fate Core zcela zásadní roli. Jsou tím, co je na těchto pravidlech typické a bez nich by Fate Core nebyl Fate Core.

\begin{itemize}
\item postavy, nejdůležitější součást hry, jsou definovány a vymezeny pomocí aspektů (\asp{Špinavý žoldák z Hlubiny, Hbitý skřítek z Imchejle, Čestný rytíř z řádu Devíti hvězd, Temperamemtní učednice Alwarina Bílého})
\item herní prostředí, tj. parkety, na kterých se odehrává drama postav, je nastaveno pomocí aspektů (\asp{Největší město na Taře, Prokleté údolí, Hora, kde sídlí bohové, Jeskyně plná drahokamů})
\item momentální stav prostředí, nálada scény, důležitý okamžitý detail jsou všechno věci vyjádřené pomocí aspektů (\asp{Podlaha v plamenech, Překážka v cestě, Tma jako v pytli, Něco tu sakra nehraje...})
\item významné předměty a vybavení jsou vytvářeny aspekty (\asp{Jiskřící čepel, Meč takzvaného Taranise, Kalich devítí duší, Havraní spár, Zaklínačské vybavení})
\item dovednosti se používají v akcích a akce zase jen upravují či těží z aspektů
\end{itemize}

Výčet výše snad vyjasnil, že aspekty jsou podstatné. Co ale tedy takový aspekt je? Aby definice byla dostatečně obecná, ale nikoli zase vyprázdněná, je potřeba jí formulovat poněkud beztvaře: Aspekt je sousloví, fráze, věta, která popisuje jakoukoliv významnou vlastnost světa či významnou věc v onom světě. \\

Pravidla aspektů unikátním způsobem kloubí vyprávění a mechanické hraní. Na jednu stranu, aspekty jsou deskriptivní - \textit{popisují}, jak scéna/místo vypadá, jaké postavy jsou; na stranu druhou je aspekty preskriptivní - \textit{stanovují}, jak scéna/místo \textit{odteď vypadá} (\asp{Všude to hoří, Po kolena ve vodě}) a jaké postavy \textit{odteď budou} (\asp{Trochu zkleslý, Zcela zmanipulovaný, Téměř mrtvý}). Tedy, mechanickým hraním lze svět kolem aktivně utvářet a spravovat, ne se jej pouze účastnit; cokoliv, co je aspektem, ve světě skutečně je a děje se. \\

Bývá zvykem, že většinu scén připravuje vypravěč, a tedy i aspekty scény. Nicméně, jakmile se hráči ve scéně objeví a začnou s aspekty interagovat, upravovat stávající a vytvářet nové, i narativ se podle toho mění - příběh vyprávějí oni. Tomu v těchto pravidlech říkáme, že všichni hráči mají \textit{vypravěčskou pravomoc}. Role vypravěčě je prostě reagovat na vyprávění ostatních hráčů - skrze nehráčské postavy (neboť např. zná jejich motivaci, cíle apod.) a skrze herní prostředí (zná, co je za těmito dveřmi, kam vede tato chodba a jak citlivá je past pod podlahou). Je ovšem zcela zásadní si uvědomit, že vyprávění příběhu je opravdu záležitostí (povinností) každého z družiny, nikoli pouze vypravěče. \footnote{A taky je to zábavnější (a o to nám jde!). Vypravěč, byť jistě v dobré vůli, nedokáže připravit dobrodružství tak, aby se všem líbilo. Je proto na zodpovědnosti každého hráče, aby pomocí své vypravěčské pravomoci vytvořil příběh, který pro něj (jeho postavu) bude dokonalý.} \\

V další části textu detailně rozebereme, s jakými aspekty se lze potkat, co vyjadřují, jak je měnit či vytvářet. Ještě předtím je ale dobré rozebrat, jak vypadá dobrý aspekt.

\subsection{Jak vypadá dobrý aspekt}
\label{sec:jakvypadadobry}

Jak jsme měli možnost vidět, aspekty tvoří hru a hra tvoří aspekty. Nicméně, aby aspekt svoji roli plnil dobře, existuje několik zásad, jak jej formulovat. Později (v další sekci) uvidíme, že takto formulovaný aspekt se jednoduše vynucuje, tedy je pro nositele dobrým zdrojem bodů osudu (pokud je aspekt spojen s postavou). Pro jednoduchost jsou vlastnosti líčeny na příkladu aspektů postav.\\

Tak v první řadě, dobrý aspekt je \textbf{dvousečný} \footnote{Ve skutečnosti žádná ``sečnost'' neexistuje, tj. neexistuje ``negativní a pozitivní'' aspekt. Aspekt je prostě věta, co sama o sobě nenese žádné hodnocení.}; říká, v čem se postava může blýsknout a v čem také může selhat. Díky tomu aspekt dobře slouží jak hráči, který jej nejspíše vyvolá, aby vysvětlil, jak mu v jeho počínání pomůže, tak stejně dobře poslouží vypravěči, aby upozornil, jak ho tento aspekt dostává do problémů. A v každém případě, dvousečně formulované aspekty jsou prostě zábavnější: \asp{Nikdo mě nikdy nemůže porazit} je munchkinovské, kdežto \asp{Říkám o sobě, že mě nikdy nikdo nemůže porazit} se může hezky zvrtnout. Podobně, aspekt \asp{Chodicí encyklopedie} neumožňuje žádné jiné použití než ukázka toho, jak inteligentní a sečtělá postava je. Na druhou stranu aspekt \asp{Podivínský milovník knih} lze kromě téhož účelu použít i v situaci, kdy se projeví, že postava většinu svého života strávila v knihovně...

Druhým důležitým specifikem dobrého aspektu je, že \textbf{říká více než jen jednu věc}. Zde začněme příkladem: aspekt \asp{Hledaný} je poměrně obecný, vlastně až přílíš. Je z něj patrná pouze jediná věc, totiž že postavu někdo hledá. Kdežto aspekt \asp{Hledá mě Noční let} specifikuje, že postavu hledá právě Noční let - a že se tedy vůči němu nějak provonil, má nějaké informace, apod. Najednou postava ožívá, má dimenzi navíc - se světem ji nyní spojuje vztah s Nočním letem. \footnote{Zde by šlo navrhnout, že specifikací Nočního letu už není možné, aby postavu hledal někdo jiný, tj. hráč přichází o možnosti vynucení. To je ovšem na šikovnosti Vypravěče - vždy se přeci může stát, že Noční let si najme někoho dalšího, že jeho identitu prodá apod.}\\

Poslední vlastností, kterou zde uvedeme, je \textbf{jasná formulace}. Aspekt je nějaká věta psaná nedokonalým jazykem a konečným počtem slov. I tak by měl být ale formulován takovým způsobem, aby všichni měli představu, co znamená. V opačném případě jej nemohou správně použít. Příkladem budiž aspekt \asp{Vzpomínky na dětství} - je cítit, že vyvolává jistou nostalgii a že pro postavu bylo asi dětství podstatné při formování osobnosti. Ale \textit{jakým způsobem podstatné?} To z aspektu není zřejmé. Kdežto aspekty \asp{Dětství plné šikany, Vyrůstal jsem mezi šlechtou na zámku, Za všechno dlužím rodině} jasně dokreslují, o jaké dětství asi mohlo jít.\\

Pozorný čtenář nyní namítne, že v uvedených příkladech u každého ze specifik se ne vždy odrážejí i specifika ostatní. A to je pravda. Je vždy totiž podstatné brát kontext - pokud se hráči shodnou, že i neideálně formulovaný aspekt je pro ně dost dobrý (resp. si ho vždy dokáží přeformulovat ve své hlavě tak, aby byl dost dobrý), není důvod trávit hodiny vymýšlením chytrých frází, které by všechny požadavky odškrtly. Dost pravděpodobně by takto formulovaný aspekt působil uměle, slovosled by byl kostrbatý a věta strašně dlouhá.

\section{Body osudu a obnova}
\label{sec:body-osudu-obnova}




\end{document}

%%% Local Variables:
%%% mode: LaTeX
%%% TeX-master: "../main"
%%% End:
