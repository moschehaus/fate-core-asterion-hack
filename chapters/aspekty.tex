\documentclass[../main.tex]{subfiles}
\graphicspath{{\subfix{../graphics/}}}
\begin{document}

\section{Aspekty, narativ a příběhová pravomoc}
\label{sec:aspekty-narativ-pribeh}

``Pod vším ve Fate hledej aspekty''. Skutečně, aspekty hrají ve Fate Core zcela zásadní roli. Jsou tím, co je na těchto pravidlech typické a bez nich by Fate Core nebyl Fate Core.

\begin{itemize}
\item postavy, nejdůležitější součást hry, jsou definovány a vymezeny pomocí aspektů (\asp{Špinavý žoldák z Hlubiny, Hbitý skřítek z Imchejle, Čestný rytíř z řádu Devíti hvězd, Temperamemtní učednice Alwarina Bílého})
\item herní prostředí, tj. parkety, na kterých se odehrává drama postav, je nastaveno pomocí aspektů (\asp{Největší město na Taře, Prokleté údolí, Hora, kde sídlí bohové, Jeskyně plná drahokamů})
\item momentální stav prostředí, nálada scény, důležitý okamžitý detail jsou všechno věci vyjádřené pomocí aspektů (\asp{Podlaha v plamenech, Překážka v cestě, Tma jako v pytli, Něco tu sakra nehraje...})
\item významné předměty a vybavení jsou vytvářeny aspekty (\asp{Jiskřící čepel, Meč takzvaného Taranise, Kalich devítí duší, Havraní spár, Zaklínačské vybavení})
\item dovednosti se používají v akcích a akce zase jen upravují či těží z aspektů
\end{itemize}

Výčet výše snad vyjasnil, že aspekty jsou podstatné. Co ale tedy takový aspekt je? Aby definice byla dostatečně obecná, ale nikoli zase vyprázdněná, je potřeba jí formulovat poněkud beztvaře: Aspekt je sousloví, fráze, věta, která popisuje jakoukoliv významnou vlastnost světa či významnou věc v onom světě. \\

Pravidla aspektů unikátním způsobem kloubí vyprávění a mechanické hraní. Na jednu stranu, aspekty jsou deskriptivní - \textit{popisují}, jak scéna/místo vypadá, jaké postavy jsou; na stranu druhou je aspekty preskriptivní - \textit{stanovují}, jak scéna/místo \textit{odteď vypadá} (\asp{Všude to hoří, Po kolena ve vodě}) a jaké postavy \textit{odteď budou} (\asp{Trochu zkleslý, Zcela zmanipulovaný, Téměř mrtvý}). Tedy, mechanickým hraním lze svět kolem aktivně utvářet a spravovat, ne se jej pouze účastnit; cokoliv, co je aspektem, ve světě skutečně je a děje se. \\

Bývá zvykem, že většinu scén připravuje vypravěč, a tedy i aspekty scény. Nicméně, jakmile se hráči ve scéně objeví a začnou s aspekty interagovat, upravovat stávající a vytvářet nové, i narativ se podle toho mění - příběh vyprávějí oni. Tomu v těchto pravidlech říkáme, že všichni hráči mají \textit{vypravěčskou pravomoc}. Role vypravěčě je prostě reagovat na vyprávění ostatních hráčů - skrze nehráčské postavy (neboť např. zná jejich motivaci, cíle apod.) a skrze herní prostředí (zná, co je za těmito dveřmi, kam vede tato chodba a jak citlivá je past pod podlahou). Je ovšem zcela zásadní si uvědomit, že vyprávění příběhu je opravdu záležitostí (povinností) každého z družiny, nikoli pouze vypravěče. \footnote{A taky je to zábavnější (a o to nám jde!). Vypravěč, byť jistě v dobré vůli, nedokáže připravit dobrodružství tak, aby se všem líbilo. Je proto na zodpovědnosti každého hráče, aby pomocí své vypravěčské pravomoci vytvořil příběh, který pro něj (jeho postavu) bude dokonalý.} \\

V další části textu detailně rozebereme, s jakými aspekty se lze potkat, co vyjadřují, jak je měnit či vytvářet. Ještě předtím je ale dobré rozebrat, jak vypadá dobrý aspekt.

\subsection{Jak vypadá dobrý aspekt}
\label{sec:jakvypadadobry}



\section{Body osudu a obnova}
\label{sec:body-osudu-obnova}


\end{document}

%%% Local Variables:
%%% mode: LaTeX
%%% TeX-master: "../main"
%%% End:
