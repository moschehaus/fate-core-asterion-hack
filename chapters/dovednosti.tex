\documentclass[../main.tex]{subfiles}
\graphicspath{{\subfix{../graphics/}}}
\begin{document}
Schopnosti postavy vystihují kromě aspektů zejména dovednosti. Vypovídají o tom, jak je která postava zdatná v oblastích relevantních pro příběh. Seznam dovedností se tedy může značně lišit podle povahy settingu a příběhu. \footnote{I v témže settingu je samozřejmě myslitelné používat jiné dovednosti. Jedná-li se například o dobrodružství odehrávající se ve škole šermu, je vhodné postihnout oblast boje více dovednostmi (rapír, fleret, šavle apod.). To stejné pro příběh o magii, kde jednotlivé obory magie mohou představovat oddělené dovednosti}.

Seznam níže je kompromis, na kterém jsme se jako družina shodli. Oproti pravidlům M16 se změny dočkaly zejména sociální a jiné ``soft'' dovednosti a boj. Konkrétně, v systému M16 je boj rozdělen na velké množství stylů: \textit{Rváč, Šermíř, Pirát, Těžkooděnec, Hrdlořez} a další, přičemž každý styl má přiděleno několik málo zbraní, které dokáže využívat. Tím však vzniká problém s balancováním (všechny styly neměly zdaleka stejně dobře využitelný set zbraní) a pro některé je toto dělení cela umělé (proč \textit{Hrdlořez} umí používat tesák a nikoli o 10 cm delší meč). Jistým řešením pak bylo stanovit dovednosti podle třídy zbraní: \textit{Jednoruční zbraně, Obouruční zbraně, Tyčové zbraně, Vrhací zbraně} etc. Nakonec jsme ovšem udělali ještě větší krok k jednoduchosti \footnote{V porovnání se sociálními dovednostmi, kdy v pravidlech M16 všeobjímající dovednost \textit{Zaujmutí pozornosti} zahrnovala \textit{Empatii, Kontakty, Provokaci i Vztahy}, je tento krok spíše obrovský. Skupině toto dělení ale vyhovuje.} a oblast boje rozdělili pouze na boj na dálku/na blízku a se zbraní/beze zbraně.

\begin{itemize}
\begin{multicols}{3}
[
\item \textit{Boj beze zbraně}
\item \textit{Boj na dálku}
\item \textit{Boj se zbraní}
\item \textit{Empatie}
\item \textit{Kondice}
\item \textit{Kontakty}
\item \textit{Klam}
\item \textit{Jezdectví}
\item \textit{Medicína}
\item \textit{Mobilita}
\item \textit{Pozornost}
\item \textit{Provokace}
\item \textit{Řemesla}
\item \textit{Skrývání}
\item \textit{Učenost}
\item \textit{Vůle}
\item \textit{Vyšetřování}
\item \textit{Vztahy}
\item \textit{Zlodějina}
]
\end{multicols}
\end{itemize}

\section{Boj beze zbraně}
\label{sec:bojbezezbrane}

Dovednost \textit{Boj beze zbraně} pokrývá formy boje, při kterém není útočník jakkoliv vyzbrojen. Může se jednat o improvizovaný boj v hospodském stylu, či např. bojové umění.

Neboť je pro naší skupinu oblast boje podstatná, jsou popisy více rozvedené v kapitole \ref{chap:jdesenavec}, stejně tak jako příklady dobré praxe.

\subsection*{Překonání}
\label{subsec:boj-prekonani}
\prekonani

\textit{Boj beze zbraně} se mimo fyzický konflikt prakticky nevyužívá, tedy jeho využítí pro akci \texttt{Překonání} je minimální.\footnote{Hody na \texttt{Překonání} jsou totiž nejčastější ve střetech, viz sekce \ref{chap:jdesenavec}.} Jedná-li se například o sportovní boj, či turnaj, v němž postavy chtějí na někoho zapůsobit (resp. musí překonat jeho neochotu), lze \textit{Boj beze zbraně} v těchto speciálních sitaucích použít.

\subsection*{Vytvoření výhody}
\label{subsec:boj-vytvoreni}
\vytvoreni

Vytváření výhod je vedle útočení nejčastějším použítím \textit{Boje beze zbraně}. Můžete popsat různé chvaty, po kterých bude mít soupeř \asp{Vyražený dech}, využít své judo a hodit protivníka, čímž se ocitne \asp{Na zemi}, nebo se k němu dostat nablízko, spojit ruce a ládovat ho, během toho, co je \asp{V klinči}. Dokonce se lze pokusit odkoukat bojovníkův styl a díky tomu pak \asp{Znám jeho příští krok}. Možností je vskutku mnoho a je hezké, když jsou popisy pestré.


\subsection*{Útok}
\label{subsec:boj-utok}
\utok

To je všeříkající. \textit{Boj beze zbraně} je ideální dovednost (a prakticky jediná) na způsobení zranění ve fyzickém konfliktu, pokud není útočící postava ozbrojena. Neboť se jedná o boj zblízka, je ovšem třeba se nacházet ve stejné zóně jako protivník.

\subsection*{Obrana}
\label{subsec:boj-obrana}
\obrana

\texttt{Obrana} pomocí \textit{Boje beze zbraně} je jeden ze způsobů, jak se vyhnout zranění ve fyzickém konfliktu. Stejně tak tuto akci lze použít k překažení soupeřova pokusu o \texttt{Vytvoření výhody}, pokud se jedná o pokus, při kterém \textit{kopání do žeber nepomůže.}

\section{Boj na dálku}
\label{sec:bojnadalku}

Základem fyzického konfliktu na dálku (střelecký souboj např.) je dovednost \textit{Boj na dálku}. Jako konflikt na dálku se považuje situace, kdy protivníci nejsou ve stejné zóně. Běžné střelné zbraně umožňují pálit do vedlejší zóny, existují ovšem výjimky. Samosebou, pro použití dovednosti \textit{Boj na dálku} je taková zbraň potřeba; pro diskuzi, kam se řadí vrhací zbraně (oficiální i improvizované), viz sekce \ref{chap:jdesenavec}.

\subsection*{Překonání}
\label{subsec:dboj-prekonani}
\prekonani

Ve speciálních případech jako je lukostřelecký turnaj či jiná sportovní střelba je možné použít \textit{Boj na dálku} pro hod na \texttt{Překonání}. Ve většině případů se ovšem tato dovednost používá ve fyzickém konfliktu, tedy nikoli ve střetech.

\subsection*{Vytvoření výhody}
\label{subsec:dboj-vytvoreni}
\vytvoreni

Základní dovedností pro vytváření výhod popsaných jako obtížné střelecké manévry je právě \textit{Boj na dálku}. Může se jednat o \asp{Krycí palbu}, odstřelovačské zalehnutí a s ním spojené \asp{Zamíření}, nebo jen snaha protivníka vystresovat množstvím \asp{Šípů ve vzduchu}. Stejně jako v konfliktu nablízko, je pěkné, když jsou popisy pestré.

\subsection*{Útok}
\label{subsec:dboj-utok}
\utok

Jako předchozí bojové dovednosti, \textit{Boj na dálku} se používá, když postava chce jinou fyzicky zranit a nenachází se přitom ve stejné zóně. Jestliže se postavy nachází ve stejné zóně, nelze (s klasickou zbraní) útočit pomocí \textit{Boje na dálku}.

\subsection*{Obrana}
\label{subsec:dboj-obrana}
\obrana

Narozdíl od jiných bojových dovedností, tato se příliš nehodí k obraně proti nepřátelské střelbě (k čemuž typicky slouží \textit{Mobilita}). Lze ji ovšem použít v případech, kdy se jiná postava pokouší o akci \texttt{Překonání} nebo \texttt{Vytvoření výhody} a příběhově dává smysl mu bránit střelbou.

\section{Boj se zbraní}
\label{sec:bojsezbrani}

Tato dovednost je základní ve fyzických konfliktech, pokud je postava vybavena zbraní. Co se dá ještě (už) považovat za zbraň je věcí narativu a settingu; pro férovost je ovšem dobré, aby tasení/přezbrojení nebylo automatické (bez příslušného triku), neboť pak dovednost \textit{Boj beze zbraně} ztrácí význam.

Stejně jako u \textit{Boje beze zbraně} se lze o použítí této dovednosti více dozvědět v sekci \ref{chap:jdesenavec}.

\subsection*{Překonání}
\label{subsec:zboj-prekonani}
\prekonani

Totožné jako u dovednosti \textit{Boj beze zbraně} \ref{subsec:boj-prekonani}.

\subsection*{Vytvoření výhody}
\label{subsec:zboj-vytvoreni}
\vytvoreni

Vytvářet výhody pomocí \textit{Boje se zbraní} lze velmi pestře. Může se jednat o sérii rychlých útoků, po kterých protivník \asp{Neví, odkud přijde další}, šermířskou finesu, co protivníka zanechá \asp{Beze zbbraně}, a nebo třeba snahu odpozorovat soupeřův styl a mít ho tak \asp{Úplně přečteného}. 

\subsection*{Útok}
\label{subsec:zboj-utok}
\utok

Prakticky kdykoliv, kdy se postava snaží jinou postavu zranit zbraní při fyzickém konfliktu, bude se jednat o hod na \textit{Boj se zbraní}, pokud se samozřejmě nachází obě postavy ve stejné zóně.

\subsection*{Obrana}
\label{subsec:zboj-obrana}
\obrana

Bránit se zbraní lze proti většině pokusům o útok (jak zbraní tak beze zbraně) vedeným zblízka. Stejně tak lze akci \texttt{Obrana} využít, kdyžpostava tvoří aktivní opozici nepříteli proti akci, při které protivníkovi nepomáhá být \textit{bit jako pes}.

\section{Empatie}
\label{sec:empatie}

Empatická postava dokáže číst nálady ostatních, objevit, jací jsou, a prokouknout snahu o neupřímnost. V jistém smyslu se jedná o emocionální obdobu dovednosti \textit{Pozornost}, neb umožňuje zachytit změny v náladách.

\subsection*{Překonání}
\label{subsec:empatie-prekonani}
\prekonani

Primární využití \textit{Empatie} na \texttt{Překonání} je v situacích, kdy se postava snaží zachytit změnu nálad, přístupu, záměru či postoje jiné postavy (nebo když dává smysl, aby toto postava mohla zjistit bez aktivní snahy). Příkladem budiž zrada spojence, nával vzteku berserka a další. Ve většině případů se ovšem \textit{Empatie} nepoužívá na překonávání přímo; pouze připravuje půdu pro použití jiných sociálních dovedností jako jsou \textit{Vztahy} či \textit{Klam} (například odhalením aspektu na kartě postavy).

\subsection*{Vytvoření výhody}
\label{subsec:empatie-vytvoreni}
\vytvoreni

Empatie vytváří výhody přečtením duševního stavu jiné postavy. Díky tomu může například odhalit aspekty jiných postav, typicky povahové jako \asp{Násilník}, \asp{Labilní bipolárník}, \asp{Nízké sebevědomí} a ty pak dále využít pro jiné akce jinými dovednostmi. Jinou možností je přečíst okamžitý stav postavy po nějaké události, která na ní mohla mít vliv - např. odhalit, že \asp{Tohle mě fakt vzalo}, či že \asp{Nevím, co dál}.

Cílová postava může použít akci \asp{Obrana} dovedností \textit{Klam, Vztahy}, aby ztěžovala postavě přečíst její duševní stav.

\subsection*{Útok}
\label{subsec:empatie-utok}
\utok

Empatií nelze ostatní zraňovat. V sociálním či argumentačním konfliktu však lze výhodně touto dovedností vytvořit výhodu a tu pak použít například na útok \textit{Provokací}.

\subsection*{Obrana}
\label{subsec:empatie-obrana}
\obrana

Tato dovednost je stěžejní při obraně o pokusu o oklamání, skrývání postranních záměrů útočníka. Je tedy stěžejní, jestliže je třeba argumentační spory vyřešit mechanicky.

\section{Jezdectví}
\label{sec:jezdectvi}

Název dovednosti \textit{Jezdectví} je všeříkající. V našem hraní není používání této dovednosti příliš rozšířené, i tak ale dává smysl ji zde úvest. Kupříkladu v dobrodružství, kde se hodně cestuje a družina si přeje cestování nepřeskakovat, či například spěchají, však dává smysl \textit{Jezdectví} ovládat.

\subsection*{Překonání}
\label{subsec:jezdectvi-prekonani}
\prekonani

Pro akci \texttt{Překonání} vyjadřuje dovednost \textit{Jezdectví} obdobu dovednosti \textit{Mobilita} pro případy, kdy k pohybu dochází na koni (praseti, kočáře, dostavníku aj.) Umožňuje tak překonávat \asp{Zatarasenou cestu}, dostat se z \asp{Naprostého obklíčení} nebo snížit nepřítelův \asp{Náskok}. 

\subsection*{Vytvoření výhody}
\label{subsec:jezdectvi-vytvoreni}
\vytvoreni

I zde je použití poměrně pestré: při jízdě lze využít \asp{Nejkratší cesty}, v honičce si bravurní jízdou získat \asp{Náskok} či na hřbetu předvádět \asp{Kousky hodné cirkusáka}. 

\subsection*{Útok}
\label{subsec:jezdectvi-utok}
\utok

V základním případě se \textit{Jezdectví} nepoužívá k způsobování zranění. Jedná-li se například o souboj ze sedla a obě postavy mají alespoň elementární dovednost jízdy, používá se k útoku nějaké bojové dovednosti. Nicméně, v případě velmi specifické situace (např. rytířské klání) lze o ofenzivním použití uvažovat. Situaci však samozřejmě může zcela měnit vhodný trik...

\subsection*{Obrana}
\label{subsec:jezdectvi-obrana}
\obrana

Tato dovednost je základní pro obranu před pokusy ostatních vytvořit si proti vám výhody při jezdeckých konfliktech, například při pokusu o předjetí, natlačení do překážky atd. V situacích, kdy je smyslem protivníkova útoku zranit osedlaného oře, lze použít \textit{Jezdectví} přímo na \texttt{Obranu} proti takové snaze. 

\section{Klam}
\label{sec:klam}


Klamání je o matení, lhaní, zatajování, zkreslování skutečnosti a dezinterpretování. Jedná se o velmi univerzální dovednost, kterou lze použít ve fyzických konfliktech, argumentačních i sociálních.

\subsection*{Překonání}
\label{subsec:klam-prekonani}
\prekonani

Snaha oblafnout stráže, odvést pozornost od skutečného záměru, přesvědčit někoho o lži či naopak využít někoho k získání informace, protože už nějaké vaší lži věří, jsou všechno hody na \texttt{Překonání} pomocí \textit{Klamu}. Stejně tak lze \textit{Klamem} odstraňovat aspekty vytvořené protivníkovým \texttt{Vytvářením výhod}mařit vyšetřování: (odstraňovat tak aspekty jako \asp{Jsou mi na stopě}), roznášet nepravdivé informace a ztěžovat tak snahu o vedení politické kampaně (odstranit \asp{Znám jako křivák}) apod. Možností je vskutku mnoho, je ale třeba mít na paměti, že s použítim \textit{Klamu} je vždy třeba v nějakém významu neříkat či zastírat pravdu; jinak by se měl použít hod na jiné dovednosti, jako jsou \textit{Provokace, Kontakty, Vztahy}.

\subsection*{Vytvoření výhody}
\label{subsec:klam-vytvoreni}
\vytvoreni

Při fyzických konfliktech lze \textit{Klam} využít např. při šermířských a chodeckých manévrech využívající plášť a fingovat tak útok odjinud (aspekty jako \asp{Neví odkud přijde}), předstírat nějaké zranění či neúspěch a protivníka přesvědčit, že jsem \asp{Zcela bezbranný} a jiné. Důležité je rozlišovat, kdy se stále jedná o fyzický konflikt a kdy spíš o střet (viz kapitoly \ref{chap:jdesenavec}) - například snaha protivníka zcela přesvědčit, že bojuje s tím nesprávným týpkem je ve většině případů solidní střet a nikoli rozhovor při vyměňování úderů.

Mimo fyzický konflikt je \textit{Klam} též velmi užitečný. Díky němu si může hráč vytvořit zástěrky jako \textit{Znám jako důvěrník}, objevit aspekty na kartě jiných postav tím, že vám sdělí nějaké informace poté, co podlehnou vašim lžím nebo třeba roznášet dezinformace (v kombinací např. s \textit{Kontakty}).

\subsection*{Útok}
\label{subsec:klam-utok}
\utok

Pouhá lež nemůže být použita k útoku. Je to však výhodná dovednost, díky které lze využít řadu výhod a ty pak vytěžit při útoku pomocí \textit{Provokace}.

\subsection*{Obrana}
\label{subsec:klam-obrana}
\obrana

Efektivním klamáním se lze bránit pokusům ostatních postav o přečtení vašich úmyslů pomocí \textit{Empatie}, zametání stop zase umožňuje bránit se proti následným pokusům o \textit{Vyšetřování}. Stejně jako v předchozích akcích, použití \textit{Klamu} je potenciálně velmi široké, vždy je ale třeba dodat patřičný narativ a nějakým způsobem lhát/zastírat pravdu.

\section{Kondice}
\label{sec:kondice}

Fyzické schopnosti vedle \textit{Mobility} nejlépe vyjadřuje právě \textit{Kondice}. Postava s vysokou kondicí má velkou výdrž, houževnatost a taky hrubou fyzickou sílu. 

\subsection*{Překonání}
\label{subsec:kondice-prekonani}
\prekonani

\textit{Kondice} nachází uplatnění při překonání libovolných překážek, které vyžadují fyzickou sílu - odvalení \asp{Stromu přes cestu}, vyražení \asp{Dveří z oceli} nebo rozmlácení \asp{Ochranného silového pole}. 

\subsection*{Vytvoření výhody}
\label{subsec:kondice-vytvoreni}
\vytvoreni

Vytvářet výhody pomocí \textit{Kondice} je snadné; ve fyzickém konfliktu ji lze využít na provádění silových triků, jako je sražení postavy, která se stane \asp{Uzemněnou} či umístění nepřítele do wrestlerského chytu \asp{Body lock} a další. Mimo fyzický konflikt lze \textit{Kondicí} poškodit konstrukci, dokud není \asp{Těsně před kolapsem}, nanosit pytle písku a vytvořit tak \asp{Provizorní krytí}.

\subsection*{Útok}
\label{subsec:kondice-utok}
\utok

Kondici \textit{nelze použít} k fyzickému ublížení jiných postav - k tomu slouží bojové dovednosti. Lze ji však výhodně použít k umístění situačních aspektů na protivníka a ty pak těžit při útocích pomocí např. \textit{Boje beze zbraně}.

\subsection*{Obrana}
\label{subsec:kondice-obrana}
\obrana

Stejně jako v případě útoku,\textit{nelze kondici použít} k obraně před akcí \texttt{Útok} nějakou bojovou dovedností. Nachází ovšem uplatnění, kdy hrubá síla může zamezit protivníkovy pokusy o akci \texttt{Překonání} - držení dveří před vyražením, zalehnutí při pokusu o útěk atp.

\section{Kontakty}
\label{sec:kontakty}

Dovednost \textit{Kontakty} hovoří o tom, jak dobré znalosti postava má a jak dobře je dokáže získávat. Roli hraje jak pasivní úroveň dovednosti, tak výsledný hod na kostkách.

\subsection*{Překonání}
\label{subsec:kontakty-prekonani}
\prekonani

Typickým využitím \textit{Kontaktů} při akci \texttt{Překonání} je naleznutí hledané postavy/společnosti v herním prostředí, typicky ve městě, na vesnici. Na odhalení skrývající postavy se samozřejmě využívají jiné dovednosti, typicky \textit{Pozornost, Vyšetřování}. To může obnášet optávání se lidí na ulici, vyvěšení plakátu, hledání v databázích a další. 

\subsection*{Vytvoření výhody}
\label{subsec:kontakty-vytvoreni}
\vytvoreni

Znalost lidí a schopnost je vyhledat bezesporu umožňuje vytvářet mnoho výhod. Tak například lze při pokusu o verbování žoldáků nalézt \asp{Nejlepšího šermíře ve městě}, nebo od svých kontaktů zjistit, že osoba zájmu je \asp{Pověstný llhář}. Zde je dobré podotknout, že tyto aspekty \textit{nemusí odrážet pravdu} - ve skutečnosti mluví o reputaci tak, jak vám ji vaše kontakty popsaly.
Obdobně lze \textit{Kontakty} využít v rámci sociálního konfliktu, kupříkladu k využití své informační sítě pro roznesení pomluvy (\asp{Podvádí svoji ženu}) a z ní pak vytěžit maximum použitím \textit{Provokace}.

\subsection*{Útok}
\label{subsec:kontakty-utok}
\utok

Tohle není dobrá dovednost pro vedení útoku. Na druhou stranu, v dlouhodobém konfliktu je nepostradatelná pro vytváření výhod.

\subsection*{Obrana}
\label{subsec:kontakty-obrana}
\obrana

Použití \textit{Kontaktů} pro obranu (samozřejmě v sociálním konfliktu) je možné s trochou důvtipu. Lze je například použít pro obranu proti roznesení nepravdivé informace (vytvoření výhody \textit{Klamem}) s tvrzením, že vaše informační síť tyto informace neroznáší. Dále se lze bránit proti přímému pokusu o oklamání, neboť vás vaše kontakty dobře informují. V neposlední řadě, maření hodů na \textit{Vyšetřování} lze též efektivně docílit použitím kontaktů.

\section{Medicína}
\label{sec:medicina}

\textit{Medicína} představuje poněkud netradiční dovednost. Její využití je hodně limitované - týká se zejména ošetřování, léčení a podobné podpůrné činnosti. 

\subsection*{Překonání}
\label{subsec:medicina-prekonani}
\prekonani

Většina použití \textit{Medicíny} bude právě na akci \texttt{Překonání}. Typicky se jedná o situaci, kdy se ošetřující snaží zmírnit následky po fyzickém konfliktu - provést nacvičený pohyb a nahodit tak  \asp{Vykloubené rameno}, obvázat \asp{Pořezané ruce} a nebo se pokusit o zázrak a od Lamiusova soudu přivézt \asp{Prakticky vykrveného}. V takových případech si postava hází na \textit{Medicínu} proti pasti dané závažnosti následku (viz sekce \ref{chap:jdesenavec}).


\subsection*{Vytvoření výhody}
\label{subsec:medicina-vytvoreni}
\vytvoreni

\textit{Medicínou} lze i v určitých případech vytvářet výhody. Typicky se jedná o výhody pramenící z přípravy těla na fyzickou činnost - zabezpečení \asp{Dobrého odpočinku}, \asp{Svalové masáže} apod. Druhou možností pak je využívat své znalosti anatomie a fyziologie nepřítelova těla proti němu - znalost \asp{Slepého bodu}, \asp{Mrtvého úhlu} nebo jeho \asp{Sleposti v šeru} mohou být všechno znalosti použité k získání výhody. Je ovšem vždy na místě se ptát, zda-li se k takovým účelům nehodí více \textit{Učenost}; v případě exotických (myšlenkových) bytostí by tomu tak skoro vždy mělo být.

\subsection*{Útok}
\label{subsec:medicina-utok}
\utok

Tohle nejde. \textit{Medicínu} použijete až \textit{po} (zdařené) akci \texttt{Útok}...

\subsection*{Obrana}
\label{subsec:medicina-obrana}
\obrana

Medicína nelze použít na aktivní obranu např. ve fyzickém konfliktu. V některých speciálních situací, jako je asistence postavě odolávající nemoci/jedu lze však \textit{Medicínu} použít, i když stojí za zvážení, zda-li takový pokus lépe nevystihuje akce \texttt{Vytvoření výhody}...

\section{Mobilita}
\label{sec:mobilita}

Mobilita je o pohybu - běhání, šplhání, skákání, uhýbání, lezení a další. Zkrátka vystihuje, jak dobré jsou postavy v hýbání svým tělem. Společně s \textit{Kondicí} popisují fyzické schopnosti postavy. V pravidlech M16 dovednost \textit{Mobilita} neexistuje - tvoří ji oddělené dovednosti \textit{Mrštnost, Atletika, Plavání, Lezení}. 

\subsection*{Překonání}
\label{subsec:mobilita-prekonani}
\prekonani

\textit{Mobilitu} použijete k překonání libovolné překážky, která vám brání v pohybu (např. situační aspekty bránící přesunu mezi zónami), ať už je třeba přeplavat \asp{Rozvodněnou řeku}, přeskočit \asp{Nekonečnou propast} nebo prokličkovat mezi obchodníky na \asp{Přelidněném trhu}. \textit{Mobilita} je zejména v fyzickém konfliktu zkrátka nepostradatelná.
Své uplatnění má ale i ve střetech jako jsou honičky a pronásledování. O tom, jak se vám daří zkracovat vzdálenosti a vypořádávat se s výhodami vašich nepřátel hovoří právě hody na \texttt{Překonání} pomocí \textit{Mobility}.

\subsection*{Vytvoření výhody}
\label{subsec:mobilita-vytvoreni}
\vytvoreni

Vytváření výhod pomocí \textit{Mobility} je pestré. Může se jednat o \asp{Akrobatické manévry}, díky kterým je pro vás snazší vyhnout se střelbě, vyšplhání na \asp{Místo s výhledem} pro vaší vlastní střelbu nebo třeba nasazení \asp{Kličkování mezi haraburdím} a setřesení pronásledovatelů.

\subsection*{Útok}
\label{subsec:mobilita-utok}
\utok

\textit{Mobilita} není primárně útočná dovednost. Své využítí ve fyzických konfliktech má při všech ostatních akcích.

\subsection*{Obrana}
\label{subsec:mobilita-obrana}
\obrana

Dovednost \textit{Mobilita} nabízí jiný způsob, jak se bránit zranění ve fyzických konfliktech (kromě bojových dovedností); bez triků je to dokonce jediná dovednost, jak se bránit útoku pomocí \textit{Boje na dálku}. Jiný způsob, jak \textit{Mobilitu} použít, je aktivně bránit cizím pokusům o \texttt{Překonání} či \texttt{Vytvoření výhody} - např. když postavu chce někdo skopnout z \asp{Rozviklaného žebříku}, nebo ji podrazit nohy a dostat ji \asp{Na zem}. 

\section{Pozornost}
\label{sec:pozornost}

\textit{Pozornost} je ostražitosti, obezřetnosti a obecně všímání si věcí. Doplňuje \textit{Vyšetřování}, oproti němu však \textit{Pozornost} funguje okamžitě a umožňuje zaznamenat jen věci, které jsou postřehnutelné vlastními smysly bez hlubšího zkoumání. 

\subsection*{Překonání}
\label{subsec:pozornost-prekonani}
\prekonani

Hody na \texttt{Překonání} pomocí \textit{Pozornosti} jsou většinou hodně podobné a týkají si povšimnutí si něčeho nevýznamného nebo skrytého. Tak například jí hráč může použít na zaznamenání \asp{Skrytě nošené zbraně}, zaregistrování \asp{Divného mechanismu} dřív, než na něj šlápne nebo včas reagovat na \asp{Padající kamení}. Ve většine případů se jedná o hody proti pasivní opozici (proti pasti) a je též dobré nezaměňovat \textit{Pozornost} za \textit{Vyšetřování}. \underline{Příklad}:

Nepřátelská postava si pomocí \textit{Skrývání} vytvořila výhodu \asp{Nikdo o mě neví} a chystá se jí použít k přepadení hrdiny ze zálohy. \textit{Má-li hrdina podezření}, že se může v místnosti někdo schovávat (prozkoumává neznámé místo apod.), může si hodit na \texttt{Překonání} pomocí \textit{Pozornosti}, aby zjistil, jestli si jeho postava nepřítele ihned všimne. Pokud ale neuspěje, útočník je skrytý a nalézt jej pak už lze jen pomocí \textit{Vyšetřování}, např. když se hrdina rozhodne místnost pořádně prozkoumat.

\subsection*{Vytvoření výhody}
\label{subsec:pozornost-vytvoreni}
\vytvoreni

Vytváření výhod pomocí \textit{Pozornosti} skvěle ilustruje vypravěčskou pravomoc \textit{všech} postav. Hodem na \textit{Pozornost} lze zahlásit (odhalit), že postava v bortící se budově nalezla \asp{Únikovou cestu}, v nepřátelské linii si všimla \asp{Nenápadné slabiny}, nebo v hospodské rvačce stihla zareagovat na \asp{Kaluž rozlitého čehosi}, kdežto její sok nikoli. Za zdůraznění stojí, že takto vytvořené výhody \textit{jsou součástí světa}, neboť jsou to aspekty - a to, jestli se ve světě objeví zde závisí na hodu na \textit{Pozornost}. Naopak, není vhodné se ptát vypravěče, jestli na zemi náhodou není \asp{Kaluž rozlitého čehosi} \footnote{Ve skutečnosti by i takto šlo vytvořit výhodu, pouze by se ale jednalo o získání volných vyvolání na již existující aspekt a nikoli vytvoření nového aspektu; to druhé je víc cool.}, ale prostě říct, že tam je.

\subsection*{Útok}
\label{subsec:pozornost-utok}
\utok

Touto dovedností nelze vést útok.

\subsection*{Obrana}
\label{subsec:pozornost-obrana}
\obrana

Bránit se \textit{Pozorností} lze proti pokusům o akce s dovedností \textit{Skrývání} nebo např. odhalení, že \asp{Mě někdo sleduje} \footnote{Odhalení aspektu většinou spadá pod akci \texttt{Vytvoření výhody}, zde je ovšem \texttt{Obrana} myšlena jako aktivní opozice proti skrytému sledování.}

\section{Provokace}
\label{sec:provokace}

\textit{Provokace} je o rozčilování, urážení, zastrašování, zkrátka vyvolávání negativních emocích v ostatních a snahách jim ublížit. Použití této dovednosti by mělo dávat smysl - pokud se z ničeho nic rozhodnete duševně zranit vašeho nejlepšího přítele, nejspíš to bude brát jako vtip. Bohatě ale stačí nějaký váš aspekt (\asp{Nevymáchaná držka, Jazyk jako břitva}), nebo aspekt vašeho protivníka (\asp{Všechno bere moc vážně, Naivka}). 

\subsection*{Překonání}
\label{subsec:provokace-prekonani}
\prekonani

Akce \texttt{Překonání} pomocí \textit{Provokace} jsou dost často manipulativní. Lze někoho vyprovokovat k tomu, aby v návalu emocí udělal něco, co vy chcete (do nervového kolapsu rýt do \asp{Důvěřivého písaře}, dokud vám nevydá všechny spisy ohledně vážného zločinu), zastrašovat ho, vydírat ho nebo se ho pokoušet vyděsit. Takto použitá \textit{Provokace} dává smysl ve střetu nebo výzvě, kdy je třeba dosáhnout jistého cíle (získat informace a někam se vloupat) - pak se buď hází proti pasivní opozici, nebo, jde-li o důležitější postavy, proti \textit{Vůli} či jiné relevantní dovednosti.


\subsection*{Vytvoření výhody}
\label{subsec:provokace-vytvoreni}
\vytvoreni

\textit{Provokace} se skvěle hodí na vytváření výhod odrážející emoční stav svého protivníka: \asp{Dohnaný k šílenství, Neví co si počít, Vystrašený} atd. 

\subsection*{Útok}
\label{subsec:provokace-utok}
\utok

Speciální stránkou dovednosti \textit{Provokace} je, že se dá použít k provedení duševního útoku při argumentačním/sociálním konfliktu. Urážky, výhružky, mačkání spouštěčů jsou všechno věci, kterými lze protivníka duševně zranit. Zde ovšem hraje velkou roli kontext a vztah obou stran - urážka náhodného člověka na ulici mu duševní zranění nezpůsobí, stejně tak vyhrožování vašemu protivníkovi, pokud má velkou převahu atd.

O argumentačních a sociálních konfliktech se lze dočíst v sekci \ref{chap:jdesenavec}.

\subsection*{Obrana}
\label{subsec:provokace-obrana}
\obrana

Dovednost \textit{Provokace} vám nijak nepomůže při akci \texttt{Obrana}.

\section{Řemesla}
\label{sec:remesla}

Dovednost \textit{Řemesla} se týká vytváření a opravování předmětů, strojů a mechanismů. Stejně tak pomocí ní lze objekty upravovat, nebo poškozovat (sofistikovaněji než \textit{Kondicí}...). Sjednocuje tak dovednosti z pravidel M16 jako \textit{Kamenictví, Kovářství, Truhlářství} atd.

\subsection*{Překonání}
\label{subsec:remesla-prekonani}
\prekonani

\texttt{Překonání} pomocí \textit{Řemesel} nejčastěji nachází využití při výzvách jakožto součást nějaké komplexnější akce - v situaci, kdy jde o to zabezpečit dům plný vesničanů před myšlenkovými bytostmi zvenku, může být jeden z hodů právě na \textit{Řemesla} a představovat tak vyztužení dveří. Stejně tak se může hod na \textit{Řemesla} hodit, když je potřeba narychlo opravit vůz, než vás dohoní pronásledovatelé. Samosebou, k tomu všemu je potřeba mít vhodné nástroje a prostředí - nelze vyrobit automatickou kuši uprostřed pouště, pokud máte jen lahev s vodou.

V případech, kdy o moc nejde, nebo \textit{není neúspěch zajímavý} není zábavné házet na \textit{Řemesla} a prostě nechat postavu, aby se jí t opodařilo.

\subsection*{Vytvoření výhody}
\label{subsec:remesla-vytvoreni}
\vytvoreni

Nejčastější způsob využítí této dovednosti je právě \texttt{Vytvoření výhody}. Díky ní lze na předměty umisťovat aspekty jako \asp{Vydrží nesmysl}, \asp{Robustní konstrukce}. Též lze konstruovat zcela nové předměty, které vám mohou být nápomocny - při pokusu přelézt stěnu se může hodit \asp{Narychlo sestrojená kladka}. Stejně dobře lze jiné předměty pomocí \textit{Řemesel} poškozovat - hodit do mechanismu samostřílu kus haraburdí a způsobit mu tak \asp{Zaseknutí palného mechanismu}.

Podobně jako v případě vytváření výhod \textit{Pozorností}, i zde je namístě myslet na vypravěčskou pravomoc postav. Například, prozkoumáním nějaké stroje může postava hodem na \textit{Řemesla} odhalit \asp{Skrytou výrobní vadu} či třeba odhalit \asp{Slabinu v konstrukci}.


\subsection*{Útok}
\label{subsec:remesla-utok}
\utok

Dovednost \textit{Řemesla} se běžně nepoužívá k útoků. Častěji pomocí ní nějaký předmět postava vyrobí (meč, kuši) a pak využívá např. bojovou dovednost k akci \texttt{Útok} pomocí tohoto předmětu.

\subsection*{Obrana}
\label{subsec:remesla-obrana}
\obrana

Podobně jako u akce \texttt{Útok}.

\section{Skrývání}
\label{sec:skryvani}

Dovednost \textit{Skrývání} je schopnosti být neviděn, splynout s prostředím, a to ať už při snaze být nehybně schován nebo se někam proplížit. Stejně tak dobrý hod na \textit{Skrývání} může způsobit, že \textit{prostě zmizíte} z dohledu.

\subsection*{Překonání}
\label{subsec:skryvani-prekonani}
\prekonani

Využití \textit{Skrývání} na \texttt{Překonání} je bohaté. Dá se použít na proplížení se do \asp{Těžce hlídaného} místa, proklouznout mezi strážemi, nezanechání důkazu na místě činu, kde \asp{K něčemu něpeknému došlo} a další. Stejně tak touto dovedností lze mařit pokusy o vypátraní nepřáteli, kteří mi \asp{Jsou na stopě}. \footnote{Zde je maření nemyšleno jako aktivní obrana (k tomu by sloužila akce \texttt{Obrana}), ale k odstranění již existujícího aspektu, jako je třeba ono \asp{Jsou mi na stopě}.}

\subsection*{Vytvoření výhody}
\label{subsec:skryvani-vytvoreni}
\vytvoreni

Být schovaný je výhoda. \textit{Skrýváním} si lze vytvářet aspekty jako \asp{Dobře schovaný}, \asp{Těžko spatřitelný} a třeba se tak ocitnout \asp{V zádech} nic netušícího nepřítele. 

\subsection*{Útok}
\label{subsec:skryvani-utok}
\utok

Bez použití triku se \textit{Skrývání} nedá použít k akci \texttt{Útok}.

\subsection*{Obrana}
\label{subsec:skryvani-obrana}


Nejčastěji po dovednosti \textit{Skrývání} sáhnete při pokusu o obraně proti akcím vedených dovednostmi \textit{Pozornost, Vyšetřování} - bránění se odhalení, když se schováváte, zametání stop, pokusy sejít z očí.

\section{Učenost}
\label{sec:ucenost}

\subsection*{Překonání}
\label{subsec:ucenost-prekonani}
\prekonani

\subsection*{Vytvoření výhody}
\label{subsec:ucenost-vytvoreni}
\vytvoreni

\subsection*{Útok}
\label{subsec:ucenost-utok}
\utok

\subsection*{Obrana}
\label{subsec:ucenost-obrana}
\obrana

\section{Vůle}
\label{sec:vule}

\subsection*{Překonání}
\label{subsec:vule-prekonani}
\prekonani

\subsection*{Vytvoření výhody}
\label{subsec:vule-vytvoreni}
\vytvoreni

\subsection*{Útok}
\label{subsec:vule-utok}
\utok

\subsection*{Obrana}
\label{subsec:vule-obrana}
\obrana

\section{Vyšetřování}
\label{sec:vysetrovani}

\subsection*{Překonání}
\label{subsec:vysetrovani-prekonani}
\prekonani

\subsection*{Vytvoření výhody}
\label{subsec:vysetrovani-vytvoreni}
\vytvoreni

\subsection*{Útok}
\label{subsec:vysetrovani-utok}
\utok

\subsection*{Obrana}
\label{subsec:vysetrovani-obrana}
\obrana

\section{Vztahy}
\label{sec:vztahy}

\subsection*{Překonání}
\label{subsec:vztahy-prekonani}
\prekonani

\subsection*{Vytvoření výhody}
\label{subsec:vztahy-vytvoreni}
\vytvoreni

\subsection*{Útok}
\label{subsec:vztahy-utok}
\utok

\subsection*{Obrana}
\label{subsec:vztahy-obrana}
\obrana

\section{Zlodějina}
\label{sec:zlodejina}

\subsection*{Překonání}
\label{subsec:zlodejina-prekonani}
\prekonani

\subsection*{Vytvoření výhody}
\label{subsec:zlodejina-vytvoreni}
\vytvoreni

\subsection*{Útok}
\label{subsec:zlodejina-utok}
\utok

\subsection*{Obrana}
\label{subsec:zlodejina-obrana}
\obrana

\end{document}


%%% Local Variables:
%%% mode: LuaTeX
%%% TeX-master: "../main"
%%% End:


