
\chapter{``Povolání''}
\label{chap:speciality}

\section{Povolání ve Fate Core}
\label{sec:povolani-fate-core}

\section{Speciální dovednosti}
\label{sec:specialni-dovednosti}

\section{Speciální triky}
\label{sec:specialni-triky}


\textbf{NÁPADY}
Berserker buď jako viking, nebo jako zoufalec: tělo berserkera reaguje extrémním způsobem na určité formy stresu, jeho instinkt přežití je boj a při něm sází na vše. Kumulací nezvládnutých emocí, provokací apod. se mysl přesvědčí o neexistenci jiného východiska než boje - zapne se berseker (cítí se zahnaná do kouta a to je obranný mechanismus).
Toto úplně nesedí na vikingského bojového šílence - ale toto není generické povolání, toto je hack; na druhou stranu, vizáž nepředvídatelného a výbušného jedince by měl mít asi jakýkoliv berserker.

\begin{itemize}
\item  nepotřebuji speciální dovednosti
\item  chci, aby berserker byl stav, do kterého je nutné se dostat; v něm lze používat některé speciální triky
\item  použít co nejvíc již existující mechaniky - stresy, následky, dovednosti
\item  postava je hodně náchylná na provokaci - hází si hráč sám nebo PJ například i za prostředí či i v případech, kdy se nejednalo vyloženě o duševní útok
\item  berserker je zapnutý po určitém množství duševního stresu - berserker je frustrovaný, zahnaný do kouta
\item  duševní stresy místo bodů zuřivosti? resp. místo bodů zuřivosti vyvolání aspektu a tolik volných vyvolání kolik předchozích bodů zuřivosti (nechceme žádnou spešl mechaniku)
\end{itemize}

\newpage
\begin{tcolorbox}[%
    enhanced, 
    breakable,
    skin first=enhanced,
    skin middle=enhanced,
    skin last=enhanced,
    ]{}
\section{Berserker}
\label{sec:berserker}
\textbf{Svolení}: Vhodný aspekt na kartě postavy (\asp{Nesmrtelný berseker, Nezabitelný vraždící magor}.)\\
\textbf{Cena}: Bod obnovy za každý speciální trik (resp. stejně jako u běžných triků).\\
\textbf{Popis}: Specialita berserker upravuje:
\begin{itemize}
    \item Hody na dovednost \dov{Provokace}
    \item Hody na dovednost \dov{Vůle}
    \item Manipulaci s duševními stresy
    \item Za určitých podmínek může postavu popadnout stav bojového šílenství, ve kterém platí zvláštní pravidla a berserker smí používat speciální triky
    \begin{itemize}
    \item \trk{Trik 1}
    \item \trk{Trik 2}
    \end{itemize}
\end{itemize}

\subsection*{Hody na dovednost \dov{Provokace}}
Mysl berserkera je křehká. Stává se proto, že hody na lecjakou dovednost v lecjaké peripetii mohou být interpretovány jako hody na \dov{Provokaci} a akci \akc{Útok} (jakoby v rámci \per{Duševního konfliktu}). Útočníkem přitom nemusí být postavy, ale i například prostředí či situace. \\
Je primárně na hráči, aby ze znalosti své postavy usoudil, kdy si říct o hody na \dov{Provokaci} proti němu; Vypravěč samozřejmě může hody iniciovat kdykoliv to považuje za vhodné. Jestliže postava používala jinou dovednost, usoudí Vypravěč, s jakým bonusem si hází na \dov{Provokaci}; k tomu taktéž stanoví, jakou formu opozice tvoří berserkerova postava.\\

\underline{Příklady:}
\begin{itemize}
\item pobyt v nehostinném prostředí (poušť, džungle, močál, tundra, voda)
\item nenaplnění fyziologických potřeb (jídlo, pití, spánek, teplo)
\item nečekaná změna (pád, zmizení/objevení něčeho)
\item krádež
\item zrada spojence
\end{itemize}

Speciálně fungují hody na \dov{Provokaci} v rámci \akc{Konfliktů}; v nich postavy jednají s jediným zájmem ubližovat a jsou tedy velikým zdrojem úzkosti. Na konci každého kola, ve kterém se berserker konfliktu přímo účastní, či v něm není od epicentra dění dostatečně vzdálen, \textit{konflikt sám} na berserkera \dov{Provokací} zaútočí. Bonus k tomuto hodu je pak dán různými faktory: co se v konfliktu v daném kole stalo (někdo např. zasáhl berserkerova kamaráda), o kolik v konfliktu jde, kdo se jej účastní apod. V základu by neměl být bonus příliš vysoký (+0).
\subsection*{Hody na dovednost \dov{Vůle}}
Ve všech případech, kdy berserker aktivně oponuje \akc{Útoku} \dov{Provokací} \footnote{Striktně vzato tímto speciálním způsobem může použít \dov{Vůli} i v jiných případech, než jen na \akc{Obranu} proti \dov{Provokaci}. Nicméně, pokud nebude výsledkem duševní zranění, není toto použití bersekerovi k ničemu.} (ve smyslu předchozí odrážky) si samozřejmě hází na \akc{Obranu} svou \dov{Vůlí}. Narozdíl od běžných postav smí ale berserker vyvolávat aspekty ve \textit{svůj neprospěch} (a to ať už aspekty vlastní či jiné) a ``přidat'' si tak postih \textit{-2}. \footnote{Případně použít vyvolání jiným způsobem jako kdykoliv jindy.}\\
Proč by mohl berserker chtít co \textit{nejnižší} hod na \dov{Vůli} se dozvíme níže.

\subsection*{Manipulace s duševními stresy}
Duševní stresy hrají v životě berserkera důležitou roli. Tak za prvé, nelze se jich snadno zbavit: běžná postava se stresu zbavuje ihned po konfliktu (resp. po dostatečném odpočinku), ale to pro berserkera neplatí. \\
Aby se berserker zbavil duševního stresu, postupuje stejně jako v případě zotavování se z následků (podrobně viz \ref{sec:zotavovanise}), přičemž Drobný (+2) následek se ztotožňuje se stresem +1, Mírný (+2) se stresem +2 a Vážný (+6) se stresem +3. Navíc, místo hodů na \dov{Empatii} jiné postavy může berserker házet na svou vlastní \dov{Vůli}, aby započal ozdravnou fázi.

\subsection*{Stav bojového šílenství}
Co dělá berserkera bersekerem je jeho \textbf{bojové šílenství}. Je to speciální stav, ve kterém platí speciální pravidla, která odrážejí děsivou skutečnost - z myslící a cítící bytosti se po hrůzostrašném přepnutí stane běsnící bestie bez úvahy a smyslů zbavená. \\
Ve stavu \textbf{bojového šílenství} berserkera ovládne jeho extrémně silná strategie pro přežití - boj. Obecně lze říci, že berserker během \textbf{bojového šílenství}:

\begin{itemize}
\item útočí na všechno, co ho ohrožuje, dokud ho to nepřestane ohrožovat
\item nepřemýšlí a nekoná racionálně - neposlouchá rozkazy, netaktizuje
\end{itemize}
V typickém případě, kdy se berserker rozzuří na základě konfliktu s nepřátelskými postavami, je berserkerova příslušnost poměrně jasná - útočí na nepřátele. Nicméně, může se stát, že berserker je do stavu \textbf{bojového šílenství} dohnán jinými okolnostmi, například neustálým vystavením nehostinným podmínkám (aldeny v džungli); pak neexistuje hmatatelný původce jeho frustrace a berserker své emoce dá najevo nekontrolovatelným a neusměrnitelným způsobem - a je fuk, že celou dobu putoval v rámci družiny...\\
Berserker se do stavu \textbf{bojového šílenství} může dostat několika způsoby
\begin{itemize}
\item jakmile poprvé obdrží následek způsobený duševním zraněním (tj. buď nemá již dostatek stresu či zasažen za mnoho posunů)
\item jakmile proti němu někdo uspěje se stylem na \akc{Útok} pomocí \dov{Provokace}, načež neuspěje na následný hod na \dov{Vůli} s pasivní opozicí +0
\item zaplacením bodu osudu
\end{itemize}

Jakmile berserker propadne \textbf{bojovému šílenství}, získává tolik volných vyvolání svého berserkerovského aspektu, kolik má utržených políček duševního stresu, a k tomu může používat speciální triky uvedené níže.

\subsection*{Speciální triky}
\begin{itemize}
\item \trk{Healing}
\item \trk{Velká rána}
\item \trk{Výhody z následků}
\item \trk{Přidám si následek}
\item \trk{Poseru vás}
\end{itemize}
\end{tcolorbox}




%%% Local Variables:
%%% mode: LaTeX
%%% TeX-master: "../main"
%%% End:
