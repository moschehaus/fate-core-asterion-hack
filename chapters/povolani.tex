% !TEX root = main.tex


\documentclass[../main.tex]{subfiles}
\begin{document}

\chapter{``Povolání''}
\label{chap:speciality}

\section{Povolání ve Fate Core}
\label{sec:povolani-fate-core}

\section{Povolání bez svolení}
\label{sec:pov-bez-svoleni}

\subsection{Bojovník}
\label{sec:pov-bojovnik}

\subsection{Střelec}
\label{sec:pov-strelec}

\subsection{Zloděj}
\label{sec:pov-zlodej}

\subsection{Hraničář}
\label{sec:pov-hranicar}

\subsection{Sicco}
\label{sec:pov-sicco}


\section{Povolání se svolením}
\label{sec:pov-se-svolenim}

\begin{tcolorbox}
\subsection{Čaroděj}
\label{sec:pov-carodej}
\textbf{Svolení}: Vhodný aspekt na kartě postavy (\asp{Magické nadání}, \asp{Bílý kouzelník}, \asp{Čarodějný akolyta}, \asp{Učeň Alwarina Bílého})\\
\textbf{Cena}: Jeden bod obnovy za každý speciální trik (jako běžné triky). První styl magie získává čaroděj zdarma, všechny další pak jako běžný trik za bod obnovy.\\
\textbf{Popis}: Čarodějník je schopen používat magii k různým účelům. Získává možnost naučit se některé speciální triky:
\begin{itemize}
\item \trk{Styly magie}: sesílání kouzel určitého stylu (oboru, zaměření) magie
\item \trk{Magický pomocník}: přivolání magického pomocníka ze Stínového světa
\item \trk{Mentální souboj}: ovládnutí techniky mentálního souboje, vhodné zejména k zneškodnění magicky aktivních protivníků
\end{itemize}
Kromě výběru triků čaroděj ke svému řemeslu potřebuje tzv. \dov{Sesílací dovednost}, na kterou si hází vždy, když neexistuje žádná vhodnější (podrobněji viz dále). \dov{Sesílací dovednost} je nějaká z existujících dovedností, například \dov{Empatie, Učenost, Vůle} apod. 
\end{tcolorbox}

\subsubsection*{Styly magie}
\label{stylymagie}

Jak bylo popsáno v \Cref{sec:magie}, magie se v této sadě pravidel značně liší od nastavení M16. Hlavní motivací je jednoduchost a vyváženost: nemít griomáry ušetří jejich sepisování, vyhýbáme se složitým ad hoc mechanikám, není třeba promýšlet férovost každého kouzla zvlášť. K tomu podněcuje v hráčích kreativitu a činí magii mnohem univerzálnější. Na druhou stranu, takto interpretovaná magie se může jevit jako slabší (např. neexistují kouzla jako Magomegamix), méně blyštivá či epická a taky nemusí hráčům vždy vyhovovat, že si všechna kouzla vymýšlejí sami. Tyto neduhy částečně řeší styly magie. Než ale zmíníme je, uveďme, jak vypadá mechanika magie z nadhledu.\\

\todo{Nasledujici}
\subsubsection*{Magický pomocník}
\label{sec:magickypomocnik}

\subsubsection*{Mentální souboj}
\label{sec:mentalnisouboj}



\begin{tcolorbox}
\subsection{Alchymista}
\label{sec:pov-alchymista}
\textbf{Svolení}:\\
\textbf{Cena}:\\
\textbf{Popis}:\\

\end{tcolorbox}

\begin{tcolorbox}
\subsection{Theurg}
\label{sec:pov-theurg}
\textbf{Svolení}:\\
\textbf{Cena}:\\
\textbf{Popis}:\\

\end{tcolorbox}

\begin{tcolorbox}  
\subsection{Kněz}
\label{sec:pov-knez}
\textbf{Svolení}: Vhodný aspekt na kartě postavy (\asp{Rytíř Gorův}, \asp{Mnich Paní země}, \asp{Aurion, můj dokonalý pán}) + vhodný roleplay \\
\textbf{Cena}: Bod obnovy za každý speciální trik (resp. stejně jako u bězných triků) \\
\textbf{Popis}: Kněz má přístup k několika speciálním trikům \\
\begin{itemize}
    \item \trk{Zázraky}: Knez umi silu sveho boha pouzit k sesilani zazraku. Ty se deli na \trk{Modlitby} a \trk{Ritualy}, pricemz jeden z techto dvou druhu ziskava knez zadarmo, za zbyly plati bodem obnovy (resp. jako u beznych triku).
\item \trk{Božský pomocník}: Knez ziskava moznost privolat si ze Stinoveho sveta bozskeho pomocnika.
\item \trk{Mentální souboj}
\end{itemize}
\end{tcolorbox}



\subsubsection*{Modlitby}
\label{sec:modlitby}
Modlitby jsou \trk{Zazraky} cilene na kneze samotneho. Muze se jednat o ruzna posileni, uzdraveni, pridani specialniho efektu a jine. 

\subsubsection*{Obřady}
\label{sec:obrady}
Obrady jsou druhym zpusobem, jak knez muze pouzivat silu sveho boha. Narozdil od Modliteb jsou tercem Obradu skupiny.\\

Budou existovat situace, kdy nepujde zcela dobre rozlisit, kdo je tercem Zazraku, tj. zda-li se jedna o Modlitbu ci Obrad. \sidenote{Knez chce pomoci zazraku seslaneho na skupinu zjistit, kdo z nich ma hledany predmet; vysledkem je sice knezova vedomost, ale na druhou stranu se zazrak jiste tyka cele skupiny.} V takovem pripade je ale vlastne jedno, o jaky Zazrak se jedna (pokud ma knez naucene oba triky).

\subsubsection*{Božský pomocník}
\label{sec:bozsky-pomocnik}

\subsubsection{Alcaril}
\label{sec:alcaril}

\subsubsection{Aurion}
\label{sec:aurion}

\subsubsection{Bongir}
\label{sec:bongir}

\subsubsection{Dunril}
\label{sec:dunril}

\subsubsection{Estel}
\label{sec:estel}

\subsubsection{Faeron}
\label{sec:faeron}

\subsubsection{Finwalur}
\label{sec:finwalur}

\textbf{Modlitby}

\begin{itemize}

\item Odstranění únavy I
\item Odolnost proti počasí [sám]
\item Uzdrav lehké zranění
\item Záblesk
\item Odstranění únavy II
\item Vidění v podzemí
\item Intuice I
\item Let 
\item Autorita
\item Intuice II
\item Odolnost proti počasí
\item Levitace
\item Malé ošetření
\item Neslyšitelnost
\item Dobré spaní
\item Orlí oči
\item Superrychlý pochod
\item Posílení vlastnosti
\item Průchod přirozenou hmotou
\item Poslání

\end{itemize}

\textbf{Obrady}
\begin{itemize}

    \item Rychlý pochod
    \item Sugesce I [odvaha, odhodlání]
    \item Orientace
    \item Hlídka
    \item Iluzorní převlek
    \item Blahodárný spánek
    \item Améba
\item Uzdrav těžká zranění
\item Vnukni myšlenku
\item Znecitlivění
\item Rozdělej oheň
\item Cesta ke světlu
\item Chození po hladině
\item Najdi bytost
\item Přivolej druida
\item Ochranná ruka 
\item Cestář
\item Neviditelnost
    \item Rozdělený poutník
    \item Pohyb mimo čas
\end{itemize}

\subsubsection{Gor}
\label{sec:gor}

\subsubsection{Lamius}
\label{sec:lamius}

\subsubsection{Mauril}
\label{sec:mauril}

\subsubsection{Mern}
\label{sec:mern}

\subsubsection{Mirtal}
\label{sec:mirtal}

\subsubsection{Rianna}
\label{sec:rianna}

\subsubsection{Siaron}
\label{sec:siaron}

\subsubsection{Siomén}
\label{sec:siomen}

\subsubsection{Sirril}
\label{sec:sirril}

\subsubsection{Tarfein}
\label{sec:tarfein}

\subsubsection{Dreskan}
\label{sec:dreskan}

\subsubsection{Gwi}
\label{sec:gwi}

\subsubsection{Inaka}
\label{sec:inaka}

\subsubsection{Meabor}
\label{sec:meabor}

\subsubsection{Ndangawa}
\label{sec:ndangawa}

\subsubsection{Sandol Kah}
\label{sec:sandol-kah}

\subsubsection{Sarapis}
\label{sec:sarapis}

\subsubsection{Šin}
\label{sec:sin}

\subsubsection{Temná kápě}
\label{sec:temna-kap}

\subsubsection{Ynnar Rút}
\label{sec:ynnar-rut}


\begin{tcolorbox}
  
\subsection{Druid}
\label{sec:pov-druid}
\textbf{Svolení}:\\
\textbf{Cena}:\\
\textbf{Popis}:\\

\end{tcolorbox}

\begin{tcolorbox}
\subsection{Berserker}
\label{sec:berserker}
\textbf{Svolení}: Vhodný aspekt na kartě postavy (\asp{Nesmrtelný berseker, Nezabitelný vraždící magor}.)\\
\textbf{Cena}: Bod obnovy za každý speciální trik (resp. stejně jako u běžných triků).\\
\textbf{Popis}: Specialita berserker upravuje:
\begin{itemize}
    \item Hody na dovednost \dov{Provokace}
    \item Hody na dovednost \dov{Vůle}
    \item Manipulaci s duševními stresy
    \item Za určitých podmínek může postavu popadnout stav \textbf{bojového šílenství}, ve kterém platí zvláštní pravidla a berserker smí používat speciální triky
\end{itemize}
\textbf{Seznam speciálních triků}:
\begin{itemize}
    \item \trk{Je to monstrum}
    \item \trk{Zacelení ran}
    \item \trk{Údery zlosti}
    \item \trk{Jaktože nechcípáš?!}
    \item \trk{Všechno nebo nic}
\end{itemize}
\end{tcolorbox}

\begin{tcolorbox}
\subsection{Netvorobijec}
\label{sec:pov-netvorobijec}
\textbf{Svolení}:\\
\textbf{Cena}:\\
\textbf{Popis}:\\
\end{tcolorbox}

\subsubsection*{Hody na dovednost \dov{Provokace}}
Mysl berserkera je křehká. Stává se proto, že hody na lecjakou dovednost v lecjaké peripetii mohou být interpretovány jako hody na \dov{Provokaci} a akci \akc{Útok} (jakoby v rámci \per{Duševního konfliktu}). Útočníkem přitom nemusí být postavy, ale i například prostředí či situace. \\
Je primárně na hráči, aby ze znalosti své postavy usoudil, kdy si říct o hody na \dov{Provokaci} proti němu; Vypravěč samozřejmě může hody iniciovat kdykoliv to považuje za vhodné. Jestliže postava používala jinou dovednost, usoudí Vypravěč, s jakým bonusem si hází na \dov{Provokaci}; k tomu taktéž stanoví, jakou formu opozice tvoří berserkerova postava.\\

\underline{Příklady:}
\begin{itemize}
\item pobyt v nehostinném prostředí (poušť, džungle, močál, tundra, voda)
\item nenaplnění fyziologických potřeb (jídlo, pití, spánek, teplo)
\item nečekaná změna (pád, zmizení/objevení něčeho)
\item krádež
\item zrada spojence
\end{itemize}

Speciálně fungují hody na \dov{Provokaci} v rámci \akc{Konfliktů}; v nich postavy jednají s jediným zájmem ubližovat a jsou tedy velikým zdrojem úzkosti. Na konci každého kola, ve kterém se berserker konfliktu přímo účastní, či v něm není od epicentra dění dostatečně vzdálen, \textit{konflikt sám} na berserkera \dov{Provokací} zaútočí. Bonus k tomuto hodu je pak dán různými faktory: co se v konfliktu v daném kole stalo (někdo např. zasáhl berserkerova kamaráda), o kolik v konfliktu jde, kdo se jej účastní apod. V základu by neměl být bonus příliš vysoký (+0).\\

Jakmile se berserker dostane do stavu \textbf{bojového šílenství} v rámcí konfliktu, na konci kola už na něj konflikt dále neútočí.
\subsubsection*{Hody na dovednost \dov{Vůle}}
Ve všech případech, kdy berserker aktivně oponuje \akc{Útoku} \dov{Provokací} \sidenote{Striktně vzato tímto speciálním způsobem může použít \dov{Vůli} i v jiných případech, než jen na \akc{Obranu} proti \dov{Provokaci}. Nicméně, pokud nebude výsledkem duševní zranění, není toto použití bersekerovi k ničemu.} (ve smyslu předchozí odrážky) si samozřejmě hází na \akc{Obranu} svou \dov{Vůlí}. Narozdíl od běžných postav smí ale berserker vyvolávat aspekty ve \textit{svůj neprospěch} (a to ať už aspekty vlastní či jiné) a ``přidat'' si tak postih \textit{-2}. \sidenote{Případně použít vyvolání jiným způsobem jako kdykoliv jindy.}\\
Proč by mohl berserker chtít co \textit{nejnižší} hod na \dov{Vůli} se dozvíme níže.

\subsubsection*{Manipulace s duševními stresy}
Duševní stresy hrají v životě berserkera důležitou roli. Tak za prvé, nelze se jich snadno zbavit: běžná postava se stresu zbavuje ihned po konfliktu (resp. po dostatečném odpočinku), ale to pro berserkera neplatí. \\
Aby se berserker zbavil duševního stresu, postupuje stejně jako v případě zotavování se z následků (podrobně viz \Cref{sec:zotavovanise}), přičemž Drobný (+2) následek se ztotožňuje se stresem +1, Mírný (+2) se stresem +2 a Vážný (+6) se stresem +3. Navíc, místo hodů na \dov{Empatii} jiné postavy může berserker házet na svou vlastní \dov{Vůli}, aby započal ozdravnou fázi.\\
Čím více nevyléčeného duševního zranění berserker má, tím labilnější a méně předvídatelný je - tím snáze například napadne přátelskou postavu, vrhne se na nevinné apod. (kromě toho, že je mechanicky pravděpodobnější, že se dostane do stavu \textbf{bojového šílenství}).

\subsubsection*{Stav bojového šílenství}
Co dělá berserkera bersekerem je jeho \textbf{bojové šílenství}. Je to speciální stav, ve kterém platí speciální pravidla, která odrážejí děsivou skutečnost - z myslící a cítící bytosti se po hrůzostrašném přepnutí stane běsnící bestie bez úvahy a smyslů zbavená. \\
Ve stavu \textbf{bojového šílenství} berserkera ovládne jeho extrémně silná strategie pro přežití - boj. Obecně lze říci, že berserker během \textbf{bojového šílenství}:

\begin{itemize}
\item útočí na všechno, co ho ohrožuje, dokud ho to nepřestane ohrožovat
\item nepřemýšlí a nekoná racionálně - neposlouchá rozkazy, netaktizuje
\end{itemize}
V typickém případě, kdy se berserker rozzuří na základě konfliktu s nepřátelskými postavami, je berserkerova příslušnost poměrně jasná - útočí na nepřátele. Nicméně, může se stát, že berserker je do stavu \textbf{bojového šílenství} dohnán jinými okolnostmi, například neustálým vystavením nehostinným podmínkám (aldeny v džungli); pak neexistuje hmatatelný původce jeho frustrace a berserker své emoce dá najevo nekontrolovatelným a neusměrnitelným způsobem - a je fuk, že celou dobu putoval v rámci družiny...\\
Berserker se do stavu \textbf{bojového šílenství} může dostat několika způsoby
\begin{itemize}
\item jakmile poprvé obdrží následek způsobený duševním zraněním (tj. buď nemá již dostatek stresu či zasažen za mnoho posunů)
\item jakmile proti němu někdo uspěje se stylem na \akc{Útok} pomocí \dov{Provokace}, načež neuspěje na následný hod na \dov{Vůli} s pasivní opozicí +0
\item zaplacením bodu osudu
\end{itemize}

Jakmile berserker propadne \textbf{bojovému šílenství}, získává tolik volných vyvolání svého berserkerovského aspektu, kolik má utržených políček duševního stresu či neléčených následků způsobených duševními útoky (tzv. ``duševní následek''), a k tomu může používat speciální triky uvedené níže. Navíc, za každý utržený \textit{duševní} stres během \textbf{bojového šílenství} dostává další volné vyvolání svého aspektu. Po skončení \textbf{bojového šílenství} se berserkerovi obnoví veškerý duševní stres - veškeré svou frustraci zanechá v boji.

\subsubsection*{Speciální triky}
\begin{itemize}
\item \trk{Je to monstrum}: Berserk svým zjevem a dojemnou nezranitelností a nepřemožitelností budí v ostatních strach. Může použít svoji vitalitu - dovednost \dov{Kondice} na \akc{Vytvoření výhody} pomocí zastrašení nepřátel (místo dovednosti \dov{Provokace}). K tomu ale musí být uvěřitelné, že z berserkera budou mít strach (neznají jeho schopnosti, jsou slabší apod.)
\item \trk{Zacelení ran}: Jednou za \per{konflikt} smí berserker vyvoláním svého berserkerovského aspektu snížit závažnost všech svých fyzických stresů a veškerých následků kromě toho extrémního (+8) o jednu úroveň (a nejnižší stres či následek případně odstranit).
\item \trk{Údery zlosti}: Když berserker používá svůj \dov{Boj beze zbraně} proti protivníkovi s nejvýše kroužkovou zbrojí (hodnocením +2), jeho údery hnané nelidským hněvem dopadají jako palcáty - s hodnocením zbraně +2. 
\item \trk{Jaktože nechcípáš?!}: Vždy, když berserker remizuje při akci \akc{Útok}, jeho tělo se začne domnívat, že protivníky nelikviduje (= z nebezpečí se nedostává) dostatčně rychle; může se rozhodnout, že místo posílení uštědří nepříteli zásah v hodnotě +2 posunů, ale v důsledku enormního vypětí sil za to musí přijmout jeden fyzický stres (libovolný).
\item \trk{Všechno nebo nic}: Berserkerovo tělo ve stavu \textbf{bojového šílenství}
 dělá vše, aby se vyhnulo ohrožení života. Jestliže se mu to ale ani tak nedaří, nasadí efektivní mechanismy zvládání fyzických zranění, které se ale neobejdou bez extrémního psychického zatížení: vždy, když berserker obdrží fyzický stres (nikoli následek), může se rozhodnout pokrýt posuny odpovídající hodnotě tohoto stresového políčka pomocí své duševní výdrže: pomocí duševního stresu či následku \sidenote{Tedy například zaškrtnutí fyzického stresu +3 může kompenzovat zaškrtnutím duševního stresu +1 a ``Drobného duševního následku''}; jedná-li se o duševní stres, získává pak okamžitě volné vyvolání jeho berserkerovského aspektu. Jestliže přijmul ``duševní'' následek výměnou za fyzický stres (následky se sice nerozlišují na fyzické a duševní, ale jsou to aspekty, tedy jsou mj. pojmenovány a dá se rozlišit, zda-li jsou fyzického či duševního původu), získává na něj zdarma vyvolání \textit{sám berserker a nikoli protivník}. Problémem pak je, že duševní zranění berserkerovi jen tak nezmizí a může se stát, že po bitce se stane lidsky zcela nepoužitelným... \sidenote{Všimněme si, že bez tohoto triku získává berserker volná vyvolání jen za obdržené duševní stresy (nikoli následky). Ovšem s \trk{Všechno nebo nic} může berserker z fyzického stresu dělat duševní stres, tj. efektivně získávat volná vyvolání z fyzického stresu, či dokonce ``duševní následek'' a na něj získat volná vyvolání, tj. efektivně získávat výhody i z získaného následku.}


\begin{tcolorbox}
\subsection{Subotamský mnich}
\label{sec:pov-subotam}
\textbf{Svolení}:\\
\textbf{Cena}:\\
\textbf{Popis}:\\
\end{tcolorbox}

\begin{tcolorbox}
\subsection{Edenův bratr}
\label{sec:pov-edenak}
\textbf{Svolení}:\\
\textbf{Cena}:\\
\textbf{Popis}:\\
\end{tcolorbox}

\end{itemize}



\end{document}


%%% Local Variables:
%%% mode: LaTeX
%%% TeX-master: "../main"
%%% End:
