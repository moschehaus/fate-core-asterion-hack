% !TEX root = main.tex

\documentclass[../main.tex]{subfiles}
\begin{document}

\section{Netvorobijec}
\label{sec:netvorobijec}

PŘEHLED
Počet dovednostních bodů k zakoupení všeho u jednotlivých povolání (spec dovednosti se počítají na +1, případně na nejnižší nutné úrovni)
Lomyš: 31 (finty zdarma?)
Válečník (průměrný): 20 (finty zdarma)
Psionik těla: 16
Chodec: 30 (se dvěma čarovnými silami)
Zloděj: 33 (má hodně dovedností)
Edenův bratr: 24 


OSOBNÍ DENÍK: Ylyndar

ASPEKTY

Stínač, Odmyslitel, Zaříkávač, Lomyš, Lopříš
Lovec nestvůr, co už něco zažil
Lovec nestvůr, pár století starý - je to profesionál, nejspíš mistr, zná všelijaký fígle
Stihl taky naštvat pár lidí i nelidí
Bez emocí
Kvůli symbiontům i kvůli ostatním, zaklínač (částečně) předstírá, že nic necítí - slitování, zármutek, výčitky atd. Na kolik je to pravda, neví možná ani on sám. Při vyplácení odměny za nestvůru narazil v jedné vísce na tvrdohlavého pajzla Sitaga, se kterým se dostal do křížku kvůli nevybíravému jednání ohledně své výplaty.
Horský elf z okraje společnosti
Narodil se v horách, měl to prostě těžký - ale zase je jako Xena - zocelený, jen ne v žáru bitev, ale životem samým. Nesnáší světlé elfy a oni jeho, lidé se mu také vyhýbají.
Z Rytířů Devíti hvězd
Vycvičen Rytíři Devíti hvězd - naučil se hlavně dobře bojovat, ale snažili se mu vštípit i jakýsi kodex cti. To sice nemělo valný účinek, občas si ale vzpomene, a to pak zabolí.
Nepodporuji, ale respektuji
před pár lety jsem lidi bral jako pouhé věci avšak můj názor dokázal zménit Edward který předvedl respektuhodný kousek. Když byl schopen zranit Fexta a to mě přesvědčilo o tom že někteří lidé nejsou jen chodící kupa hnoje.


DOVEDNOSTI (24 pyramida)
válečník lomyš+4
mrštnost+3
atletika+1
odolnost+2
pozornost+2
gramotnost
vůle
vyšetřování+1
bylinkářství
SPECIALITY
rychlé léčení (II)
pevné zdraví (I)
hovnožrout (I)
zrychlení (II)
přelévání života (I)
posílení smyslů (I)
vidění ve tmě (I)
rychlé tasení


PRAVIDLA

\subsection{Gammathorax}
\label{sec:lopr-gamma}

Stínač smí použít speciální vlastnosti symbionta v těle tolikrát, kolik má aspektů plus kolik specialit do tohoto stromu. Výjimkou jsou dovednosti Rychlé léčení a Hovnožrout - ty fungují vždy.\\


Rychlé léčení (II)\\
Vyžadovaný aspekt: I\\
Prerekvizita: Průměrná Kondice\\
Trvání: Vždy\\
Popis: Odmyslitelův symbiont se neustále stará o svého hostitele a zaceluje jeho ranky (doufejme) dříve, než mu stíhají přibývat. Proto nepotřebuje úspěšný hod na Medicínu, aby započal proces léčení. Navíc, z Mírných (+4) následků se netvorobijec zotaví po vydatnějším odpočinku (namísto po celém sezení)\\


Muž  (nebo žena) z oceli (I)\\
Vyžadovaný aspekt: I\\
Prerekvizita: Slušná Kondice, Průměrná Vůle\\
Trvání: Scéna\\
Popis: Symbiont těsně před dopadajícím úderem nelidsky pevně zatne Lopříšovy svaly, čímž zvýší jeho hodnocení zbroje o 1. To se ale  bohužel nedá dále zvyšovat.\\


Hovnožrout (chálec jak démon) (I)\\
Vyžadovaný aspekt: II\\
Prerekvizita: Průměrná Kondice\\
Trvání: ano\\
Popis: Příživník v těle upravuje zaklínačův trávicí trakt tak, aby byl schopen jíst i jinak nepoživatelná jídla - syrové maso, jedovaté Byliny, plesnivá jídla. Zároveň zaklínač získává bonus +1 k hodu při porovnání kondice proti jedům.\\


Vytěsnění z magie (II)\\
Vyžadovaný aspekt: II\\
Prerekvizita: Slušná Vůle, Slušná Učenost, Průměrné Bylinkářství\\
Vyvolání: půlden (např. celá noc)\\
Trvání: dokud se psionik znovu nenají normálního jídla a neodpočine si\\
Požadavky: magické suroviny v hodnotě 50 zl, každé další jídlo 5 zl, viz níže.\\
Popis: Lomyš kombinuje hladovění, jedení speciálních pokrmů, lázně v magických substancích a hlubokou meditaci. Není to nic příjemného, ale účinky jsou výrazné. Dokáže ze svého těla vytěsnit magickou energii, tím pádem šance na seslání veškerých kouzel na psionika (i těch prospěšných) je o 1 nižší. Bohužel (?) to ovlivňuje i lektvary coby magické substance – je šance 1:6, že lektvar po vypití nebude účinkovat. Zaklínač může v proceduře pokračovat za cenu poškození vlastního těla radikálnější kůrou. Za každý život, který procedurou ztratí, bude postih ke kouzlení na psionika vyšší o 1. Tím pádem i šance na účinkování lektvarů bude o 1:6 nižší za každý ztracený život. Za každý půlden je možné snížit úspěšnost na seslání o 2. maximální postih, který bude mít magie na psionika nemůže být vyšší než stupeň psionikova aspektu. Účinek bude trvat tak dlouho, jak dlouho psionik chce. Celou dobu bude mít ovšem stržené životy (všechny zakřížkované) a musí celou dobu jíst speciální jídlo (5 zl za jeden chod – snídani/oběd/večeři). Poté co se nají obyčejného jídla a důkladně si odpočine (spánek, půlden odpočinku atp.).\\


Zrychlení (III)\\
Vyžadovaný aspekt: III\\
Trvání: scéna\\
Popis: 	Symbiont vyplaví enormní dávky nadledvinových hormonů, čímž dočasně urychlí Zaklínačovo tělo. Ten pak získává aspekt Zrychlený, díky kterému má vždy:\\
bonus +2 při porovnávání Pozornosti pro potřeby určení iniciativy v konfliktu\\
na začátku každého kola se přemístit na libovolnou sousední zónu, nebrání-li mu v tom aspekt prostředí\\
za zaplacení bodu osudu smí provést dvě akce v jednom kole, jako kdyby byl dvěma hráči zároveň, tedy, např. v případě, že se proti němu protivník se stylem ubrání, získává proti Stínači posílení\\


Přelévání života (I)\\
Vyžadovaný aspekt: IV\\
Prerekvizita: Průměrná Kondice, Průměrná Vůle\\
Trvání: scéna\\
Popis: Zaklínačův symbiont perfektně ovládá jeho metabolismus. Dokáže tedy nevídané - zvyšovat jednorázovou sílu na úkor výdrže a naopak. Lomyš smí zvýšit jednu ze svých dovedností (Mobilita, Kondice) na úkor snížení druhé, ale to vždy jen o jedna. Každé kolo tuto volbu může změnit\\


\subsection{Deltacranium}
\label{sec:lopr-delta}

Stínač smí použít speciální vlastnosti symbionta v hlavě tolikrát, kolik má aspektů plus investovaných dovednostních bodů do tohoto stromu. Výjimkou je dovednost Vycítění myšlenkových bytostí. 

Vycítění myšlenkových bytostí (I)\\
Vyžadovaný aspekt: I\\
Prerekvizita: Slušná Pozornost\\
Trvání: ano\\
Popis: Pán hry si při přítomnosti myšlenkové bytosti hodí. Podle výsledky pak může Stínači prozradit, zda-li se v jeho blízkosti vyskytují nějaké myšlenkové bytosti.\\


Posílení smyslů (I)\\
Vyžadovaný aspekt: I\\
Prerekvizita: Průměrná Pozornost nebo  Průměrné Vyšetřování\\
Trvání: scéna\\
Popis: Deltacranium vylepšuje Odmyslitelovy smysly. Ten pak perfektně rozezná jednotlivé provensálské bylinky a černý pepř od kamptoského, stejně tak jako holuba hřivnáče od hrdličky zahradní. V případě potřeby získává bonus +1 k Pozornosti nebo Vyšetřování.\\


Vidění ve tmě (I)\\
Vyžadovaný aspekt: II\\
Prerekvizita: Posílení smyslů\\
Trvání: scéna\\
Popis: Odmyslitelův symbiont vybudí čípky a tyčinky v oku, čímž poskytne Lomyšovi jakousi verzi vidění ve tmě. To ale nelze použít v dokonalé tmě (stejně jako všechny jiné). V případě nutnosti porovnání hodů neguje postih způsobený tmou, tedy Lomyš běžně používá dovednosti Pozornost a Vyšetřování\\


Viděni neviditelnosti (I)\\
Vyžadovaný aspekt: III\\
Prerekvizita:  Posílení smyslů\\
Trvání: scéna\\
Popis: Magická podstata Deltacrania odhaluje i magicky skryté bytosti a předměty. Odmyslitel získává bonus +3 k Pozornosti či Vyšetřování proti neviditelnosti, jejímž základem je magie.\\


Potlačení bolesti (II)\\
Vyžadovaný aspekt: III\\
Prerekvizita:  Slušná Vůle nebo Slušná Kondice\\
Trvání: 6 kol\\
Popis: Dočasným odpojením nervových synapsích Zaklínač doslova necítí bolest. Ignoruje maximálně tolik typů zranění, kolik říká tabulka:\\


Míra úspěchu (hod na Odolnost, nebo Vůli)\\
Počet zranění\\
0 - 1\\
1\\
2 -3\\
2\\
4+\\
3\\

Trans (II)\\
Vyžadovaný aspekt: III\\
Prerekvizita:  Slušná Vůle nebo Slušná Kondice\\
Trvání: Scéna; předtím rituál\\
Popis: Stínač se před očekaváným střetem pohrouží do hluboké meditace kombinovanou s evokacemi a konzumací Bylin - dostane se do Transu.  Deltacranium během něj stimuluje jeho nervové synapse a umožňuje Lopříšovi přežít nemyslitelné - získává na souboj jedno políčko stresu o hodnocení 2 navíc. Na druhou stranu, Lopříš jest profesionál - a obzvlášť, je-li v Transu, může ignorovat projevy své lidskosti o to více. \\


Sharingan (II)\\
Vyžadovaný aspekt: V\\
Prerekvizita: Dobrá Pozornost\\
Trvání: Scéna\\
Popis: Deltacranium obtiskne do Lopříšova mozku pozorované pohyby a ten tak získává možnost zkopírovat libovolnou fintu, kterou uvidí; tu pak smí sam použít. Může si jí taktéž držet v paměti, v takovém případě ale nesmí využít jiné schopnosti Deltacrania a je obtěžkán aspektem Pekelně se soustředí, dokud ji nepoužije. Pokud je proti němu finta použita přímo v boji, může se pokusit ji ihned zreplikovat - pak si hází na Pozornost proti úrovni aspektu finty.\\


\subsection{Manus magikus II/II/II}
\label{sec:lopr-manus}

Manus Magikus umožňuje Lomyšovi sesílat jednoduchá Znamení. K tomuto je ale třeba mít magické runy s nimi související zhotovené fyzicky - v nejhorším na pergamenu, ideálně na plíšku, kamínku, nebo pálené hlíně. Mezi pouštními elfy jsou pro tento účel oblíbená tetování. Manus vstřebovanou magickou energii dokáže usměrnit pomocí obtisklé runy, čímž se sešle zaklínadlo. Před vyčerpáním (za scénu) je Lomyš schopen seslat maximálně tolik zakletí, kolik je jeho aspekt + počet naučených Dvojic.\\
Znamení se Lomyš učí vždy po dvou. Nákup každé dvojice stojí dva dovednostní body. Při naučení každé dvojice se zároveň Lomyšovi zvyšuje dovednost, kterou používá při sesílání. Maximálně tedy smí ovládat 6 Znamení s Výbornou (+3) dovedností. Doba vyvolání závisí na režimu seslání - v základu je určena číslem před lomítkem; alternativní seslání se řídí číslem za lomítkem a stojí dvě many místo jedné.


Aard\\
Dosah: 3 sáhy (čím blíž, tím silnější, lineární závislost)\\
Vyvolání: 0/\\
Popis: \\
Základní režim: Jednoduchý výboj vzduchu (nebo vody), který je vrhnut určitým směrem. Lze ho využít k bezpečnému otevírání dveří, podrážení nohou, víření prachu atp. Jako přímý fyzický útok se nevyužívá, není tak silný. Lze ho použít v nulovém kole během další akce (např útoku) pro Rozhození či Ztráty rovnováhy - v takovém případě si protivník hází na Kondici či Mobilitu proti Lopříšově seslání, při výhře získává Lopříš jednorázové Posílení.\\
Alternativní režim:\\


Igni\\
Vyvolání: 0/2\\
Popis: \\
Základní režim: Toto znamení se využívá při zapalování ohně, svíček, pochodní. Lze použít i na efektní triky. V žádném případě nefunguje jako ohnivá koule, či plamenomet. Podle počtu použitých vyvolání lze zvyšovat výsledek - např. lze použít jako lucernu, za cenu 1 vyvolání / 2 kola atp. V boji proti nestvůrám štítících se ohně se se zapáleným Igni budou od Lopříše tyto bytosti držet dál. Při pokusu o intimidaci (např. dovedností Provokace) získává Posílení Stříbro na upíry. Přirozeně, perfektně se hodí pro zapalování hydřích hlav, což lze provést v nulovém kole po úspěšném útoku.\\
Alternativní režim: Prostě fajrbol. Hodnocení zbraně +2, ignoruje nemagickou zbroj. \\


Quen\\
Vyvolání: 1/1 (při plné obraně)\\
Popis: \\
Základní režim: Quen slouží jako první vrstva obrany před útoky. Zaklínač ji sesílá před bojem, kouzlo pak pohltí první libovolný zásah - ať už stres s posunem 1, nebo extrémní následek.\\
Alternativní režim: Lopříš alternativní režim Quenu používá při plné obraně. Zhmotní před sebou (a všemi, kdo jsou za ním) průhledný štít velikosti pukléř/umbo/mandle, čímž je mu jednak umožněno používat Boj při obraně proti Střelbě, druhak k tomuto získává bonus +1/+2+/+3. Velikost štítu je určena mírou úspěchu, a to 0/+2/+5. V obraně proti útokům zblízka mu alternativní Quen výhodu neposkytuje, nejedná-li se o situaci, během které lze dělat např. jinou činnost druhou (či další? B) ) rukou.\\



Heliotrop\\
Vyvolání: 0/1 (při plné obraně)\\
Popis: \\
Základní režim: Poslední Lopříšovou snahou, než na něj dopadne ohnivá koule, blesk, či magický výboj, je načrtnout ve vzduchu Heliotrop. Díky němu může odrazit či pohltit kouzlo. Výsledek záleží na MÚ, viz tabulka. Porovnává se dovednost sesílání Lopříše vs Čaroděje\\

Míra úspěchu\\
Výsledek\\
<0\\
Zásah za MÚ, nebo zásah někoho jiného\\
0\\
Lopříš získává otřes\\
>0\\
kouzlo odraženo bezpečně pryč\\

Alternativní režim: What doesn’t kill makes you stronger - Netvorobijec tváří v tvář magii usoudí, že nejlepší možnost se jí vyhnout jí prostě použít sám. Při plné obraně (tedy s patřičným bonusem +2) se může svojí dovedností Znamení bránit kouzlům a v případě, že uspěje, nepřítelovo kouzlo pohltí. Obnoví si tak tolik políček many, jaký byl stupeň seslaného kouzla (Ohnivá koule je +2, Blesk z čistého nebe +5 atd.).\\


Yrden\\
Vyvolání: 1/3\\
Popis: \\
Základní režim: Jedno z mocnějších zakletí, kterou Lomyš používá, chce-li se svým protivníkem zaklesnout do bojového tance, či před ním naopak ochránit neprofesionály. Lopříš opíše ve vzduchu kružnici, která se později zhmotní ve sféru. Každá myšlenková bytost se z této arény nemůže dostat ven, stejně tak se nedokáží dostat dovnitř. Tato kopule je samozřejmě prolomitelná, její výdrž závisí na MÚ. Maximální poloměr této sféry je určen tabulkou.\\
Alternativní režim: Zaklínač precizně volí formulaci Yrdenu tak, aby z vzniklého magického pole vytěžil maximum. Jakékoliv projektily magické (příšerácké) povahy, jsou polem zpomaleny, všechny postavy v kopuli Yrdenu se ocitají Pod ochranou fialova a pasivně získávají hodnocení zbroje +1 proti těmto útokům (kromě možnosti standardně s aspektem pracovat). Naopak, útoky zvnitřku nemagické nepříšerácké povahy jsou polem zepičtěny, získávají tedy bonus +1 k hodnocení zbraně.\\

MÚ\\
Maximální poloměr (sázích)\\
0 -1\\
4 \\
2 -3\\
6\\
4 +\\
10\\


\subsection{Lebeňák}
\label{sec:lopr-lebe}

Vyžadovaný aspekt: V\\
Prerekvizita: Dobrá Vůle, Dobrá Kondice\\
Popis: Kdo příliš zírá do propasti… Z Lovce se stane Velmistr a jako takový podstupuje finální proměnu - získává nového symbionta. Při děsivém rituálu se Lopříšovi načne lebka a ta je nahrazena Lebeňákem. Lebeňák umožňuje Lopříšovi nezprostředkovaně komunikovat s bytostmi ze Stínového světa - okamžitě je do jeho zraku zahrnuta perfektně schopnost vnímat myšlenkové bytosti. Stejně dokáže hlubokou meditací odhrnout Stříbrný závoj a své vědomí přemístit do Stínového světa, rodištěm všech myšlenkových bytostí. Proniká tak přímo k problematice svého povolání a dokáže střety s nestvůrami řešit mnohem efektivněji - stane se z něho vyjednávač, výhružník, soudce a kat v jedné osobě. Při střetu s bytostmi ve Stínovém používá relevantní dovednosti - ať už sociální či fyzické.\\

Emocionální guru\\


\subsection{Válečnické speciality}
\label{sec:lopr-val}

Odhad zbraně/zbroje (I)\\
Vyžadovaný aspekt: I\\
Prerekvizita: Slušná Bojová dovednost, Průměrné Vyšetřování\\
Postava dokáže určit parametry, princip zbraně/zbroje. Obtížnost závisí na tom, jaký přístup má ke zbrani (pohled, dotyk, boj). S dalším pozorováním se obtížnost zmenšuje\\

Válečný oř (I)\\
Vyžadovaný aspekt: I\\
Prerekvizita: Slušná Jízda\\
Další požadavky: kůň z dobrého chovu, určený k boji\\
Popis: Postava umí koně vychovat tak, že si ho najde, může umět speciální povely, být rychlejší či vytrvalejší, než je běžné atp.\\


Rychlé tasení (I)\\
Vyžadovaný aspekt: III\\
Prerekvizita: Slušná Bojová dovednost, Dobrá Pozornost, Dobrá Mobilita nebo Slušná Zlodějna \\
Popis: Nikdo nepřátelský postavu nezastihne jinak, než se zbraní v ruce (pokud tedy nějakou zbraň má).\\


Zteč (I)\\
Vyžadovaný aspekt: II\\
Prerekvizita: Slušná Mobilita\\
Popis: Válečník se rychlým sprintem (nebo naopak šikovnými kličkami) rychle blíží ke střelci, a tím mu sníží počet výstřelů na jeho osobu o 1.\\


Pevné zdraví (I)\\
Vyžadovaný aspekt: III\\
Prerekvizita: Dobrá Kondice\\
Popis: Postava získá jedno políčko stresu o hodnocení 2 navíc.\\


\subsection{Obecné speciality}
\label{sec:lopr-obec}

Grimoáry (I)\\
Prerekvizita: Gramotnost\\
Popis: Stínači dostávají přístup k starým svazkům o různých typech nestvůr. Hráč získává možnost si vypůjčit modulové Klíče bytostí astrálních.\\

Zaklínačské vybavení (V)\\
Popis: Lopříš velmistr je váženou osobou jak ve světě lidí, tak ve světě Lovců. Získává přístup k vysoce specializovanému zaklínačskému vybavení, které se perfektně hodí k jeho práci. Ve městech a jiných místech smí navštívit obchody, které toto vybavení prodávají, stejně tak jako za bod osudu kdykoliv vhodný předmět “vytáhnout z batohu” - svěcenou vodu, chránič na krk, zrcadlo apod.\\

\subsection{Bylinkářství (N)}
\label{sec:lopr-byliny}

Popis: Funguje jako standardní dovednost Medicína - umožňuje sbírat byliny a z nich připravovat masti, výluh a další, zároveň lze použít jako improvizovanou první pomoc, je-li proto vhodné prostředí a narativ.\\

Byliny\\

Lovci přízraků používají v boji mnoho prostředků, mezi něž patří i různé byliny. Mezi nejpoužívanější patří byliny proti myšlenkovým bytostem, kterými potírají své zbraně či si na ně tyto byliny přivazují. Dále využívají byliny odstraňující působení různých kouzel, prokletí či paralýz, na které jim nestačí jejich schopnosti. Vcelku se jedná o byliny jedovaté, které ovšem dokáže lovec přízraků bez problému přežít. Ne všechny látky je ovšem schopen zneutralizovat hned od začátku, a proto je vždy za názvem rostliny číslo, udávající příslušnou úroveň,od které ji může bezpečně využívat. Cena je jen orientační, protože tyto rostliny nejsou běžně k dostání. Na konci popisu je uvedeno zranění, které utrpí obyčejný člověk či lovec přízraků nemající dostatečnou úroveň.Tento seznam je jen orientační a PJ může obdobně vytvářet další látky. Lovec přízraků je schopen zpracovat zde uvedené byliny pro použití.\\

Arka polární \\
Sklizeň: celý rok\\
Použitelná část: květy\\
Výskyt: oblasti s nízkou teplotou\\
Nalezení: 1 × 5 \\
Cena: 10 zl (svazek 12 rostlin)\\
Četnost: 2 dny\\
Z bílých květů se dělá čaj, který odstraňuje následky paralýzy. Účinkuje i na nej závažnější paralýzy, a proto je velmi hojně využívaná.\\
trvání 5 kol/ihned.\\

Blatan\\
Sklizeň: podzim\\
Použitelná část: listy\\
Výskyt: bažinaté oblasti\\
Nalezení: 2 × 10 \\
Cena: 5 zl (svazek 12 rostlin)\\
Četnost: 3 dny\\
Listy se v surové podobě žvýkají a pomáhají očistit duši i tělo. Po 10 minutách žvýkání se odstraní tolik bodů únavy, kolik činí odolnost postavy.\\
Trvání 3 kol/ihned\\

Fiwari\\
Sklizeň: po celý rok\\
Použitelná část: stonek\\
Výskyt: pláně\\
Nalezení: 2 × 20 \\
Cena: 30 zl (svazek 12 rostlin)\\
Četnost: 1 den\\
Spolu s několika dalšími, lehce sehnatelnými látkami (5 md) se vytvoří zelená směs, která nevábně vypadá i chutná. Dokáže ovšem mnohonásobně zrychlit léčivý proces. Sama si ovšem také bere živiny z těla, a proto je nutné po použití živiny opět doplnit. Léčí 5–30 životů, zároveň ovšem přidává 2–12 bodů únavy.\\

Odl ~ 12 ~ 2–12 bodů únavy/4–24\\
bodů únavy, trvání ihned/1 kolo.\\

Hatrba vlkodlačí\\
Sklizeň: při úplňku\\
Použitelná část: kořeny\\
Výskyt: dna vyschlých řek\\
Nalezení: 1 × 5 \\
Cena: 200 zl (12 kořenů)\\
Četnost: 1 měsíc\\
Drcené kořeny hatrby pomáhají proti nemoci zvané lykantropie. Při včasném užití je pravděpodobnost nakažení téměř nulová. Při pravidelném používání je možno omezit přeměny ve vlka o úplňcích (první fáze nemoci) a nakonec úplně vyléčit. Při používání ve druhé fázi nemoci je omezení přeměny jen částečné, záleží na vůli léčeného. Třetí fázi nelze léčit.\\
Trvání 1 hodina/ihned.\\

Lavelaj královský \\
Sklizeň: na konci léta\\
Použitelná část: celá rostlina\\
Výskyt: deštné pralesy\\
Nalezení: 3 × 5 \\
Cena: 50 zl (svazek 12 rostlin)\\
Četnost: 2 dny\\
Vývar z této nahnědlé rostliny dokáže odstranit jedovaté látky z těla či aspoň jejich větší část. Nevýhodou je, že zřeďuje krev, což by normálního člověka po několikátém užití zabilo.\\

Petro daleký \\
Sklizeň: brzy z jara\\
Použitelná část: listy a květy\\
Výskyt: údolí v Zelanských vrchách\\
Nalezení: 2 × 10 \\
Cena: 35 zl (svazek 12 rostlin)\\
Četnost: 2 dny\\
Drcené květy a listy se zalijí vařící vodou a tím vznikne odvar, který pomáhá proti temným kouzlům. Kdo ho vypije, ten se zbaví působení všech kouzel černé magie a příštích 12 hodin na něj nebudou působit.\\
trvání ihned/10\\


Siltavia \\
Sklizeň: po celý rok\\
Použitelná část: celá rostlina\\
Výskyt: téměř všude\\
Nalezení: 3 × 70 \\
Cena: 5 st (svazek 12 rostlin)\\
Četnost: 12 hodin''
Tato rostlina se uchytí téměř všude, kde je dostatek slunce. Vodu čerpá přímo ze vzduchu, takže kromě pouští ji lze najít kdekoli. Její výrazné rudé listy jsou vidět už z velké dálky.\\
Zkušenější lovci přízraků se o ní nezajímají, těm mladším může přijít vhod. Dokáže posílit smysly celého těla a tím zabránit nepříjemnému překvapení. \\

Vertika ansala\\
Sklizeň: zimní slunovrat\\
Použitelná část: plody\\
Výskyt: silná ranga\\
Nalezení: 1 × 5 \\
Cena: 200 zl (1 plod)\\
Četnost: 10 dní\\
Tato rostlina, která má magické účinky, je velice žádaným zbožím. Alchymisté oceňují její magickou energii (500–700 magů) a lovci přízraků její schopnosti. Snědením jednoho plodu této rostliny se stane lovec přízraků naprosto neviditelný pro myšlenkové bytosti na 6 hodin. Tato rostlina velmi pomáhá při boji s nižšími myšlenkovými bytostmi.\\


\end{document}
%%% Local Variables:
%%% mode: LaTeX
%%% TeX-master: "../../main"
%%% End:
