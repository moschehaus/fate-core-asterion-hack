% !TEX root = main.tex

\documentclass[../main.tex]{subfiles}
\begin{document}

\section{Edenův bratr}
\label{sec:edenuvbratr}

Organizace
pod hlavičkou státu
cílem je elitní organizace
experiment, povedl se
využití propojení dvojčat

\subsection{Mechanismus kouzel}
\label{sec:eden-mech-kouzla}

Postava s příslušným aspektem získává za každý další získaný body aury zdarma. Na první úrovni aspektu získá zdarma bod aury 1. stupně, za druhý aspekt bod aury 2. stupně atd. Za každý jeden investovaný dovednostní bod se zvýší aura o jeden bod a Edenův bratr si odemkne 2 kouzla. Bonus k hodu při kouzlení odpovídá úrovni aspektu (na prvním aspektu je bonus 0, na druhém +1 atd.)\\

\subsection{Válečnické věci + Posílení}
\label{sec:eden-valecnici-posileni}

Odhad zbraně/zbroje (I)\\
Dovednost:\\
bojová dovednost – slušná,\\
zkoumání – průměrné.\\
Postava dokáže určit parametry, princip zbraně/zbroje. Obtížnost závisí na tom, jaký přístup má ke zbrani (pohled, dotyk, boj). S dalším pozorováním se obtížnost zmenšuje.\\
Edenův bratr zjistí slabinu zbraně/zbroje. Bojovník se může rozhodnout zaútočit na tuto slabinu, při úspěchu nezraní nositele, ale způsobí větší poškození předmětu.\\


Válečný oř (II)\\
Dovednost:\\
 jezdectví – slušné\\
Další požadavky: kůň z dobrého chovu, určený k boji\\
Postava umí koně vychovat tak, že si ho najde, může umět speciální povely, být rychlejší či vytrvalejší, než je běžné atp.\\
Pomocí telepatických schopností zbraně může Edenův bratr udělit koni jednoduchý příkaz shrnutý v jedné větě. Například: “Jdi tam a čekej dokud nepřijdu”. Jaký příkaz splní a jaký už ne závisí na MU.\\

Zastrašení (II)\\
Dovednost: \\
zaujmutí pozornosti – slušné nebo klam – slušný\\
Požadavky: jakákoliv zbraň\\
Bojovník nepřátele znejistí, nebo přímo přesvědčí, že by z boje s ním nevyšli bez újmy, takže se střetu raději vyhnou.
Postava zároveň s akcí útok použije na cíl dovednost zastrašení. Hází si proti vůli cíle. Pokud se jí hod povede, cíl se stane nejistým, toto kolo nejistá postava nezaútočí a má -1 ke všem hodům proti postavě. Pokud válečník uspěje s MU 3 a více, cíl vystraší a ten opustí boj. \\
Po použití tohoto posílení se může bojovník pokusit vystrašit až skupinu 3 nepřátel. Nebo toto posílení může použít ve chvíli, kdy se mu už jednou podařilo cíl vystrašit, čímž způsobí, že sníží hranici na úplné vystrašení na MU 1.\\

Improvizace v krvi (III)\\
Dovednost\\
bojová dovednost související s improvizovaným bojem – dobrá\\
Bojovník smí přezbrojit (za zbraň v okolí) v nulovém čase (pokud je jeho nová zbraň přiměřeně po ruce, nebo pokud to dokáže dobře vysvětlit).\\
Edenův bratr smí po zaplacení aury přivolat svoji zbraň až ze vzdálenosti jednoho kola sprintu. Pokud zbraň není přístupná (schovaná, pod něčím) je třeba ověření hodem.\\

Taneční parket (III)\\
Dovednost:\\
mrštnost – dobrá\\
Postava získává bonus +1 k obraně, používá-li u toho mrštnost. Nesmí mít zbroj těžší než je kožená.\\
Bojovník smí udělat akrobatický kousek hodný akčního filmu - efekt je ekvivalentní s použitím bodu chi u bojového monka.\\

Rychlé tasení (III)\\
Dovednost:\\
bojová dovednost – slušná,\\
obezřetnost – dobrá,\\
mrštnost – dobrá, nebo kapsářství – slušné.\\
Nikdo nepřátelský postavu nezastihne jinak, než se zbraní v ruce (pokud tedy nějakou zbraň má).\\

Předvídavost (IV)\\
Dovednost:\\
obezřetnost – dobrá\\
Šermíř nahlásí během boje dvě akce a při lámání chleba určí jednu z nich. Při prvním kole může mít bojovník otřesení.\\
Šermíř smí provést dvě akce během jednoho kola. \\

Boj proti přesile (VI)\\
Dovednost:\\
bojová dovednost – výborná,\\
mrštnost – výborná nebo odolnost – výborná.\\
Další požadavky: víc nepřátel\\
Pro mistry boje přestávají platit obvyklé hranice. Bojovník nedostává žádný postih za přečíslení, ale za obklíčení (popř. útok do zad) ano\\
Z Edenova bratra se stane totální mašina na krájení a nedostává tak postih ani za obklíčení, ani za útok do zad.\\


\subsection{Znamení}
\label{sec:eden-znameni}


Zostření smyslů (I) \\
Trvání: 1 scéna\\
Magická podstata zbraně podpoří přirozené smysly bojovníka a činí je nadlidsky přesnými. Bratr tak lépe vidí (i v noci), slyší neslyšitelné a po hmatu rozezná králíka od zajíce stejně dobře jako svíčkovou od smrtícího jedu. \\

Infravidění (I)\\
Trvání: 1 scéna\\
Zbraň propůjčí nositeli infravidění, totožné, jako má trpaslík. Je tedy schopen rozeznat tepelné stopy bytostí, nevidí ale přes objekty.\\

Mentální štít (II)\\
Trvání: viz popis\\
Mysl bojovníka je magicky zpevněna bratrem ve zbrani. Tento bonus plati proti jakémukoliv pokusu o vniknutí do mysli (libovolného povolání), jako je mentální úder, čtení myšlenek, sugesce atd. Základní bonus je +1, bratr se ale může rozhodnout bonus zvýšit na úkor trvání - co o řádek nižší doba trvání, to další bonus +1.\\

MU\\
Doba trvání\\
0-1\\
4 kola\\
2-3\\
6 kol\\
4-5\\
souboj \\
6 a více\\
do konce sezení\\



Extra síla (II)\\
Trvání: 1 kolo (akce)\\
Bratr napne své síly a pro jednu akci dokáže nepředstavitelné. Může vyrážet dveře, zvedat balvany, držet jednou rukou kamarády… Při použití v boji způsobí jednorázový bonus +1 k hodnocení zbraně.\\

Ignorace bolesti (III)\\
Trvání: 6 kol (viz popis)\\
Bratr se naučí lépe využívat svojí životní energii a dokáže se tak naplno soustředit na boj, nevnímaje zranění. Dle MU může ignorovat až 3 typy zranění (otřes, lehké zranění, těžké zranění). Za každé snížení doby trvání o dvě kola si bratr může zvýšit MU o 2 a naopak - za snížení MU o 2 si smí zvýšit dobu trvání o dvě kola.\\

MU\\
Počet ignorovaných typů zranění\\
0-1\\
1\\
2-4\\
2\\
5 a více\\
3\\


Falešná smrt (III)\\
Trvání: dle MU\\
Bojovník umí přelévat esenci života mezi sebou a zbraní, čímž dokáže fingovat smrt. Vezme-li mu v průběhu předstírání smrti někdo jeho zbraň, zkrátí se doba trvání na polovinu (tedy pokud mu někdo zbraň odcizí po tom, co byl více než polovinu v předstírané smrti, ihned se probere!). Navíc smí každé kolo hodit na Kondici a odškrtnout si taková políčka stresu, aby součet jejich velikostí byl nejvýše MU ( s pastí 0).\\

MU\\
Doba trvání\\
0-1\\
6 kol\\
2-4\\
scéna\\
6 a více\\
libovolně, maximálně do konce sezení\\



Stínové zranění (IV)\\
Trvání: viz popis\\
Bratr může toto znamení použít dvěma způsoby - při úspěšném útoku, chce-li jeho efekt posílit, nebo naopak při neúspešném útoku, ve snaze dosáhnout alespoň něčeho. Při úspěšném útoku ošálí meč zraněného a ten si bude myslet, že byl velice vážně zraněn. Cíl si hází na vůli proti aspektu útočníka a je “zraněn” za rozdíl. Toto zranění ale není opravdové, není škrtnuto v tabulce a nedá se ho zbavit ani lektvarem. Projevují se ale jako běžné následky - tedy aspekt s volným vyvoláním a vše další. Úplně stejně může Bratr postupovat v případě nepodařeného útoku. . Stínová zranění mizí stejně rychle jako ta běžná (těžké může odpadnout rychleji, uvědomí-li si terč, že ve skutečnosti není zraněn)\\

Šestý smysl (IV)\\
Trvání: dle MU\\
Živoucí zbraň poskytuje svému společníkovi informace o okolí. Získává pro dobu trvání aspekt Šestý smysl s volným počtem vyvolání odpovídající pravidlům. Nemůže se tak stát, že bude překvapen útokem ze zad, při porovnávání (např. proti pastím) smí tento aspekt vyvolat. Varuje pouze před nebezpečím.\\
MU\\
Doba trvání\\
0-1\\
nejbližší podnět\\
2-4\\
4 kola\\
6 a více\\
scéna\\



Výkřik (V)\\
Trvání: okamžitě\\
Bratrova zbraň vyšle telepatický výkřik, který ochromí všechny bytosti v okruhu 5 m. Ty si hodí na vůli - neuspějí li, jsou mentálně omráčeni na 1k6 kol. Na začátku každého kola si hází na vůli, hodí-li na kostkách více, než hodil Bratr původně, proberou se. Počínaje druhým kolem získávají každé kolo k tomuto hodu bonus +1.\\

Berserk (V)\\
Trvání: scéna\\
Toto znamení dodá Bratrovi dávku mixu sebevědomí a agrese, které způsobí, že si bude připadat neporazitelný a bude tak i navenek působit. Získává na scénu aspekt Berserk, který může jednou zdarma vyvolat a k němu i bonus na Zastrašení dle MU.\\

MU\\
Bonus k zastrašení\\
0-2\\
+1\\
3-5\\
+2\\
6 a více\\
+3\\


Vysaj sílu (VI)\\
Trvání: souboj\\
Bratr se zaměří na jednoho protivníka, kterého očaruje, ten si hází na vůli. Za veškeré způsobené zranění protivníkovi se Bratr léčí, plus si přidává bonusové životy. Dle svého původního MU se Bratrovi navyšuje MU do každého útoku, nicméně zraní pouze za původní útočné MU. Svůj vyléčený podíl získá tak, že k míře způsobeného zranění připočte bonusové MU - toto “zranění” odpovídá maximálnímu možnému vyléčenému políčku bratra.\\

MU\\
Bonusové MU\\
0-1\\
0\\
2-4\\
+1\\
5 a více\\
+2\\


Antimagická zbraň (VI)\\
Trvání: scéna\\
Bratr ve zbrani využije svůj plný magický potenciál a způsobí, že zbraň bude krájet magická pole jako máslo. Protivníci si nepočítají hodnocení zbroje způsobené magii, magické štíty, ani magicky zpevněné zbroje.\\


\subsection{Rychlá smrt}
\label{sec:eden-rychla}


Bleskový útok (I)\\
Ed smí zaútočit tak rychle, že většina nepřátel nepostřehne jeho pohyb. Smí použít dovednost Pozornost k provedení Bleskového útoku na nepřítele v téže zóně, tedy jako kdyby používal běžnou dovednost Boj. To ho stojí políčko stresu (či následků) a k tomu na sebe umisťuje aspekt Multitasking s jedním volným vyvoláním.\\

První rána (II)\\
Na začátku souboje, kdy má Ed jednat jako první (tedy má nejvyšší Pozornost), smí provést na kohokoliv Bleskový útok bez nutnosti platit stres jako akci navíc.\\

Protego (II)\\
Kdykoliv, kdy Ed uspěje se stylem na obranu, smí získané posílení použít na provedení Bleskového útoku bez nutnosti platit stres jako akci navíc.\\

Zawardo (IV)\\
Ten pak získává aspekt Zrychlený na scénu, díky kterému má vždy:
bonus +2 při porovnávání Pozornosti pro potřeby určení iniciativy v konfliktu\\
na začátku každého kola se přemístit na libovolnou sousední zónu, nebrání-li mu v tom aspekt prostředí\\
za zaplacení bodu osudu smí provést dvě akce v jednom kole, jako kdyby byl dvěma hráči zároveň, tedy, např. v případě, že se proti němu protivník se stylem ubrání, získává proti Edovi  posílení\\

Adrenalin rush (III)\\
Ed dokáže silou vůle ovládat vyplavování adrenalinu ve svých nadledvinkách a tím si odstranit zaškrtnutá políčka stresu. Hodí si na vůli a obnoví si taková políčka stresu, aby součet jejich velikostí byl nejvýše roven MÚ (s pastí 0). Tím ale své tělo vytěžuje více, než je schopno dlouhodobě snášet a musí si tedy hodit i na Kondici. Pokud je hod na Vůli nižší, umístí na sebe Ed aspekt Víra v ledvina.\\

Tanec s čepelí (V)\\
V euforii z rychlého pohybu a umírajících nepřátel se Edovi do krevního oběhu vylévá nadpozemské množství katecholaminů, což má za následek rychlost. Za každého nepřítele, kterého zabije Bleskovým útokem získává další Bleskový útok zdarma jako akci navíc bez nutnosti platit stres.\\




\end{document}

%%% Local Variables:
%%% mode: LaTeX
%%% TeX-master: "../../main"
%%% End:
