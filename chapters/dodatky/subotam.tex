\documentclass[../../main.tex]{subfiles}
\graphicspath{{\subfix{../../graphics/}}}
\begin{document}

\subsection{Válečnické speciality}
\label{sec:val-spec-sub}

Bylinkářství \\
Vyžadovaný aspekt: I\\
Popis: Funguje jako standardní dovednost Medicína - umožňuje sbírat byliny a z nich připravovat masti, výluh a další, zároveň lze použít jako improvizovanou první pomoc, je-li proto vhodné prostředí a narativ.\\


Rychlé tasení\\
Vyžadovaný aspekt: III\\
Prerekvizita: Slušná Bojová dovednost, Dobrá Pozornost, Dobrá Mobilita nebo Slušná Zlodějna \\
Popis: Nikdo nepřátelský postavu nezastihne jinak, než se zbraní v ruce (pokud tedy nějakou zbraň má).\\


Zteč \\
Vyžadovaný aspekt: II\\
Prerekvizita: Slušná Mobilita\\
Popis: Válečník se rychlým sprintem (nebo naopak šikovnými kličkami) rychle blíží ke střelci, a tím mu sníží počet výstřelů na jeho osobu o 1.\\


Taneční parket \\
Vyžadovaný aspekt: II\\
Prerekvizita: mobilita – dobrá\\
Popis: Postava se může pokusit v nulovém kole přesunout do vedlejší zóny. Hází si standardně na past (akce Překonání) a uspěje-li, v témže kole se do zóny přesune. Platí všechny výsledky akce Překonání.\\


Předvídavost \\
Vyžadovaný aspekt: IV\\
Prerikvizita: obezřetnost – dobrá\\
Popis: Šermíř nahlásí během boje dvě akce a při lámání chleba určí jednu z nich. \\



Boj proti přesile\\
Vyžadovaný aspekt: VI\\
Prerekvizita: bojová dovednost – výborná, mrštnost – výborná nebo odolnost – výborná.\\
Další požadavky: víc nepřátel\\
Popis: Pro mistry boje přestávají platit obvyklé hranice. V případě, že bojuje proti více nepřátelům, nemohou použít Týmovou akci - musí s ním bojovat každý zvlášť.\\


Sart\\


\subsection{Zázraky}
\label{sec:subotam-zazraky}

\subsubsection{Mechanismus zázraků}
Zázraky se používají za seslání. Kněz na každé úrovni aspektu obdrží ke každému zázraku nižší úrovně než je jeho dosavadní aspekt jedno seslání zdarma. Za každé další platí dovednostní.  Zázraky se zpravidla sesílají v nulovém kole, není-li řečeno jinak.

\subsubsection{Krátkodobé modlitby (5)}

Nápaditý boj\\
Seslání: 0\\
Popis: Subotam si obtiskne do  mozku pozorované pohyby a  získává tak  možnost zkopírovat libovolnou fintu, kterou uvidí; tu pak smí sam použít. Může si jí taktéž držet v paměti do prvního použití; stejně tak ji může vypozorovat přímo v boji.\\


Odhad soupeře\\
Seslání: 0 pokud kněz vyhraje iniciativu/1\\
Popis: Kněz zrakem svého boha prohlédne soka. Hodí si na svoji Pozornost proti protivníkově Klamu. V případě úspěchu pak odhalí jeho bojový aspekt a dokáže-li vhodným narativem popsat, jak ho využije pro získání výhody, má na něj jedno volné vyvolání. Lze použít na každého nepřítele jednou/souboj. \\


\subsubsection{Krátkodobé obřady (4)}


Dorovnej šance\\
Seslání: 0 \\
Popis: Bůh boje po prosbě kněze sejme negativní aspekt spojený s bojem z postavy (vlastní či cizí).  Kněz si hodí na vůli proti MÚ, která negativní aspekt na postavu umístila a vyhraje-li, aspekt je sejmut, je-li jednorázový. \\


Neviditelnost\\
Seslání: 1\\
Popis: Kněz se zakryje roušku neviditelnosti. Na scénu se stane neviditelným; pro mechanické porovnávání má aspekt Neviditelný a bonus +1 na relevantní dovednosti. V boji neviditelnost končí. \\
Naopak, v průběhu boje může tímto zázrakem svoji zbraň učinit neviditelnou. Ta pak získává aspekt Smrt z neznáma (bez volného vyvolání) a pasivní bonus +1 k hodnocení zbraně.\\


\subsubsection{Dlouhodobé modlitby (3)}


Šampión bohů\\
Seslání: 0\\
Popis: Seš kněz boha války. Chápeš. Po seslání zázraku se kněžímu obnoví všechny políčka stresu a získá bonusové políčko drobného (+2) a mírného (+4) následku a k tomu se kněžímu neutralizují všechny DoT efekty jako Krvácení, Otrava. Trvá celou scénu.\\

Heroismus\\
Seslání: 0\\
Popis: Pořád seš kněz boha války. V případě, kdy osudí není na tvé straně - nepřátel je mnohem více, jste v bezradné situaci apod. (podrobnosti na uvážení PJ), Gor si uslintne a dá ti, co potřebuješ.  Získáváš aspekt Heroismus s jedním volným vyvoláním, který je použitelný např. pro strhnutí davu, zastrašení apod., bonus +1, který smíš do konce scény přiřadit libovolné dovednosti, a nakloníš si osud zpět k sobě - získávaš 2 body osudu. 
	V případě krajní nouze: Subotam se rozhodne, že toto je vhodná chvíle skončit. Sešle Heroismus, a dokud svůj úkol nesplní, nemůže zemřít. Je na rozhodnutí PJ, zda-li se souboj odehraje mechanicky, či jestli bude zajímavější jej pouze odvyprávět. V případě, že se Subotam pro tuto možnost rozhodne, nesmí používat žádné Gorovy zázraky (jeho přízeň si vyčerpal), dokud jej osobně nenavštíví v jeho zahradě ve Stínovém světě.\\

\subsubsection{Dlouhodobé obřady (2)}


Požehnání v boji se zlem\\
Seslání: 1 (mimo boj)\\
Popis: Subotam provede rituál a připraví se a své spolubojovníky tak na nadcházející souboj s temnou stranou síly.  Každý zúčastněný dostane bonus +2, který smí přerozdělit do posílení svého Hodnocení zbraně či zbroje. Trvá celou scénu. \\


Posílení vlastnosti\\
Seslání: 1 (mimo boj)\\
Popis: Subotam provede rituál a připraví se a své spolubojovníky tak na nadcházející souboj s temnou stranou síly. Každý zúčastněný dostane bonus +2, který smí přerozdělit do posílení svých dovedností. \\








\end{document}

%%% Local Variables:
%%% mode: LaTeX
%%% TeX-master: "../../main"
%%% End:
