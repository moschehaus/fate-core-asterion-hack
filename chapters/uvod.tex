\documentclass[../main.tex]{subfiles}
\graphicspath{{\subfix{../graphics/}}}
\begin{document}

Tento text vzniká pro potřeby skupiny hráčů (``družiny'' v nerdovském žargónu) s velmi specifickými požadavky. Tak ponejprv již název tohoto hacku prozrazuje, že tyto pravidla jsou namíru ušita fantasy settingu Asterion. Neznamená to a priori, že je nelze použít \textit{jinde}; ve skutečnosti je Asterion generické fantasy s jen málo odlišnostmi. 

Ony odlišnosti však v naší skupině hrají důležitou roli:

\begin{itemize}
\item v družině je čarodějník (alchymista, theurg, mág, čaroděj) $\Rightarrow$ magie musí být ``asterionská''
\item v družině je kněz, a ten je navíc lovec nestvůr \Rightarrow kněžství a nestvůry musí být asterionské
\item v družině je sicco (špeh) \Rightarrow společenství a politické struktury musí být asterionské
\end{itemize}

Dlouhou dobu jsme si vystačili s jinými pravidly, co dal dohromady náš dobrý kamarád. Motivace jejich vzniku byla prakticky totožná s motivací vzniku těchto - poskytnout most mezi Asterionem \footnote{Asterion je setting a jako takový nemá žádný privilegovaný systém pravidel, ve kterých by se měl hrát. Nicméně, autoři podpořily aplikaci pravidel na herní systémy DrD 1.6 a DrD+} a pravidly Fate. M16-ova pravidla jsou skvělá, originální a v rozsahu těžko překonatelná; důvodem, proč ale vznikají tyto, je další podivné specifikum naší skupiny: požadavek osvojit si systém pravidel, který:

\begin{itemize}
\item dokáže uspokojivě pracovat s charaktery postav - ale ne čistě roleplayově
\item zvládne postihnout celou šíři konfliktů, aniž by každý musel mít svá pravidla (fyzický, argumentační, politický, sociální, magický apod.)
\item umožňuje postavám růst a vyvíjet se, získávat nové dovednosti
\item dokáže jednotně popsat magické a jiné speciální předměty
\item umožnit pseudotaktický hodokvas při fyzických soubojích a přitom neunavit jejich dynamiku
\end{itemize}

\textit{A přitom pravidla nesmí být moc dlouhá a komplexní, protože je dost pravděpodobně nebude číst nikdo jiný než vypravěč.}

Pravidla M16 se ukázala být v těchto kritériích nedostatečná; experiment však ukázal, že nadějným kandidátem na pro-nás vhodný systém se \textit{může} stát \href{https://fate.nepocitacovehry.cz/?do=StahniFC}{Fate Core}, zejména kvůli jeho vylepšeným zacházením s \asp{Aspekty}. \footnote{Pravidla M16 stavějí na prehistorické verzi Fate Core zvané Fudge. V tomto smyslu lze tento hack brát jako update M16 pravidel na moderní verzi Fate Core.} Další kapitoly si proto kladou za cíl sepsat to podstatné z Fate Core, co je ovlivněno asterionským settingem, a přitom text ponechat co nejstručnějším. 


\end{document}
 
