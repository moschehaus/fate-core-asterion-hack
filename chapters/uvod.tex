\chapter{Úvod}
\label{chap:introduction}

\section{Proč vznikají tato pravidla}
\label{sec:proc-pravidla}


Tento text vzniká pro potřeby skupiny hráčů (``družiny'' v nerdovském žargónu) s velmi specifickými požadavky. Tak ponejprv již název tohoto hacku prozrazuje, že naše pravidla jsou namíru ušita fantasy settingu Asterion. Neznamená to a priori, že je nelze použít \textit{jinde}; ve skutečnosti je Asterion generické fantasy s jen málo odlišnostmi. 

Ony odlišnosti však v naší skupině hrají důležitou roli:

\begin{itemize}
\item v družině je čarodějník (alchymista, theurg, mág, čaroděj) $\Rightarrow$ magie musí být ``asterionská''
\item v družině je kněz, a ten je navíc lovec nestvůr \Rightarrow kněžství a nestvůry musí být asterionské
\item v družině je sicco (špeh) \Rightarrow společenství a politické struktury musí být asterionské
\end{itemize}

Dlouhou dobu jsme si vystačili s jinými pravidly, co dal dohromady náš dobrý kamarád. Motivace jejich vzniku byla prakticky totožná s motivací vzniku těchto - poskytnout most mezi Asterionem \footnote{Asterion je setting a jako takový nemá žádný privilegovaný systém pravidel, ve kterých by se měl hrát. Nicméně, autoři podpořily aplikaci pravidel na herní systémy DrD 1.6 a DrD+} a pravidly Fate. M16-ova pravidla jsou skvělá, originální a v rozsahu těžko překonatelná; důvodem, proč ale vznikají tyto, je další podivné specifikum naší skupiny: požadavek osvojit si systém pravidel, který:

\begin{itemize}
\item dokáže uspokojivě pracovat s charaktery postav - ale ne čistě roleplayově
\item zvládne postihnout celou šíři konfliktů, aniž by každý musel mít svá pravidla (fyzický, argumentační, politický, sociální, magický apod.)
\item umožňuje postavám růst a vyvíjet se, získávat nové dovednosti
\item dokáže jednotně popsat magické a jiné speciální předměty
\item umožnit pseudotaktický hodokvas při fyzických soubojích a přitom neunavit jejich dynamiku
\end{itemize}

\textit{A přitom pravidla nesmí být moc dlouhá a komplexní, protože je dost pravděpodobně nebude číst nikdo jiný než vypravěč.}

Pravidla M16 se ukázala být v těchto kritériích nedostatečná; experiment však ukázal, že nadějným kandidátem na pro-nás vhodný systém se \textit{může} stát \href{https://fate.nepocitacovehry.cz/?do=StahniFC}{Fate Core}, zejména kvůli jeho vylepšeným zacházením s \asp{Aspekty}. \footnote{Pravidla M16 stavějí na prehistorické verzi Fate Core zvané Fudge. V tomto smyslu lze tento hack brát jako update M16 pravidel na moderní verzi Fate Core.} Další kapitoly si proto kladou za cíl sepsat to podstatné z Fate Core, co je ovlivněno asterionským settingem, a přitom text ponechat co nejstručnějším. 

\section{Smysl hraní aneb kdy používat mechaniku}
\label{sec:proc}

Než se vrhneme do vysvětlování mechaniky, je na místě se zamyslet, v jakých případech ji používat a čeho tím chceme docílit. Hluboce netriviálním filozofováním nad smyslem našeho konání jsme došli k závěru (rozuměj: prostě jsme si to řekli), že primárním cílem hraní je pro nás \textbf{zábava}; a tedy ne utváření osudu Asterionu, rozvoj osobnostních dovedností ani skupinová terapie. Chceme si užít zábavu.\\

Autor se domnívá, že smysl činnosti by měl být vždy na mysli při vykonávání činnosti samé. Neboli, smysl hry, zábava, je vždy nadřazená všemu jinému - mechanice, pravidlům, ``toho jak je to správně''. Stalo se a jistě se i stane, že jsme narazili na situaci, jejíž pravidlové postihnutí by bylo buď složité, nebo nezábavné, nebo dokonce zcela znemožňující situaci vyřešit dle vůle postav (což by se ve světě, který je celý smyšlený, dít nemělo). V takových případech jsme si osvojili \textbf{Zlaté pravidlo}:

\begin{tcolorbox}
  \centering
  \textbf{Zlaté pravidlo}: Nejdřív vyprávěj, pak hledej mechaniku.
\end{tcolorbox}

Pravidlo jednoduše říká, že nejprve postavy vypráví, co dělají a až poté hledáme vhodná pravidla, kterými jejich akce popíšeme. A ne naopak. Jen si to představte: epický souboj na mostě s nepřítelem, co družině už 4 dobrodružství ničí životy, při kterém se najednou most zhroutí a všichni se ocitnou ve vodě. Nepřítel je černokněžník, neumí dobře plavat - proto dobrodruha napadne, že se pokusí záporáka utopit. \textbf{Jenže neexistuje mechanika topení}. Nikde se nepíše ``kolik posunů způsobí kolo topení'', ``na jakou dovednost si házím já a na jakou protivník''. Vypravěč by však neměl říci, že ho nemůžete utopit, jenom protože na to oficiálně nejsou pravidla. Naopak, měl by družinu pochválit za originální nápada a pak společně nalézt způsob, jak situaci vyřešit.\\

Pravidla používáme, abychom férovým, objektivním a konzistentním způsobem popsali akce postav a okolí. Pokud pravidla nějakou akci neumí popsat, neznamená to, že nemůže nastat (resp. že by neměla nastat).

\section{Slovník pojmů a značení}
\label{sec:slovnik}
Pro lepší orientaci v textu zde uvedeme používané značení.

\begin{itemize}
\item Veškeré aspekty jsou vysázeny tučně a kurzívou: \asp{Solidně napsaná kniha}.
\item Názvy dovedností jsou vysázeny kurzívou: \textit{Vůle, Učenost, Kontakty}
\item Typy akcí jsou vysázeny monospace fontem (psací stroj): \texttt{Překonání, Vytvoření výhody, Útok, Obrana}
\end{itemize}


 

%%% Local Variables:
%%% mode: LaTeX
%%% TeX-master: "../main"
%%% End:
