\chapter{Střety, výzvy a konflikty}
\label{chap:jdesenavec}

\section{Jeden hod nebo peripetie?}
\label{sec:hod/peripetie}

Dalším rozdílem oproti pravidlům M16 je klasifikace nětkerých specifických usílích jako zvláštní typ akce. V našem slovníku jim budeme říkat \underline{Peripetie}. Než se pustíme do prezentace těchto typů, je dobré se uvědomit, že ve většině případů stačí jediný hod na dovednost (nebo dokonce žádný), abychom rozhodli, jak ono snažení dopadne. Nemusí přitom nutně záležet na tom, jak dlouho úsilí trvá - překonat skalnatou roklinu, i když se jedná o náročnou celodenní tůru, může být pořád jeden hod na \textit{Mobilitu}, projití Džungle padlých stromů, seč anabáze na několik aldenů, může stále být jediný hod na \textit{Kondici}.

Existují ale situace, které chceme odehrát ve větších podrobnostech, i kdyby se jednalo jen o několik minut herního času; typicky jsou to akce, jejichž výsledek není samozřejmý, (ne)úspěch by byl zajímavý nebo akce prostě nejsou natolik přímočaré, aby se daly vyřešit jediným hodem. Jaké úsilí nemá smysl řešit vůbec, jaké bez hodu, jaké s jedním hodem a jaké jako peripetii není dáno ultimátně. Dobře slouží přirovnání hry s knihou nebo seriálem; ve většině případů není zajímavé sledovat, jak postava jí nebo jde spát (a navíc to zvládá bez potíží). Může se ale stát, že jídlo je otrávené, v posteli je nastrčený jedovatý had, nebo za dveřmi čeká zabiják - v takovém případě je na místě situaci s jezením vyřešit (např.) jedním hodem na \textit{Pozornost} a souboj se zabijákem jako konflikt. Na druhou stranu, pokud je celé dobrodružství protkané zabijáky, co na vás číhají na každém rohu, není třeba každé střetnutí řešit konfliktem - tentokrát zde třeba postačí jediný hod.

V těchto pravidlech budeme rozlišovat základní tři typy peripetií:

\begin{itemize}
\item \underline{Výzva} je o překonání dynamických komplikovaných překážek či vyřešení prekérních situací, na které nestačí jediný hod na \texttt{Překonání}
\item \underline{Střet} je neshoda dvou či více postav ohledně téhož zájmu, ale s různým cílem, které je strukturovanější než jediný hod na \texttt{Překonání}
\item \underline{Konflikt} je snaha postav si ublížit - fyzicky, duševně, magicky, peněžně atd.
\end{itemize}

\section{Výzvy}
\label{sec:vyzvy}

Jako \underline{Výzvu} se hodí uchopit situaci, která by na první pohled volala po akcích \texttt{Překonání}, ale je mnohem komplikovanější. Nejedná se prostě o vypáčení zámku: je taky potřeba to stihnout rychle, protože strop se hroutí, někdo musí držet jiné dveře, aby jimi do místnosti nevpadli další nepřátelé a taky by se hodilo opravit tu díru, skrz kterou do místnosti někdo střílí...

\underline{Výzva} je efektivně série hodů na \texttt{Překonání} pomocí \textit{různých} dovedností \footnote{V případě, že by se jednalo o hody na tutéž dovednost, je lepší tuto situaci vyřešit jediným hodem.}, které je třeba provést, aby se komplexní situace vyřešila. Vypravěč popíše situaci a stanoví, na jaké dovednosti je potřeba si hodit. V závislosti na konkrétní situaci pak postava buď provádí všechny hody (když je např. ve scéně samotná), nebo se \underline{Výzvy} účastní více postav a každá hází na jinou dovednost. Poté, co jsou provedeny všechny hody, výsledek \underline{Výzvy} se vyhodnotí a hra se podle toho vyvine - je možné, že bude následovat další \underline{Výzva} (pokud se překážka přetvořila takovým způsobem, že je třeba ji překonat jinými dovednostmi), může dojít ke \underline{Střetu} (za zmiňovanými dveřmi bude hlavní záporák a ten se bude snažit odteleportovat se pryč) či dojde k vyhrocení a nastane \underline{Konflikt} (nepodaří se dveře vypáčit dřív, než do místnosti vtrhnou nepřátelé).

Vyhodnocení \underline{Výzvy} samozřejmě závisí na výsledcích hodů na stanovené dovednosti. Může se stát, že i přes neúspěch na jednu dovednost se podaří výzvu překonat (když jej například kompenzuje vysoký úspěch na jinou dovednost).

Jako \underline{Výzvu} se hodí uchopit situaci, která by na první pohled volala po akcích \texttt{Překonání}, ale je mnohem komplikovanější. Nejedná se prostě o vypáčení zámku: je taky potřeba to stihnout rychle, protože strop se hroutí, někdo musí držet jiné dveře, aby jimi do místnosti nevpadli další nepřátelé a taky by se hodilo opravit tu díru, skrz kterou do místnosti někdo střílí.

\underline{Výzva} je efektivně série hodů na \texttt{Překonání} pomocí \textit{různých} dovedností \footnote{V případě, že by se jednalo o hody na tutéž dovednost, je lepší tuto situaci vyřešit jediným hodem.}, které je třeba provést, aby se komplexní situace vyřešila. Vypravěč popíše situaci a stanoví, na jaké dovednosti je potřeba si hodit. V závislosti na konkrétní situaci pak postava buď provádí všechny hody (když je např. ve scéně samotná), nebo se \underline{Výzvy} účastní více postav a každá hází na jinou dovednost. Poté, co jsou provedeny všechny hody, výsledek \underline{Výzvy} se vyhodnotí a hra se podle toho vyvine - je možné, že bude následovat další \underline{Výzva} (pokud se překážka přetvořila takovým způsobem, že je třeba ji překonat jinými dovednostmi), může dojít ke \underline{Střetu} (za zmiňovanými dveřmi bude hlavní záporák a ten se bude snažit odteleportovat se pryč) či dojde k vyhrocení a nastane \underline{Konflikt} (nepodaří se dveře vypáčit dřív, než do místnosti vtrhnou nepřátelé).

\subsection{Výhody ve výzvách}
\label{sec:výhody-výzvy}

Hraní akce \texttt{Vytváření výhody} ve \underline{Výzvách} je perfektně možné - prostě si hodíte na relevantní dovednost a podle výsledku pak můžete výhodu čerpat. Samozřejmě, vytvoření výhody se nepočítá do hodů na \texttt{Překonání} potřebných pro odehrání \underline{Výzvy}. A hlavně - v případě, že se výhodu nepodaří vytvořit, ovlivní to negativním způsobem nějaký jiný cíl - např. zvýší jeho obtížnost, poskytne volné vyvolání nepříteli apod.

\subsection{Příklad výzvy}
\label{sec:příklad-výzvy}

Příklad \underline{Výzvy} přebíráme z Fate Core, protože je prostě dobrý:

Tajemný Zird se pokouší dokončit Quirický zasvěcovací rituál, aby posvětil půdu zájezdního hostince a dodal mu tak ochranu Quirických božstev. Normálně by to nebylo moc zajímavé, jenomže se to pokouší dokončit dříve, než do hostince vtrhne horda bezduchých, masa-chtivých zombií, které omylem osvobodil v dřívější části příběhu.\\
Amanda ve scéně vidí několik různých prvků. Zaprvé tu je samotný rituál, zadruhé potřeba udržet hostinec zabarikádovaný a nakonec tu je potřeba udržet v klidu panikařící návštěvníky lokálu. To vyžaduje \textit{Učenost, Řemesla}a některou ze sociálních dovedností – Ryan se okamžitě rozhodne pro \textit{Vztahy}. Tudíž si Ryan musí hodit zvlášť na tři různé dovednosti, jednou pro každý prvek, který Amanda ve scéně zdůraznila. Pro každou z nich mu dá Dobrou (+3) opozici – chce mu dát fér šanci, ale zároveň chce nechat prostor pro různé možné výsledky.\\
S tím jsou připraveni začít. Ryan se z hluboka nadechne a prohlásí: „Ok, tak do toho.“, s čímž vezme kostky do ruky. Nejprve se rozhodne zajistit bezpečnost lokálu, takže si hodí na svá Dobrá (+3) \textit{Řemesla} a na kostkách mu padne 0. To znamená remízu, což mu umožní uspět za drobnou cenu. Amanda na to řekne: „Tak řekněme, že proti tobě dostanu posílení \asp{Zbrklá práce}, pokud budu potřebovat. Nakonec pracuješ dost ve spěchu. Ryan si povzdechne a přikývne a pak se pustí do druhé části výzvy, což je uklidnění hostů v lokále svými Dobrými (+3) \textit{Vztahy}. Hodí si a padne mu příšerných -3! Teď má možnost selhat nebo uspět za výraznou cenu. Rozhodne se uspět a nechá na Amandě, aby nějakou dobrou výraznou cenu vymyslela. Amanda o tom chvíli přemýšlí. Jakou cenu chtít za uklidňování vesničanů? Pak se ušklíbne. „Takže, je to sice víc o příběhu než o mechanikách ale víš co... používáš Vztahy, takže nejspíš všechny dost inspiruješ a motivuješ. Myslím, že jsi omylem přesvědčil pár farmářů a pasáků, že zombie ve skutečnosti nejsou hrozba a že můžou klidně jít ven a bojovat s nimi bez větších důsledků. Protože tvá magie je samozřejmě ochrání, že?“ „Ale aby to fungovalo, tak musí zůstat v hostinci!“ odvětí Ryan. Amanda se jen ušklíbne a Ryan si znovu povzdechne. „Ok, fajn. Pár lidí asi dostane naprosto zcestný nápad a nejspíš se půjdou nechat zabít. Už teď to slyším... Zirde, proč jsi nechal mého manžela zemřít? Ach jo.“ Amanda se ušklíbne ještě o něco víc. Ryan se pustí do poslední části výzvy – samotného rituálu, který bude sesílat svou Skvělou (+4) \textit{Učeností}. Amanda vyvolá posílení, které získala dříve, a prohlásí „Jo, rozhodně tě docela rozptylují ty zombie co se prolamují tou barikádou, kterou jsi jim předtím postavil do cesty. Hodně rozptylují.“ To posune obtížnost závěrečného hodu na Vynikající (+5). Ryan hodí +2 a dostane Fantastický (+6) výsledek, což mu stačí na to, aby uspěl bez ceny. Amanda přikývne a společně dokončí popis scény – Zird dokončuje rituál jen tak tak na čas a svatá moc Qiriku sestoupí na hostinec. Zombie které se zrovna sápaly průlomem jsou zasaženy svatou aurou a Zird si v tu chvíli velmi oddechne... dokud nezaslechne panický křik vesničanů před hostincem.\\
To ale bude až další scéna.


\section{Střety}
\label{sec:strety}

Kdykoliv se dvě či více postav dostanou do peripetie, ve které má každý jiný zájem (a nelze to, opět, vyřešit jediným hodem na \texttt{Překonání} s aktivní opozicí) a přitom se nesnaží si způsobit zranění, je na místě situaci odehrát jako \underline{Střet}. Sem spadají všechny honičky, snahy získat si přízeň dříve než můj sok, turnaje, veřejné debaty.

Do střetu se zabrušuje zhruba následovně:
\begin{enumerate}
\item Stanoví se, jaké strany proti sobě stojí - jsou to dvě postavy, dvě družiny, nebo družina proti postavě? V případě, kdy proti sobě stojí dvě skupiny, je výhodné použít pravidla týmového snažení.
\item Každá ze stran stanoví svůj záměr a s ním i dovednost, na kterou si bude házet; jedná se vlastně o hody na akci \texttt{Překonání}. Každá ze stran (dokonce každá z postav) může samozřejmě házet na jinou dovednost - v rámci debaty někdo může používat \textit{Provokaci, Klam, Vztahy} nebo třeba \textit{Kontakty}. 

\item Vypravěč stanoví, kolik úspěchů je potřeba získát k vítězství ve \underline{Střetu} - dobrý standard jsou tři. Úspěch znamená větší hod než protivník (v případě aktivní opozice, např. u sportovního turnaje) nebo největší úspěch proti pasivní opozici (např. ve snaze přesvědčit třetí stranu). Jestliže postava (strana) jako jediná uspěla se stylem, připisuje si dokonce dva úspěchy.
\item V případě remízy se stane něco nečekaného - např. se změní prostředí (resp. jeho aspekty), do \underline{Střetu} vstoupí další postava...
\item Strana, která získá jako první potřebný počet úspěchů, vítězí a \underline{Střet} končí.
\end{enumerate}

Po konci \underline{Střetu} se může stát mnoho věcí: další \underline{Střet} (ovšem s jiným cílem), objevit se nová \underline{Výzva}, nebo dokonce \underline{Konflikt} (družiníkovi se podařilo odteleportovat dříve, než zbytek družiny přiběhl a ponechal je tak mezi nepřáteli).

\subsection{Výhody ve střetech}
\label{sec:výhody-střety}

Podobně jako ve \underline{Výzvách}, i ve \underline{Střetech} lze hrát akce \texttt{Vytváření výhody}. Ještě předtím, než si hodíte na svoji akci \texttt{Překonání} k získání potřebného úspěchu, lze se pokusit vytvořit výhodu pomocí klasických pravidel. V případě úspěchu (se stylem) nebo remízy můžete výhody využít hned v onom hodu na \texttt{Překonání}; jestliže ale neuspějete, \textit{propadá váš hod na \texttt{Překonání}} \footnote{Jestliže vytváření výhody probíhá pomocí týmového snažení a proběhne neúspěšně, nemůže vedoucí postava výhodu použít (protože nevznikla), ale může se stále pokoušet získat úspěch.} - a ztrácíte tak možnost ovlivnit, kdo vyhraje tuto výměnu.

\subsection{Příklad střetu}
\label{sec:příklad-střet}

Tajemný Zird byl poražen v bitvě se záludnou skupinou vrahů, která jeho a Cynere přepadla kousíček před městem! Cynere ukončí konflikt tím, že zařízne posledního z nich a vrhne se ke svému padlému příteli. \\
Zrovna v ten moment se ale vůdce zabijáků, chmaták Mrštný Teran, kterého velmi dobře zná, za pomoci teleportační magie zjeví přímo vedle Zirdova bezvědomého těla. Ihned začne sesílat další teleportační kouzlo, očividně s úmyslem zmizet i se Zirdem. Cynere se rozeběhne. Zvládne se k nim dostat dřív, než Teran dokončí své kouzlo?\\
Amanda se rozhodne situaci řešit jako \underline{Střet}. Předchozí scéna konfliktu měla situační aspekt \asp{Blátivý terén}, který se rozhodne ve hře ponechat.\\
Je zřejmé, že Teran a Cynere jednají proti sobě navzájem, takže si budou navzájem poskytovat aktivní opozici. Teran si bude ve střetu házet svou dovedností \textit{Učenost}, protože sesílá kouzlo. Protože se z pohledu Cynere jedná o situaci vyžadující pouze rychlý pohyb, Amanda a Lily se shodnou, že nejvhodnější dovednost k hodu pro ní bude \textit{Atletika}.
Cynere má Skvělou (+4) \textit{Atletiku}. Teran má Dobrou (+3) \textit{Učenost}.\\
V první výměně to Lily za Cynere padne špatně a skončí jen na Průměru (+1). Amandě na kostkách padne 0 a zůstane na Dobrém (+3) výsledku. Amanda má víc, takže Teran vyhrává výměnu a získává 1 vítězství. Amanda popíše, jak Teran dokončuje první významnou runu rituálu a ve vzduchu se objevuje plápolavá zelená záře.\\
V druhé výměně se pro Lily věci obrátí, hodí si velmi dobře a získá Vynikající (+5) výsledek, zatímco Amanda se za Terana dostane jen na Slušný (+2). To je pro Lily úspěch se stykem, takže si zapíše dvě vítězství a převezme tak vedení. Lily popíše, jak se naplno rozeběhla Teranovi vstříc.\\
Ve třetí výměně dojde k remíze na Dobrém (+3) výsledku! Amanda teď musí ve střetu představit nějaký nečekaný zvrat. Krátce se nad tím zamyslí a pak prohlásí: „Fajn, takže to vypadá, že nějaké magické ingredience z váčku na Zirdově opasku nějak podivně zareagovaly s Teranovým kouzlem, což okolí zaplnilo \asp{Magickými výboji}.“ Tento situační aspekt napíše na kartičku a položí na stůl. \\
Ve čtvrté výměně dojde k další remíze, tentokrát na Skvělém (+4) výsledku. Lily ale prohlásí: „Zapomeň na to. Chci vyvolat dva aspekty. Zaprvé \asp{Kreju Zirdovi záda} z mojí karty a zadruhé \asp{Magické výboje}, protože si jsem jistá že ty interagují mnohem víc se sesíláním kouzla než s běháním.“ S tím Amandě předá dva body osudu.\\
To posune její celkový výsledek na Legendární (+8), což je další úspěch se stylem a další dvě vítězství. To jí dává čtyři vítězství na jedno Teranovo, čímž vyhrává výměnu i celý střet!\\
Amanda a Lily popíší, jak chytí Zirda těsně před tím, než Teran dokončí své kouzlo a odteleportuje se pryč bez své vytoužené kořisti.

\section{Konflikty}

\section{Fyzické konflikty}
\label{sec:boj}

Poslední peripetií, kterou zde pokryjeme, jsou \underline{Konflikty}. Uvidíme, že oproti \underline{Výzvám} a \underline{Střetům} jsou komplikovanější a že jsou pro postavy mnohem nebezpečnější (ve velmi přímém smyslu). Místo toho, abychom jen podali přežvýkání pravidel Fate Core, u \underline{Konfliktů} se zastavíme o trochu déle a některé záležitosti rozebereme více do podrobna. Důvodem je zejména v úvodu \ref{chap:introduction} zmiňovaný zájem družiny o taktičtější hraní konfliktů; kromě toho nabídneme řešení na některé z typických situací, co se během konfliktu běžně vyskytují: povalení, nečekaný útok, plošné útoky, více akcí během jednoho kola a další.

Na začátek zmiňme několik obecných pravidel, co průběh \underline{Konfliktů} řídí. Konkrétní realizaci těchto pravidel v různých typech konfliktů (fyzický, duševní, magický) pak rozebereme více dopodrobna v dalších sekcích.

\subsection{Obecná pravidla konfliktů}
\label{sec:obecna-pravidla}

\subsubsection{Vsuvka: Střeto-konflikty a výzvy v konfliktech}
\label{sec:hybrid}

Jako \underline{Konflikt} myslíme peripetii, ve které je hlavním cílem postav někoho zranit - fyzicky, duševně, nebo třeba majetkově. To je velký rozdíl oproti \underline{Výzvám} a \underline{Střetům}, ve kterých je hlavním cílem vždy něco jiného - vyřešit situaci nebo překonat protinvíka. Pokud se ve \underline{Výzvě} či \underline{Střetu} stane, že se postavy uchýlí k pokusu se zranit, vždy by se mělo přejít ke \underline{Konfliktu} a jinou peripetii přerušit.\\
Jistě se ale vyskytnou situace ``tak někde mezi'': kupříkladu během \underline{Střetu} se postava rozhodne uspět za velkou cenu (tedy v situaci, kdyby běžně neuspěla) a vypravěč se rozhodne postavě uštědřit následek (např. je na střeše a snaží se dohonit nepřítele na ulici, tak ze střechy seskočí a zvrtne si u toho kotník) - což striktně vzato v pravidlech \underline{Střetu} není dobře možné.\\
Naopak, snadno se uvnitř \underline{Konfliktu} vynoří jiná peripetie: bojuje se v komplexním prostředí a postava zabijáka se nejprve rozhodne zcela zmizet (hod na \textit{Skrývání}), přehopsat přes střechu na protější budovu (\textit{Atletika}), tam nalíčit past (\textit{Řemesla}) a nakonec do ní nalákat nic netušícího protivníka (\textit{Provokace}) - čímž efektivně plní \underline{Výzvu}.\\
Takovéto případy nejsou úplně ojedinělé a tak by se mohlo zdát, že tato pravidla jsou prakticky bezzubá kdykoliv nelze akci snadno zaškatulkovat do jedné z peripetií - což nelze často (a tedy že pravidla jsou bezzubá skoro vždy). To autor nehodlá přiznat a argumentuje prvním a nejdůležitějším pravidlem tohoto textu - \textbf{Zlaté pravidlo: nejdřív vyprávění, pak až pravidla}.\footnote{Pro hatery nekonečných smyček: aplikace Zlatého pravidla má výjimku...} Tedy, jestliže dává smysl, že postava ve \underline{Střetu} obdrží následek i když se nejedná o konflikt, prostě to tak zahrajte. Pokud je součástí konfliktu něco, co se tváří jako \underline{Výzva}, klidně z toho \underline{Výzvu} udělejte - a jedna z dovedností, na kterou si postava bude házet, může být \textit{Boj} nebo \textit{Mobilita}.\\

Dělení na peripetie je obzvlášť výhodné, když nejsou pochybnosti, o jaké snažení se jedná; v těchto chvílích poskytuje jednoduchá univerzální pravidla, jak takové situace hladce řešit. Pokud není jednoduché rozlišit, o jakou peripetii jde, prostě si udělejte vlastní kategorii.



Zpět k pravidlům konfliktu. Ještě než se začne zběsile házet kostkami, je dobré připravit prostředit a vykreslit scénu:

\begin{itemize}
\item Jaké postavy se ve scéně vyskytují? Existují nějaké strany?
\item Jaké aspekty má scéna? Je \asp{Všude hromada haraburdí} nebo se bojuje \asp{Uprostřed ničeho}?
\item Jaké zóny se ve scéně vyskytují a jak se mezi nimi dá přesouvat?
\end{itemize}

Jednotlivé body nyní rozeberme podrobněji.

\subsubsection{Určení stran}
\label{sec:urcenistran}
Tohle se týká spíše nehráčských postav - postav vypravěče. Bývá výhodné například nějaké postavy shluknout a považovat je mechanicky za jednu postavu (se vším všudy, tedy i s motivací), či družině přiřadit nějakého pomocníka jakožto volných vyvolání aspektů \asp{Posila}.

Taky je dobré si rozmyslet, čeho chtějí postavy v konfliktu dosáhnout. Pobít všechny? Dostat se přes nepřátele k jejich šéfovi? Zastavit plán úhlavního nepřítele? Konflikty, jejímž hlavním cílem není všechny zabít (resp. v každé výměně zabít svého protivníka), se ukazují jako zábavnější. Umožňují lépe pracovat s následky (jež budou brzy představeny), odstoupením a vracejícimi se postavami. Bylo by nuda, kdyby jediný konflikt s hlavním záporákem nastal až na konci scénáře; na druhou stranu by bylo ještě nudnější, kdyby při prvním konfliktu zemřel...

\subsubsection{Aspekty scény}
\label{sec:aspektysceny}

Vybarvení scény pomocí aspektů skvěle navodí atmosféru konfliktu: jedná se o špinavou bitku \asp{V chudinské čtvrti}, nebo dramatický duel \asp{Na palubě lodi}? Taky výborně vyzdvihují, co je ve scéně podstatného - pokud se celé dobrodružství odehrává na palubě lodi, není zajímavé z tohoto dělat aspekt scény; naopak, pokud ke konfliktu dojde při přistávání, aspekt \asp{Na pevné zemi} je najednou podstatný (byť přítomnost země pod nohama není běžně nijak podstatná).\\
Nejde jen o navození atmosféry a vyzdvižení důležitých herních prvků. Aspekty scény hrají velmi důležitou roli při vyvolávání a vynucování; umožňují i postavám bez aspektů vhodných k vyvolání během konfliktu být užitečné. Je proto dobré, když před začátkem každého aspoň trochu významného konfliktu hráči společně s vypravěčem obdaří scénu několika aspekty.

\subsubsection{Zóny a pohyb}
\label{sec:zónyapohyb}

Protože není úplně zábavné počítat, kolik sáhů \footnote{OG název pro metr.} která postava za kolo dokáže překonat, zavádí se pojem zóny. Zóna vlastně představuje nejmenší jednotku vzdálenosti - nijak podrobněji se už nerozlišuje, kde přesně v zóně postava stojí. Platí tedy, že postavy v zóně mohou libovlně interagovat (bojovat spolu, předávat si vybavení apod.).\\
Jak velká zóna není přesně definované. To je dobře, protože velmi záleží, jaké prostředí chcete na zóny dělit; jedna zóna tak může být jedna místnost, její část, nebo třeba celé patro či budova. Při stanovování je dobré zvážit, jak moc je konflikt lokalizovaný: pokud veškerá akce probíhá v jedné místnosti, nabízí se místnost rozdělit na několik zón; pokud konflikt probíhá například v jednom domě, každá místnost může být jedna zóna. Doporučený počet zón je 2 - 4 podle počtu postavu a rozsahu konfliktu.\\
Pokud není důvod předpokládat něco jiného, není problém se mezi zónami přemisťovat. Pro větší plynulost hry volíme, že přemístění do vedlejší zóny probíhá v nulovém čase - tedy postava může například bojovat a k tomu se přemisťovat. V situaci, kdy se postava chce přemístit o více než jednu zónu či jí v přesunu něco brání, vyžádá si tato snaha hod na akci \texttt{Překonání}, což musí provést ve svém tahu (tj. zabere jí to celé kolo). Jakou opozici musí překonat pak závisí na povaze situace: jestliže se například postava snaží dostat do vedlejší zóny, ale není to samozřejmé, neboť ta je \asp{Dva metry pod vodou}, dává smysl stanovit pasivní opozici a házet na \textit{Mobilitu}. Podobně, chce-li postava překonat velké množství zón rychle v jednom kole, dobré je stanovit pasivní opozici hodu na například +1/+2 za přesun do každé zóny oproti té vedlejší.\\
Může se též stát, že přesunu do jiné zóny brání jiná postava (viz akce \texttt{Obrana} a \texttt{Překonání}) - v takovém případě je opozice aktivní.

\subsubsection{Průběh kola}
\label{sec:prubehkola}

Každé kolo probíhá podle stejné šablony:

\begin{enumerate}
\item Stanoví se pořadí, ve kterém budou postavy jednat (tzv. iniciativa).
\item Ve svém tahu postavy hrají akce \texttt{Útok}, \texttt{Vytvoření výhody}, \texttt{Překonání} nebo zahlásí \texttt{Plnou obranu}, hází kostkami, přičítají dovednosti a případně vyvolávají/vynucují aspekty. Postava smí zahrát právě jednu akci a k tomu se pohnout do vedlejší zóny (pokud jí v tom něco nebrání).
\item Mimo svůj tah postavy hrají akce \texttt{Obrana} (reagují na akce postavy, co je na tahu) případně tvoří aktivní opozici hodům na \texttt{Překonání}. Teoreticky může postava zahrát libovolné množství akcí \texttt{Obrana} - tj. bránit se každé nepřátelské akci.
\end{enumerate}

Ve chvíli, kdy postava ve svém tahu z konfliktu odstoupí, či je z konfliktu vyřazena, už se jej nemůže dále účastnit. Stejně tak, pokud se postava dostane na řadu až poté, co ji někdo způsobil následek, může tento aspekt někdo vyvolat, aby si pomohl při obraně proti akci, jež se zraněná postava snaží vykonat. Tím se prostě myslí, že akce probíhají ve stanoveném pořadí - a nikoli ``naráz'', jako tomu bylo v pravidlech M16.

\subsubsection{Iniciativa}
\label{sec:iniciativa}

Každé kolo začíná hráč s nejvyšší iniciativou. Co přesně je iniciativa záleží na povaze konfliktu:

\begin{itemize}
\item ve fyzickém konfliktu je iniciativa rovna \textit{Pozornosti}, v případě remízy pak \textit{Mobilitě} a případně \textit{Kondici}
\item v duševním konfliktu je iniciativa rovna \textit{Empatii}, v případě remízy \textit{Vztahům} a \textit{Vůli}
\item v magickém konfliktu JEŠTĚ NEVÍM
\end{itemize}

Postavy pak přicházejí na řadu podle úrovně svých dovedností (nehází se!) a v případě, že nějaké postavy mají patřičnou dovednost na stejné úrovni, rozhoduje se podle úrovně další zmíněné dovednosti.\\
Je do jisté míry otázkou, nakolik je výhodné začínat hrát jako první - dá se argumentovat, že hráči na tahu později pak mohou na tah nejiniciativnějšího hráče reagovat. Na druhou stranu, být poslední na tahu vždy znamená, že vaši rozmyšlenou akci vám někdo překazí prostě tím, že změní kontext konfliktu a vy ji musíte celou vymyslet znovu. K tomu taky postava riskuje, že se na tah ani nedostane, protože bude předtím vyřazena...

\subsection{Stres a následky}
\label{sec:stres-nasledky}

Konflikty jsou o zraňování. Ukazateli, kolik toho postava vydrží, jsou právě stresy a následky.\\
Po neúspěšné akci \texttt{Obrana} proti nepřátelské akci \texttt{Útok} se musí postava s obdrženým zraněním nějak popasovat. Konkrétněji, musí pomocí svých políček stresů a následků pokrýt (vyrovnat) celkový rozdíl hodů na \texttt{Útok} a \texttt{Obranu} - tzv. posun. Pokud postava není schopna posun pokrýt (či jej nechce pokrýt), je z konfliktu vyřazena; vyřazení \textbf{není} totéž co smrt - viz další sekce.\\
Na pokrytí posunu po každé neúspěšné obraně může postava využít nejvýše jedno políčko stresu a libovolný počet následků; k tomu platí, že na pokrytí fyzického zranění lze použít pouze políčka fyzického stresu a vice versa pro duševní zranění a stres. Políčka následků jsou sdílené jak pro fyzické, tak i duševní zranění.\\
Každá postava začíná hru se dvěma políčkami fyzického stresu a se dvěma políčkami duševního stresu; v obou případech je hodnota políček +1 a +2. Kromě stresu má postava k dispozici tři políčka následků: +2 (drobný), +4 (mírný), +6 (vážný) a jeden tzv. extrémní následek o hodnotě +8 (pro extrémní následek platí speciální pravidlo, viz \ref{sec:následky}). Uváděné hodnoty představují, kolik \textbf{nejvýše} posunů je schopno políčko pokrýt. V případě, kdy postava neuspěje na obranu, musí zaškrtnout taková políčka, aby celkový součet jejích hodnot byl roven nebo větší posunu. Může se tedy stát, že v pokročilé fázi konfliktu bude postava nemajíc již žádná políčka stresu nucena zaškrtnout vážný (+6) následek - tomu hovorově říkáme \textbf{Spirála smrti}.\\

\subsubsection{Stres}
\label{sec:stres}

Stres není v pravém slova smyslu zranění - nezpůsobuje postavě žádné přímé postihy. Je lepší jej proto interpretovat jako ``schopnost vyhnout se následkům''; proto se taky po skončení konfliktu, kdy mají postavy chvilku na vydýchání, všechna zaškrtnutá políčka stresu obnoví.\\
Postavy s vyššími hodnotami dovedností \textit{Kondice} a \textit{Vůle} toho vydrží ve fyzických resp. duševních konfliktech o něco více. Konkrétně, mají-li \textit{Kondici/Vůli} na úrovni alespoň +2, získávají jedno políčko fyzického/dušvního stresu +2 navíc, jestliže mají dovednost na úrovni alespoň +3, dostávají jedno políčko fyzického/duševního stresu +4 navíc a pokud mají dokonce dovednost na úrovni +5, získávají jeden drobný (+2) následek navíc.

\subsubsection{Následky}
\label{sec:následky}

Následky narozdíl od stresu zranění představují. A ještě o něco více - zranění podstatným způsobem ovlivňují realitu postižené postavy - a tato skutečnost je reprezentována aspektem. Tedy, následky jsou něco jako ``negativní aspekty'' postavy\footnote{Zde jen připomeňme, že v pravém slova smyslu negativní aspekty neexistují, viz sekce \ref{chap:aspekty}. Je ovšem třeba značného rétora, aby pro svůj prospěch využil, že je \asp{Téměř rozčtvrcen.}}, neboli postava s následkem je snadným terčem pro vyvolání tohoto aspektu. Zásadní je, že postava (resp. strana), která postavě následek uštědří, na něj získává volné vyvolání. To je analogie postihu za Otřes/Lehké zranění/Těžké zranění z pravidel M16; zde následek neposkytuje konstantní postih, ale spíše jakousi dynamickou výhodu prezentovanou právě oním volným vyvoláním.\\
Protože následky jsou aspekty postavy, každý bod osudu za jejich vynucení putuje do rukou zraněného. Nicméně, ten své body osudu může používat až po skončení konfliktu (jinak by nebylo způsobování následků moc výhodné). Nabízí se otázka, proč používat na vynucení právě následky, když tím de facto podporuji svého nepřítele: stejně ale jako v případě každého jiného vynucení, může se často stát, že postava nemá (resp. scéna nemá) jiné vhodné aspekty k vyvolání - typicky nebojová postava nemá moc možností, jak využívat své aspekty v konfliktu, ale díky následkům (a bodům osudu) může být stále relevantní.\\
Každá postava může zaškrtnout libovolné množství (volných) následků, aby pokryla potřebný posun. Způsob, jakým se postava z následků zotavuje, je popsán v sekci \ref{sec:zotavovanise}.\\
Zmiňme zde výjimku extrémního (+8) následku. Ten je dost často poslední šancí dobrodruha, neboť umožňuje pokrýt obrovský zásah. Není to ale zadarmo; tak významné zranění, jako je extrémní následek, doslova mění, kým postava je - a dobrodruh si musí tímto následkem nahradit jeden ze svých aspektů postavy.


\subsubsection{Vyřazení a smrt}
\label{sec:vyrazeniasmrt}

Zbývá zmínit co se stane, když postava není schopna pokrýt posuny. K tomu může dojít například\footnote{Poznamenejme, že možnost tzv. oneshotu je vlastně dost nepravděpodobná - za předpokladu, že postava vstupuje do konfliktu zcela zdravá, pak by musela obdržet zásah v hodnotě 2 + 2 + 4 +6 + 8 + 1 = 23 posunů. Ilustrujme na příkladu nebojového dobrodruha, co se brání \textit{Mobilitou} na úrovni +0 proti živoucí asterionské legendě s \textit{Bojem} na +7. Uvažme navíc, že dobrodruh měl smůl a hodil -4, kdežto legenda měla štěstí a hodila +4, a taky že dobrodruh zbroj nemá a legenda má zbraň hodnou jeho postavení s hodnocením +4. Aby v takovémto případě legenda oneshotnula průměrného dobrodruha, musela by stále použít dva aspekty. } v delším souboji se silnými protivníky, kdy již postava přijmula vícero následků.\\ 
Odpovězme: ve chvíli, kdy postava není schopna posuny pokrýt, je vyřazena. Vyřazená postava již do konfliktu nemůže nijak zasahovat a ovlivňovat jeho průběh. K tomu nemá moc jistoty, jak bude s jeho postavou naloženo - může být okradena, odvlečena atp.; zkrátka, jakýkoliv záměr útočník měl, povedl se mu. \\
Vyřazení v boji neznamená vždy smrt. Jak jsme zmiňovali výše, smyslem konfliktu vždy není někoho zabít (zranit jistě ano, jinak by se nejednalo o konflikt); konflikt může nastat po \underline{Střetu}, ve kterém poražený neudrží emoce - a může stačit ho zranit, ``aby se uklidnil''. Jestliže je v konfliktu nějaká \underline{Výzva} (postavy se snaží vyřadit pekelný stroj hl. záporáka, který je zaměstnává svými pohunky), dá se předpokládat, že záměrem družiny bude ji splnit (a záměrem protivníků jí v tom zamezit). \textbf{Zkrátka a dobře, vyřazení v konfliktu neznamená vždy smrt postavy.}\\
Stojí za zamyšlení, nakolik by postavy vlastně měly umírat - tedy, nakolik je zajímavé a zábavné, aby postavy umíraly. Jistě, nikdo netvrdí, že postavy by měly být neporazitelné a chovat se, jako kdyby byly nesmrtelné - ale vzhledem k úsilí, které hráči do vytváření svých postav vložili, nebo vzhledem k nejednoduchosti zapojit novou postavu do družiny (viz Trio fází \ref{chap:postup}), je smrt spíš otrava. Navíc, těžké zranění/blízké setkání se smrtí může být podstatná příležitost pro vývoj postavy - slovy Terezy Matějčkové: ``Co nás nezabije, to nás přizabije a trochu posílí.'' Například bude kněz, který se v následování svého boha dostal do situace, kdy sám málem zemřel (či jeho spoludružiník), stále bezmezně důvěřovat ve svého Pána? A nejen to - téměř vyřazení může představovat i nějaké trvalé zranění (viz extrémní následek), jako je chybějící končetina, oko atd.; to je přesně něco, co vyžaduje inovaci ve hraní, a tudíž je zajímavé.

\subsubsection{Odstoupení}
\label{sec:odstoupeni}

Než hráč vyškrtá všechny následky a protočí všechny body osudu aby zjistil, že tenhle konflikt stejně nevyhraje, může se rozhodnout z něj odstoupit. To ale musí zahlásit v čas - před hodem kostek hráče, jehož akci se chce vyhnout (a též akcím všech hráčů, kteří hrají po něm). Jinými slovy, pokud se na řadu dostane hráč, který vám již způsobil dost zranění a vy máte tušení, že v tom bude pokračovat (resp. si můžete počkat na to, až zahlásí svůj záměr), můžete se rozhodnout odstoupit (i mimo svůj tah); nelze se ale rozhodnout odstoupit až na základě výsledku hodu.\\
Odstoupení dá hráči něco málo bolestného: jeden bod osudu v základu a k tomu jeden za každý utržený (v onom konfliktu) následek. Hlavně ale hráč získá částečnou vypravěčskou pravomoc ohledně jeho dalšího osudu - nemůže se stát, že útočník bude mít s odstoupeným volnou ruku, do jisté míry se jedná o domluvu. Tedy, mohou se například dohodnout, že přijde o nějakou část vybavení, získá nějakou příběhovou výhodu, odhalí nějaký jeho aspekt atd. \\
Odstoupení je tak vlastně docela dobrý způsob, jak z těžkého konfliktu odejít jakž takž dobře - s nějakými body osudu a jistotou, že se vám nestane to nejhorší. 

\subsection{Příklad jedné výměny}
\label{sec:priklad-vymena}

Pro lepší představu dynamiky zde uvedeme, jak může vypadat typická výměna a jakým způsobem se pracuje s stresem a následky.

\section{Duševní a sociální konflikty}
\label{sec:dusevni-soc-konflikty}

\section{Magické konflikty}
\label{sec:magicke-konflikty}

\section{Zotavování se z následků}
\label{sec:zotavovanise}



%%% Local Variables:
%%% mode: LaTeX
%%% TeX-master: "../main"
%%% End:
