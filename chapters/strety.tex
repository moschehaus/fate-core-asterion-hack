\documentclass[../main.tex]{subfiles}
\graphicspath{{\subfix{../graphics/}}}
\begin{document}

Dalším rozdílem oproti pravidlům M16 je klasifikace nětkerých specifických usílích jako zvláštní typ akce. V našem slovníku jim budeme říkat \underline{Peripetie}. Než se pustíme do prezentace těchto typů, je dobré se uvědomit, že ve většině případů stačí jediný hod na dovednost (nebo dokonce žádný), abychom rozhodli, jak ono snažení dopadne. Nemusí přitom nutně záležet na tom, jak dlouho úsilí trvá - překonat skalnatou roklinu, i když se jedná o náročnou celodenní tůru, může být pořád jeden hod na \textit{Mobilitu}, projití Džungle padlých stromů, seč anabáze na několik aldenů, může stále být jediný hod na \textit{Kondici}.

Existují ale situace, které chceme odehrát ve větších podrobnostech, i kdyby se jednalo jen o několik minut herního času; typicky jsou to akce, jejichž výsledek není samozřejmý, (ne)úspěch by byl zajímavý nebo akce prostě nejsou natolik přímočaré, aby se daly vyřešit jediným hodem. Jaké úsilí nemá smysl řešit vůbec, jaké bez hodu, jaké s jedním hodem a jaké jako peripetii není dáno ultimátně. Dobře slouží přirovnání hry s knihou nebo seriálem; ve většině případů není zajímavé sledovat, jak postava jí nebo jde spát (a navíc to zvládá bez potíží). Může se ale stát, že jídlo je otrávené, v posteli je nastrčený jedovatý had, nebo za dveřmi čeká zabiják - v takovém případě je na místě situaci s jezením vyřešit (např.) jedním hodem na \textit{Pozornost} a souboj se zabijákem jako konflikt. Na druhou stranu, pokud je celé dobrodružství protkané zabijáky, co na vás číhají na každém rohu, není třeba každé střetnutí řešit konfliktem - tentokrát zde třeba postačí jediný hod.

V těchto pravidlech budeme rozlišovat základní tři typy peripetií:

\begin{itemize}
\item \underline{Výzva} je o překonání dynamických komplikovaných překážek či vyřešení prekérních situací, na které nestačí jediný hod na \texttt{Překonání}
\item \underline{Střet} je neshoda dvou či více postav ohledně téhož zájmu, ale s různým cílem, které je strukturovanější než jediný hod na \texttt{Překonání}
\item \underline{Konflikt} je snaha postav si ublížit - fyzicky, duševně, magicky, peněžně atd.
\end{itemize}

\section{Výzvy}
\label{sec:vyzvy}

Jako \underline{Výzvu} se hodí uchopit situaci, která by na první pohled volala po akcích \texttt{Překonání}, ale je mnohem komplikovanější. Nejedná se prostě o vypáčení zámku: je taky potřeba to stihnout rychle, protože strop se hroutí, někdo musí držet jiné dveře, aby jimi do místnosti nevpadli další nepřátelé a taky by se hodilo opravit tu díru, skrz kterou do místnosti někdo střílí.

\underline{Výzva} je efektivně série hodů na \texttt{Překonání} pomocí \textit{různých} dovedností \footnote{V případě, že by se jednalo o hody na tutéž dovednost, je lepší tuto situaci vyřešit jediným hodem.}, které je třeba provést, aby se komplexní situace vyřešila. Vypravěč popíše situaci a stanoví, na jaké dovednosti je potřeba si hodit. V závislosti na konkrétní situaci pak postava buď provádí všechny hody (když je např. ve scéně samotná), nebo se \underline{Výzvy} účastní více postav a každá hází na jinou dovednost. Poté, co jsou provedeny všechny hody, výsledek \underline{Výzvy} se vyhodnotí a hra se podle toho vyvine - je možné, že bude následovat další \underline{Výzva} (pokud se překážka přetvořila takovým způsobem, že je třeba ji překonat jinými dovednostmi), může dojít ke \underline{Střetu} (za zmiňovanými dveřmi bude hlavní záporák a ten se bude snažit odteleportovat se pryč) či dojde k vyhrocení a nastane \underline{Konflikt} (nepodaří se dveře vypáčit dřív, než do místnosti vtrhnou nepřátelé).

\subsection{Výhody ve výzvách}
\label{sec:výhody-výzvy}

Hraní akce \texttt{Vytváření výhody} ve \underline{Výzvách} je perfektně možné - prostě si hodíte na relevantní dovednost a podle výsledku pak můžete výhodu čerpat. Samozřejmě, vytvoření výhody se nepočítá do hodů na \texttt{Překonání} potřebných pro odehrání \underline{Výzvy}. A hlavně - v případě, že se výhodu nepodaří vytvořit, ovlivní to negativním způsobem nějaký jiný cíl - např. zvýší jeho obtížnost, poskytne volné vyvolání nepříteli apod.

\subsection{Příklad výzvy}
\label{sec:příklad-výzvy}

Příklad \underline{Výzvy} přebíráme z Fate Core, protože je prostě dobrý:

Tajemný Zird se pokouší dokončit Quirický zasvěcovací rituál, aby posvětil půdu zájezdního hostince a dodal mu tak ochranu Quirických božstev. Normálně by to nebylo moc zajímavé, jenomže se to pokouší dokončit dříve, než do hostince vtrhne horda bezduchých, masa-chtivých zombií, které omylem osvobodil v dřívější části příběhu.\\
Amanda ve scéně vidí několik různých prvků. Zaprvé tu je samotný rituál, zadruhé potřeba udržet hostinec zabarikádovaný a nakonec tu je potřeba udržet v klidu panikařící návštěvníky lokálu. To vyžaduje \textit{Učenost, Řemesla}a některou ze sociálních dovedností – Ryan se okamžitě rozhodne pro \textit{Vztahy}. Tudíž si Ryan musí hodit zvlášť na tři různé dovednosti, jednou pro každý prvek, který Amanda ve scéně zdůraznila. Pro každou z nich mu dá Dobrou (+3) opozici – chce mu dát fér šanci, ale zároveň chce nechat prostor pro různé možné výsledky.\\
S tím jsou připraveni začít. Ryan se z hluboka nadechne a prohlásí: „Ok, tak do toho.“, s čímž vezme kostky do ruky. Nejprve se rozhodne zajistit bezpečnost lokálu, takže si hodí na svá Dobrá (+3) \textit{Řemesla} a na kostkách mu padne 0. To znamená remízu, což mu umožní uspět za drobnou cenu. Amanda na to řekne: „Tak řekněme, že proti tobě dostanu posílení \asp{Zbrklá práce}, pokud budu potřebovat. Nakonec pracuješ dost ve spěchu. Ryan si povzdechne a přikývne a pak se pustí do druhé části výzvy, což je uklidnění hostů v lokále svými Dobrými (+3) \textit{Vztahy}. Hodí si a padne mu příšerných -3! Teď má možnost selhat nebo uspět za výraznou cenu. Rozhodne se uspět a nechá na Amandě, aby nějakou dobrou výraznou cenu vymyslela. Amanda o tom chvíli přemýšlí. Jakou cenu chtít za uklidňování vesničanů? Pak se ušklíbne. „Takže, je to sice víc o příběhu než o mechanikách ale víš co... používáš Vztahy, takže nejspíš všechny dost inspiruješ a motivuješ. Myslím, že jsi omylem přesvědčil pár farmářů a pasáků, že zombie ve skutečnosti nejsou hrozba a že můžou klidně jít ven a bojovat s nimi bez větších důsledků. Protože tvá magie je samozřejmě ochrání, že?“ „Ale aby to fungovalo, tak musí zůstat v hostinci!“ odvětí Ryan. Amanda se jen ušklíbne a Ryan si znovu povzdechne. „Ok, fajn. Pár lidí asi dostane naprosto zcestný nápad a nejspíš se půjdou nechat zabít. Už teď to slyším... Zirde, proč jsi nechal mého manžela zemřít? Ach jo.“ Amanda se ušklíbne ještě o něco víc. Ryan se pustí do poslední části výzvy – samotného rituálu, který bude sesílat svou Skvělou (+4) \textit{Učeností}. Amanda vyvolá posílení, které získala dříve, a prohlásí „Jo, rozhodně tě docela rozptylují ty zombie co se prolamují tou barikádou, kterou jsi jim předtím postavil do cesty. Hodně rozptylují.“ To posune obtížnost závěrečného hodu na Vynikající (+5). Ryan hodí +2 a dostane Fantastický (+6) výsledek, což mu stačí na to, aby uspěl bez ceny. Amanda přikývne a společně dokončí popis scény – Zird dokončuje rituál jen tak tak na čas a svatá moc Qiriku sestoupí na hostinec. Zombie které se zrovna sápaly průlomem jsou zasaženy svatou aurou a Zird si v tu chvíli velmi oddechne... dokud nezaslechne panický křik vesničanů před hostincem.\\
To ale bude až další scéna.


\section{Střety}
\label{sec:strety}


\section{Fyzické konflikty}
\label{sec:boj}

\subsection{Stres a následky}
\label{sec:stres-nasledky}

\section{Duševní a sociální konflikty}
\label{sec:dusevni-soc-konflikty}

\section{Magické konflikty}
\label{sec:magicke-konflikty}



\end{document}

%%% Local Variables:
%%% mode: LaTeX
%%% TeX-master: "../main"
%%% End:
