\documentclass[../main.tex]{subfiles}
\begin{document}

\chapter{Akce a výsledky}
\label{chap:akce}

Klíčem k univerzálnosti Fate Core je klasifikace veškerých činností (resp. těch, které potřebují mechanicky ošetřit) do čtyř akcí. To je skutečně velmi elegantní a umožňuje situace, které se jeví neporovnatelně řešit prakticky identicky: sociální konflikt jako fyzický, bitvu armád jako duel jednotlivců, vyšetřování na místě činu jako hledání informací v knihovně. V sekci o specialitách a povolání uvidíme, že tento přístup pravidla velmi usnadňuje a t06 sjednocuje speciální předměty či například magii.

\section{Čtyři akce}
\label{sec:4-akce}

Popišme nyní, jak vypadá základní mechanika akcí. Jestliže postava dělá něco, co je samozřejmé nebo u čehož není (ne)úspěch zajímavý, prostě se to stane - jezení, chození, spánek. Pokud postava dělá něco, co není samozřejmé či jí v tom někdo brání, je průběh následující

\begin{Pravidlo}[Jak používat akce]
\begin{enumerate}
\item Hráč popíše, čeho chce svojí akcí dosáhnout.
\item Vypravěč s hráčem stanoví jednu ze čtyř akcí popsaných níže, která záměr nejlépe vystihuje.
\item Hráč s vypravěčem vyberou, pomocí jaké dovednosti bude hráč akci vykonávat.
\item Vypravěč stanoví, zda-li se jedná o pasivní či aktivní opozici.
\item Hráč si hodí kostkami, připočte bonus za dovednost a svůj výsledek porovná s opozicí.
\item V případě, že nějaká ze stran chce, vyvolá či vynutí aspekty.
\item Podle celkového výsledku hráče se vyhodnotí, jakého výsledku jeho postava dosáhla, přičemž čtyři možné výsledky jsou vypsané níže.
\end{enumerate}
\end{Pravidlo}

Tento postup je univerzální a používá se ve všech případech, kdy je potřeba činnosti ošetřit mechanicky (hodit si kostkami). Zmiňovaný seznam akcí:

\begin{akce}
  \iconitem{prekonani-ikona}{\akc{Překonání:} Jakákoliv činnost s hlavním cílem něco překonat je akce \akc{Překonání}. Může se jednat o přeskočení propasti, vylezení na strmou skálu, rozražení či vypáčení dveří či třeba rozlousknutí šifry. V širším smyslu slova překonání je pak uvěřitelné, že sem spadají i činnosti jako vyhledávání (překonávání ``neznámého''), přesvědčování apod. Většina hodů mimo výzvu, střet nebo konflikt jsou de facto hody na překonání: pokus najít informátora na ulici, uplatit strážného, vyhandlovat lepší cenu, to všechno spadá pod \akc{Překonání}.

  Pomocí akce \akc{Překonání} se lze zbavovat existujících aspektů - vymanit se z \asp{Zalehnutí}, dostat se z \asp{V řetězech}, spřátelit se a odstranit tak aspekt \asp{Nedůvěřivý} a další.}
  \iconitem{vytvoreni-ikona}{\akc{Vytvoření výhody:}\sidenote{Pravidlové postihnutí vytváření výhody je velký posun oproti pravidlům M16.} Akce \akc{Vytvoření výhody} je základní způsob, jak zvýšit své šance na úspěch při všech ostatních akcích. Pomocí této akce se buď dají vytvořit nové aspekty (pronést plamennou řeč a stát se \asp{Miláčkem davu}, hodit písek do očí protivníkovi a dočasně ho \asp{Oslepit}, vylepšit povoz a učinit ho tak \asp{Opancéřovaným}), objevit již existující aspekty (po chvíli zkoumání získat \asp{Usvědčující důkaz}, rozhovorem zjistit, že strážnému \asp{Jde hlavně o peníze} atd.), nebo získat volná vyvolání na již existující aspekty (připravit si nečekaný útok v místnosti, kde je \asp{Všude tma}, trochu někoho vyprovokovat a využít, že se \asp{Lehko naštve} a další). }
  \iconitem{utok-ikona}{\akc{Útok:} Akce \akc{Útok} je přímočará - způsobit nepříteli zranění, fyzické či duševní. Může se jednat o seky mečem, údery a kopy, střelbu, nebo např. pokus někoho urazit, rozlítostnit, deprimovat, rozhodit. }
  \iconitem{obrana-ikona}{\akc{Obrana:} Po akci \akc{Obrana} sáhne hráč vždy, když chce jiným postavám ztěžovat pokusy vykonávat svoje akce \akc{Vytvoření výhody i Útok}. Všimněte si, že bránit se akci \akc{Překonání} technicky není možné; možné je ale tvořit aktivní opozici, jež musí ``útočník'' přehodit. \akc{Obrana} jediná akce, kterou může postava zahrát mimo své kolo a to v libovolném množství. \sidenote{Může se jevit, že je velmi silné, aby se postava mohla bránit kolikrát chce. Ve skutečnosti toto ale velmi urychluje a usnadňuje konflikty, neboť nedává prostor diskuzím o tom, zda-li dává smysl, aby se postava mohla bránit.}} 
\end{akce}

\section{Čtyři výsledky}
\label{sec:4-vysledky}

Podle toho, jak veliký je součet hod + úroveň dovednosti, nastane právě jedna z akcí:

\begin{Pravidlo}[Možné výsledky \akc{Akcí}]
  \begin{itemize}
	  \item neúspěch,
	  \item remíza,
	  \item úspěch,
	  \item úspěch se stylem.
  \end{itemize}
\end{Pravidlo}

\subsection{Úspěch}
\label{sec:uspech}

Úspěch znamená, že váš celkový výsledek je o 1-2 vyšší než celkový výsledek opozice (ať už pasivní nebo aktivní, tedy jiné postavy). V takovém případě se postavě prostě podaří její záměr.
\begin{akce}
  \iconitem{prekonani-ikona}{Postava překoná překážku či odstraní existující aspekt.}
  \iconitem{vytvoreni-ikona}{Postavě se podaří vytvořit nový aspekt a získá na něj volné vyvolání/objeví existující aspekt a získá na něj volné vyvolání/získá volné vyvolání na stávající aspekt.}
  \iconitem{utok-ikona}{Postava úspěšně protivníka zasáhne.}
  \iconitem{obrana-ikona}{Postavě se úspěšně podaří ubránit se (tj. překazit) akci nepřítele.}
\end{akce}

\subsection{Úspěch se stylem}
\label{sec:sestylem}

Úspěch se stylem je speciální případ úspěchu, kdy je celkový výsledek vyšší alespoň o 3. Představuje situaci, kdy se postavě její záměr povedl velmi dobře a získává díky tomu nějaký bonus navíc.

\begin{akce}
  \iconitem{prekonani-ikona}{Postava získává navíc posílení - něco jako \asp{Moment překvapení}, \asp{Rychlost}, \asp{Vítr v plachtách} apod.}
  \iconitem{vytvoreni-ikona}{Ve všech případech postava získává dvě volná vyvolání místo jednoho. Navíc, obě tato vyvolání může použít naráz (a tedy týž aspekt vyvolat vícekrát na jeden hod)}
  \iconitem{utok-ikona}{Útočník se může rozhodnout snížit svůj celkový výsledek a namísto toho si vzít posílení - \asp{Úsměv štěstěny, Amok} atd.}
  \iconitem{obrana-ikona}{Obránce navíc získává posílení - \asp{Vteřina odpočinku, Dokonalá obrana} aj.}
\end{akce}

\subsection{Remíza}
\label{sec:remiza}

Jestliže je celkový výsledek postavy stejný jako opozice, znamená to remízu. Ta typicky představuje úspěch za drobnou cenu.

\begin{akce}
  \iconitem{prekonani-ikona}{Překonání se podaří, ale za nějakou drobnou cenu - může se jednat o příběhový detail, např. jiná postava uvidí co se stalo, efekt nastane i pro někoho jiného atd.,  či nějakou drobnou komplikaci pro postavu - poškodí se ji vybavení, o nějaké přijde a další.}
  \iconitem{vytvoreni-ikona}{V případě, kdy se postava pokouší vytvořit nový aspekt či objevit stávající, získá místo aspektu jen posílení. Alternativně se může rozhodnout, že aspekt vytvoří (objeví), ale za drobnou cenu. Nicméně, pokud postava vytváří volná vyvolání na již existující aspekt,\textit{remíza znamená totéž co úspěch - získá volné vyvolání.}}
  \iconitem{utok-ikona}{Postavě se nepodaří způsobit žádné zranění, ale získává posílení.}\sidenote{Díky tomuto jsou konflikty svižnější i v případě vyrovnaných nepřátel.}
  \iconitem{obrana-ikona}{Nastává efekt remízy akce útočníka, před kterým se postava brání.}
\end{akce}

\subsection{Neúspěch}
\label{sec:neuspech}
\begin{akce}
  \iconitem{prekonani-ikona}{Akce se nepodařila. Alternativně, může se podařit za vysokou cenu - protivník získá nějaký aspekt, způsobí zranění, postava přijde o kus vybavení apod.}
  \iconitem{vytvoreni-ikona}{Žádný aspekt nebyl vytvořen/objeven. Alternativně, je možné aspekt vytvořit/objevit, nicméně volné vyvolání na něj získá někdo jiný, typicky protivník. Příkladem budiž snaha vytvořit výhodu \asp{Haraburdí všude na zemi}, z něhož bude čerpat protivník.}
  \iconitem{utok-ikona}{Útok se nezdařil.}
  \iconitem{obrana-ikona}{Obrana nezdařila, tj. útočník uspěl (případně uspěl se stylem).}
\end{akce}

Za povšimnutí stojí, že výsledky akce \akc{Obrana} zrcadlí výsledky akcí \akc{Vytvoření výhody} a \akc{Útok} (tj. těm, proti kterým se lze bránit). Znamená to prostě, že když obránce uspěje, útočník ne a vice versa. To je podstatné, aby nedošlo ke zvláštnímu skládání efektů.


\end{document}
%%% Local Variables:
%%% mode: LaTeX
%%% TeX-master: "../main"
%%% End:

