\documentclass[../main.tex]{subfiles}
\graphicspath{{\subfix{../graphics/}}}
\begin{document}

Klíčem k univerzálnosti Fate Core je klasifikace veškerých činností (resp. těch, které potřebují mechanicky ošetřit) do čtyř akcí. To je skutečně velmi elegantní a umožňuje situace, které se jeví neporovnatelně řešit prakticky identicky: sociální konflikt jako fyzický, bitvu armád jako duel jednotlivců, vyšetřování na místě činu jako hledání informací v knihovně. V sekci o specialitách a povolání uvidíme, že tento přístup velmi usnadňuje a sjednocuje též speciální předměty či například magii.

\section{Čtyři akce}
\label{sec:4-akce}

Ať se již postava rozhodne dělat cokoli, co vyžaduje hod kostkami, bude se jednat o jednu ze čtyř akcí:

\begin{akce}
  \iconitem{prekonani-ikona}{\texttt{Překonání:}} Jakákoliv činnost s hlavním cílem něco překonat je akce \texttt{Překonání}. Může se jednat o přeskočení propasti, vylezení na strmou skálu, rozražení či vypáčení dveří či třeba rozlousknutí šifry. V širším smyslu slova překonání je pak uvěřitelné, že sem spadají i činnosti jako vyhledávání (překonávání ``neznámého''), přesvědčování apod. Většina hodů mimo výzvu, střet nebo konflikt jsou de facto hody na překonání: pokus najít informátora na ulici, uplatit strážného, vyhandlovat lepší cenu, to jsou všechno hody na překonání.

  Pomocí akce \texttt{Překonání} se lze zbavovat existujících aspektů - vymanit se z \asp{Zalehnutí}, dostat se z \asp{V řetězech}, spřátelit se a odstranit tak aspekt \asp{Nedůvěřivý} a další.
  \iconitem{vytvoreni-ikona}{\texttt{Vytvoření výhody:}}
  \iconitem{utok-ikona}{\texttt{Útok:}}
  \iconitem{obrana-ikona}{\texttt{Obrana:}}
\end{akce}



\section{Čtyři výsledky}
\label{sec:4-vysledky}


\end{document}

%%% Local Variables:
%%% mode: LaTeX
%%% TeX-master: "../main"
%%% End:
