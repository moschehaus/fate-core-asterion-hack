\documentclass[../main.tex]{subfiles}
\graphicspath{{\subfix{../graphics/}}}
\begin{document}
Schopnosti postavy vystihují kromě aspektů zejména dovednosti. Vypovídají o tom, jak je která postava zdatná v oblastích relevantních pro příběh. Seznam dovedností se tedy může značně lišit podle povahy settingu a příběhu. \footnote{I v témže settingu je samozřejmě myslitelné používat jiné dovednosti. Jedná-li se například o dobrodružství odehrávající se ve škole šermu, je vhodné postihnout oblast boje více dovednostmi (rapír, fleret, šavle apod.). To stejné pro příběh o magii, kde jednotlivé obory magie mohou představovat oddělené dovednosti}.

Seznam níže je kompromis, na kterém jsme se jako družina shodli. Oproti pravidlům M16 se změny dočkaly zejména sociální a jiné ``soft'' dovednosti a boj. Konkrétně, v systému M16 je boj rozdělen na velké množství stylů: \textit{Rváč, Šermíř, Pirát, Těžkooděnec, Hrdlořez} a další, přičemž každý styl má přiděleno několik málo zbraní, které dokáže využívat. Tím však vzniká problém s balancováním (všechny styly neměly zdaleka stejně dobře využitelný set zbraní) a pro některé je toto dělení cela umělé (proč \textit{Hrdlořez} umí používat tesák a nikoli o 10 cm delší meč). Jistým řešením pak bylo stanovit dovednosti podle třídy zbraní: \textit{Jednoruční zbraně, Obouruční zbraně, Tyčové zbraně, Vrhací zbraně} etc. Nakonec jsme ovšem udělali ještě větší krok k jednoduchosti \footnote{V porovnání se sociálními dovednostmi, kdy v pravidlech M16 všeobjímající dovednost \textit{Zaujmutí pozornosti} zahrnovala \textit{Empatii, Kontakty, Provokaci i Vztahy}, je tento krok spíše obrovský. Skupině toto dělení ale vyhovuje.} a oblast boje rozdělili pouze na boj na dálku/na blízku a se zbraní/beze zbraně.

\begin{itemize}

\item \textit{Boj beze zbraně}
\item \textit{Boj na dálku}
\item \textit{Boj se zbraní}
\item \textit{Empatie}
\item \textit{Kondice}
\item \textit{Kontakty}
\item \textit{Klam}
\item \textit{Jezdectví}
\item \textit{Medicína}
\item \textit{Mobilita}
\item \textit{Pozornost}
\item \textit{Provokace}
\item \textit{Řemesla}
\item \textit{Skrývání}
\item \textit{Učenost}
\item \textit{Vůle}
\item \textit{Vyšetřování}
\item \textit{Vztahy}
\item \textit{Zlodějina}
\end{itemize}

\section{Boj beze zbraně}
\label{sec:bojbezezbrane}

\section{Boj na dálku}
\label{sec:bojnadalku}

\section{Boj se zbraní}
\label{sec:bojsezbrani}

\section{Empatie}
\label{sec:empatie}

\section{Kondice}
\label{sec:kondice}

\section{Kontakty}
\label{sec:kontakty}

\section{Klam}
\label{sec:klam}

\section{Jezdectví}
\label{sec:jezdectvi}

\section{Medicína}
\label{sec:medicina}

\section{Mobilita}
\label{sec:mobilita}

\section{Pozornost}
\label{sec:pozornost}

\section{Provokace}
\label{sec:provokace}

\section{Řemesla}
\label{sec:remesla}

\section{Skrývání}
\label{sec:skryvani}

\section{Učenost}
\label{sec:ucenost}

\section{Vůle}
\label{sec:vule}

\section{Vyšetřování}
\label{sec:vysetrovani}

\section{Vztahy}
\label{sec:vztahy}

\section{Zlodějina}
\label{sec:zlodejina}

\end{document}
